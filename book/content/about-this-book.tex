
C++作为一种支持多范式的编程语言,自计算机科学发展的早期阶段便已存在。它最初以“更好的C”(即带有类的C语言)面貌出现,如今已经演进为一种基于国际标准、持续积极开发的语言。截至本文撰写之时,C++20标准已经正式发布,而C++23标准正处于最终审查阶段,同时C++26标准的相关制定工作也在紧锣密鼓地进行中。C++的应用范围极其广泛,无论是规模微小还是庞大的项目,无论开发者经验深浅,都能在软件开发的大多数领域找到其身影。

据保守估计,当前正在生产环境中运行的C++代码量达到了数百亿行之巨。这些代码由来自不同背景的无数团队以及数百万名开发人员,在过去的几十年间共同书写完成。每位开发者对于何为正确的编程模型、优质的C++代码,以及适宜的开发方法都有着各自独特的见解。

这些代码在正确性、可读性、效率及性能等方面的表现参差不齐。大量代码遵循命令式编程范式;部分代码采用面向对象的方式构建;仅有少量代码体现了函数式编程的特点。其中一部分是由初学者编写,但更多的则出自专家之手,尽管这些专家未必都是C++语言设计与实现方面的权威。值得注意的是,相当大比例的现存代码在1998年首个C++标准化进程启动之前创建,而在那之后编写的代码虽然可能符合某一种特定的标准,但从外观上来看仍保留着前标准化时代的风格特征。

本书聚焦于使用C++进行程序开发过程中所遇到的一些关键问题。所谓现代C++,通常是指从C++11标准开始的一系列版本。鉴于前现代(或称经典)C++遗留下来的庞大代码库,深入理解这些问题变得尤为重要,而解决这些问题更是势在必行。通过阅读本书,开发者应当能够更加精准地识别并修正书中提及的问题。此外,对于那些未详细探讨的错误,开发者也将具备更强的识别能力和修复技巧。深入思考这些问题有助于开发人员,更好地把握语言特性和开发流程中的细微差异,从而提升代码质量。

\mySubsectionNoFile{}{适读人群}

无论代码的起源、遵循的标准、采用的范式以及技术成熟度如何,总需要有人积极地对其进行维护。这包括添加新功能、修改或增强现有功能、修复缺陷以及优化性能低下之处。尽管代码库中可能存在诸多问题,但程序依然能够运行并生成有意义的结果。

考虑到读者可能是C++开发领域的新人,其背景可能源于个人自学、大学教育或只接触过其他编程语言。对于大多数开发者而言,很少有机会从零开始构建一个全新的项目;因此,读者很可能会更多地参与到既有代码的工作当中。所以各位的职责将涉及为这些庞大的旧代码库开发新特性,并解决其中已有的问题。

各位这时面临的挑战在于,如何在这种环境下编写C++代码。实际上,C++很少是在完全自由无约束的环境中开发出来的,在这样的环境中开发者可以独自做出所有的决策。相反,真实的开发环境与单纯学习如何用C++编写代码有所不同,因为公司或者团队通常会制定各种指导原则、风格指南、命名规则以及其他约束。此外,现有的代码库已经确立了可接受的架构模式、命名约定、使用规范,以及针对常见问题的解决方案。不过,这些听起来的确与实际的C++编码工作没有什么直接关联……

\mySubsectionNoFile{}{本书的路线图}

第一部分所指出的错误展示了对C++语言及其特性的不当使用,而这些问题可以通过采用更新的语言特性和标准模板库(STL)功能来加以改善。具体而言,第2章深入探讨了类与数据类型的正确运用,特别强调了类的设计原则以及智能指针的应用。第3章则聚焦于编程过程中常见的陷阱,并探讨了可能需要进一步挖掘和利用的语言特性。第4章则详细介绍了C++语言近年来的重要变革,同时提出了几种能够有效管理和缓解常见问题的技术手段。

由于C++的历史渊源可以追溯到C语言,本书第二部分展示的错误反映了现代C++的一些深层次问题。许多此类问题可以通过引入现代技术和实践得到显著改善,从而为开发者带来巨大的便利和效率提升。第5章分析了一些在不知不觉间融入C++开发中的传统习惯与做法,尽管这些方法往往并非最优解。第6章则着重分析了早期C++编程中流行的一些不良实践,这些做法至今仍广泛存在于众多代码库中。

第三部分讨论的错误类型主要集中在遗留代码库中普遍存在的挑战,开发者往往难以直接借助现代技术手段进行彻底解决。尽管如此,我们仍然可以针对若干问题实施改进措施,进而逐步提高整体代码质量。第7章深入研究了良好类设计中存在的若干关键议题,以及构建坚固可靠的类时可能遭遇的各种麻烦。第8章继续围绕良好类设计展开讨论,揭示了那些可能意外导致设计方案失败的细节问题。第9章则集中探讨了类所提供的多种操作方式;若处理不当,可能会不经意间引发复杂的问题。第10章关注系统资源管理方面的内容,列举了几种因操作失误可能导致的问题。第11章则专注于函数调用及参数传递环节可能出现的各类状况。最后,第12章回顾了使用前现代C++进行常规编码时可能遇到的一系列问题。

\mySubsectionNoFile{}{源码库}

您可以从本书的 liveBook(在线)版本 \url{https://livebook.manning.com/bo
ok/100-c-plus-plus-mistakes-and-how-to-avoid-them} 获取可执行代码片段
。书中示例的完整代码可从 Manning 的网站下载:\url{https://www.manning.com/books/100-c-plus-plus-mistakes-and-how-to-avoid-them}。

\mySubsectionNoFile{}{liveBook论坛}

购买本书的读者可免费访问 Manning 的在线阅读平台 liveBook。使用 liveBook 的独可免费访问 Manning 的在线阅读平台 liveBook。使用 liveBook 的独家讨论功能,可以对书籍全局或特定章节或段落发表评论。可以做笔记、提出和回答技术问题以及获得作者和其他用户的帮助。要访问论坛,请访问 \url{https://livebook.manning.com/book/100-c-plus-plus-mistakes- and-how-to-avoid-them/discussion}。还可以在 \url{https://livebook.manning.com/discussion} 上了解有关 Manning 论坛和行为准则的更多信息。

Manning 对我们读者的承诺是提供一个平台,让读者之间以及读者和作者之间能够进行有意义的对话。这并不是对作者参与论坛的任何具体次数的承诺,作者对论坛的贡献仍然是自愿的(并且是无偿的)。我们建议您尝试向作者提出一些有挑战性的问题,以免他失去兴趣!只要这本书还在印刷中,就可以从出版商的网站上访问论坛和之前讨论的存档
