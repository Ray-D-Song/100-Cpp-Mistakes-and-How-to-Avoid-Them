此错误会影响正确性和可读性,并对效率产生轻微的负面影响。对效率的负面影响是故意的;C++ 风格的强制类型转换很大、很丑陋,而且很难编写,因此建议谨慎使用。

\mySamllsection{问题}

需要进行类型转换的主要情况有四种:

\begin{itemize}
\item
绕过const限制

\item
将底层数据重新解释为新类型

\item
将一种类型转换为另一种类型

\item
将基指针向下转换为派生指针
\end{itemize}

C 风格的强制转换可以完成所有四种行为,而不会导致编译问题,但成功的编译与有意义的编译不同。考虑清单 5.12 中的代码。第一部分围绕字符数组的 const 性进行强制转换,并允许数据修改。第二部分将整数值视为字符序列。第三部分是伪装成双指针的字符数组,这毫无意义。第四部分采用基类对象,并将其视为派生对象。

\filename{清单5.12 使用危险地C风格强制转换}

\begin{cpp}
struct B {
  virtual int compute() { return 0; }
};
struct D : public B {
  int n;
  D() : n(42) {}
  int compute() { return n; }
};

int main() {
  const char msg[] = "Hello, world";
  char* p = (char*)msg; // 1
  *(p+1) = 'a';
  std::cout << msg << '\n';

  int n = 0x2A2A2A2A;
  char* c = (char*)(&n); // 2
  for (int i = 0; i < sizeof(int); ++i)
    std::cout << *(c+i);
  std::cout << '\n';

  std::cout << *((double*)msg) << '\n'; // 3

  B* b = new B();
  D* d = (D*)b; // 4
  std::cout << d->compute() << '\n';

  return 0;
}
\end{cpp}

{\footnotesize
注释1:绕过字符数组的 const 属性

注释2:将整数视为字节集合

注释3:这没有任何意义,也可以通过编译

注释4:D 是 B,但 B 可能不是 D
}

\mySamllsection{分析}

第一个问题违反了数据类型的语义,应该很少发生。编译器默认不会发现错误,但可以配置为发现错误。这种“松散性”,可让开发者对数据做任何事情。

第二个问题很“聪明”,但通常有点牵强;虽然它能工作,但需要更好的类型和设计。

第三个错误令人厌恶,完全破坏语义——两种类型不兼容且不可翻译,但这段代码可以编译,导致未定义的行为。

最后,第四个问题可能有应用(很少),并且以这种方式完成时很危险。派生类中的虚函数没有正确调用,如果正确调用,则会访问不存在的实例变量——未定义的行为!

\mySamllsection{解决}

每种强制类型转换情况,都需要比 C 风格强制类型转换更有意义的东西替代。C 风格强制类型转换存在两个主要问题:首先,需要传达开发者的意图,其次,需要澄清一些转换。C++ 风格强制类型转换故意设计的很长,感觉很笨拙,但它们可以很好地记录意图,让开发人员三思而后行。通常,设计更好的代码可以消除强制类型转换的需要,但有时也必须要使用。

const\_cast 有意除去(或添加,如果不是常量)实体的 const 性。混合 C 和 C++ 代码时,这种强制类型转换通常是必要的 — 请负责任地使用它。

reinterpret\_cast 应谨慎使用。此强制类型转换告诉编译器,忽略它所知道的指向数据(其类型),并假装它是要强制类型转换的类型;这是一种“假装”的情况。编译器会屈从于开发者的愿望,但不承担任何责任。使用数字数据写入和读取二进制文件就是这种情况,强制类型转换很好地记录了意图。

static\_cast 将一种兼容的数据类型转换为另一种。编译器检查数据类型的兼容性;如果不兼容,会发出错误。在所有强制类型转换中,这是最有可能需要的。例如,使用此强制类型转换可以快速将双精度值截断为整数。

最后,dynamic\_cast 将基类指针或引用转换为派生类指针或引用。如果使用此技术,请确保至少有一个虚函数(没有虚函数就不要继承!);否则,没有规范的方法来存储用于向下转换的类类型。转换的结果在运行时进行检查,因此可以干净地编译但在运行时崩溃。兼容转换的结果是一个指针,而不兼容转换的结果是一个空指针,但开发人员应该极力避免采用这种方法。如果引用转换失败,则会引发异常。C++ 为此提供了更好的机制,尤其是具有多态行为的继承。

\filename{清单5.13 更谨慎、更负责地使用C++风格强制转换}

\begin{cpp}
struct B {
  virtual int compute() { return 0; }
};

struct D : public B {
  int n;
  D() : n(42) {}
  int compute() { return n; }
};

int main() {
  const char msg[] = "Hello, world";
  char* p = const_cast<char*>(msg); // 1
  *(p+1) = 'a';
  std::cout << msg << '\n';

  int n = 0x2A2A2A2A;
  char* c = reinterpret_cast<char*>(&n); // 2
  for (int i = 0; i < sizeof(int); ++i)
    std::cout << *(c+i);
  std::cout << '\n';

  // std::cout << static_cast<double*>(msg) << '\n'; // 3
  std::cout << static_cast<double>(n) << '\n'; // 4

  B* b = new B();
  D* d = dynamic_cast<D*>(b); // 5
  if (d)
    std::cout << d->compute() << '\n';
  else
    std::cout << "incompatible downcast\n";
  return 0;
}
\end{cpp}

{\footnotesize
注释1:假设这是正确的,并且意图可查

注释2:另一个可查的案例

注释3:编译器确定类型不兼容

注释4:合理的类型转换

注释5:如果兼容,则指针有效,否则为 NULL(nullptr)
}

如果使用不当,C 风格强制类型转换和 reinterpret\_cast 可能会导致未定义行为。如果发现强制类型转换是必要的,请考虑重新设计代码。如果必须使用强制类型转换,请仅在转换尽可能安全的情况下才这样做。

\mySamllsection{建议}

\begin{itemize}
\item
尽可能避免强制类型转换。

\item
将 C++ 代码与 C 函数接口时使用const\_cast。

\item
检查 dynamic\_cast 的结果以确保指针兼容。

\item
使用 reinterpret\_cast 之前请仔细考虑;并非所有架构都以相同的方式表示数据,因此转换结果可能不同;这样做很容易陷入未定义行为的陷阱。
\end{itemize}





