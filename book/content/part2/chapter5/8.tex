此错误会影响可读性和有效性;在某些情况下,性能也会受到影响。C 风格字符串是字符数组,其最后一个字符为零值,零值是标记值,表示字符串的结尾。

\mySamllsection{问题}

C 没有 class 或所谓的 smart objects 的概念。相反,程序代码往往充满了用于管理 dumb objects 的特定命令。每个开发人员都倾向于在很多地方处理 C 风格的字符串,经常重复代码和重复(通常通过复制和粘贴)功能。

清单 5.17 中的代码是众多可能示例之一。较旧的代码倾向于始终使用C 样式的字符串,因为它们很熟悉。许多函数都有 char* 参数,鼓励在整个代码库中使用它们。

考虑以下代码,开发人员计算 C 风格字符串中某个字符的出现次数。该代码很简洁,并且完全满足了需要。它看起来性能很好,是一个 O(n) 解决方案。谁会对此不满意呢?

\filename{清单5.17 简单的搜索和追加算法}

\begin{cpp}
int freq(const char* s, char k) {
  int count = 0;
  for (int i = 0; i < strlen(s); ++i) // 1
    if (s[i] == k)
      ++count;
  return count;
}

char* concat(const char* lhs, const char* rhs) {
  char* buffer = new char[strlen(lhs) + strlen(rhs)]; // 2
  strcpy(buffer, lhs);
  strcat(buffer, rhs);
  return buffer;
}

int main() {
  const char* msg = "Hello, world";
  char letter = 'l';
  std::cout << letter << " occurs " << freq(msg, letter) << " times\n";
  const char* msg2 = ", come on in!";
  std::cout << concat(msg, msg2) << '\n';
}
\end{cpp}

{\footnotesize
注释1:对字符串进行线性搜索

注释2:动态内存分配— —正确吗?
}

\mySamllsection{分析}

对于相对较短的字符串,这种方法效果很好,因为它很容易编写;但是,对于长字符串,其性能损失可能是不可接受的。使用分析器来显示需要更好的解决方案的地方。循环分析似乎显示线性搜索,实际上,它是线性 O(n)。算法是二次 O(n2)。为什么?C 样式字符串很笨;代码不能简单地向字符串询问信息以了解其长度。相反,每次请求时都必须计算大小。循环必须在循环的每次迭代中计算字符串的长度;也就是说,每次执行 strlen 本身都是线性 O(n),并且执行 n 次,从而得到二次解。值得庆幸的是,大多数编译器都会检测到这种情况并将strlen 函数移出循环。但是,不要相信编译器会为所有可能性找出答案。这是那些了解问题会导致简单解决方案的情况之一。但是,这种方法并不理想,因为它仍然专注于 C 样式字符串。清单 5.18 中的代码提高了性能,但并没有真正改进方法。

循环范围之外的任何其他代码都必须重新计算字符串的长度。如果字符串处理代码遍布整个代码库,那么在一个地方进行长度计算不会对其他地方有益。

管理字符数组也很困难。必须记住 strcpy 和 strcat 之间的区别,并记住如何处理空字符。需要确保目标缓冲区的长度至少为字符串的长度加上终止字符的长度。问题是开发人员必须记住并管理所有细节,大多数情况下这些细节都是正确的。然而,在极少数情况下,如果操作失误,就会出现未定义的行为。通常,如果目标太小,复制或附加似乎仍然有效,但会损坏数据。好的系统会崩溃(并非所有系统都是好的)。

freq 函数中的字符串长度计算不会使 concat 函数中的相同计算受益。此外,concat 函数中获得的动态内存未释放。这个事实在这里 可能看起来无害,但这个错误会影响正确性。

\filename{清单5.18 改进的搜索和追加算法}

\begin{cpp}
int freq(const char* s, char k) {
  int count = 0;
  int len = strlen(s); // 1
  for (int i = 0; i < len; ++i)
    if (s[i] == k)
      ++count;
  return count;
}
char* concat(const char* lhs, const char* rhs) {
  char* buffer = new char[strlen(lhs) + strlen(rhs) + 1]; // 2
  strcpy(buffer, lhs);
  strcat(buffer, rhs);
  return buffer;
}

int main() {
  const char* msg = "Hello, world";
  char letter = 'l';
  std::cout << letter << " occurs " << freq(msg, letter) << " times\n";
  const char* msg2 = ", come on in!";
  std::cout << concat(msg, msg2) << '\n';
  delete [] msg;
}
\end{cpp}

{\footnotesize
注释1:计算一次该函数的长度

注释2:确保空终止符有足够的空间
}

\mySamllsection{解决}

C++ 比 C 风格的字符串有了显著的改进。它们的问题非常严重,以至于自 C++ 早期以来提供的首批主要用户编写的类之一就是 string 类。C++ 风格的 string 是一个智能对象;开发人员可以向它提问并获得答案。以下代码利用了这一事实来查询 string 的长度。

\filename{清单5.19 使用c++字符串解决检测到的问题}

\begin{cpp}
int freq(const std::string& s, char k) {
  int count = 0;
  for (int i = 0; i < s.length(); ++i) // 1
    if (s[i] == k)
      ++count;
  return count;
}
std::string concat(const std::string& lhs, const std::string& rhs) {
  return lhs + rhs; // 2
}

int main() {
  std::string msg = "Hello, world";
  char letter = 'l';
  std::cout << letter << " occurs " << freq(msg, letter) << " times\n";
  std::string msg2 = ", come on in!";
  std::cout << concat(msg, msg2) << '\n'; // 3
}
\end{cpp}

{\footnotesize
注释1:长度的恒定时间查询

注释2:开发人员不必关心数据移动的长度

注释3:自动处理为临时对象获取的内存
}

该解决方案可以更好,但在改进之前的代码的同时,也符合之前的代码。使用字符串长度初始化的局部变量更好,但这种方法展示了智能对象的好处。将字符串长度查询留在循环中表明,尽管该函数被调用了 n 次,但循环的计算成本为 O(n)。由于查询长度不会触发长度计算,因此保留了此时间特性。字符串在构造时确定其长度并将结果存储在实例变量中,因此长度查询在 O(1) 时间内完成。

通过让对象本身确定所需的内存,两个字符串的连接解决了管理结果字符串长度的难题。开发人员不需要知道数据是如何处理的——它是否使用空终止符?此外,不需要删除动态内存,因为字符串使用 RAI I 模式管理其数据。

\mySamllsection{建议}

\begin{itemize}
\item
学习在字符序列使用时使用 C++ 样式的字符串;字符串类经过高度优化且易于使用。

\item
尽可能消除 C 样式的字符串,并引入 C++ 样式的字符串来代替它们。

\item
字符串自动处理动态内存;当字符串超出范围时,将调用其析构函数。
\end{itemize}
