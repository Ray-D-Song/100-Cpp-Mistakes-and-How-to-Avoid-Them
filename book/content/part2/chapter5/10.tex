这个错误与有效性和可读性有关;它会影响正确性,如果是这样,会产生不利影响。C 提供了一个内置数据容器,被广泛使用。数组是可以用一个名称寻址并通过索引区分的元素序列。

\mySamllsection{问题}

许多编程问题都需要实体集合,无论是内置的还是用户编写的。如果实体是用户编写的并存储在数组中,则必须存在默认构造函数。在某些情况下,此构造函数是有意义的,但在许多情况下,它没有意义。
毫无意义的默认构造函数表明需要更正或完成设计。但是,使用数组不是一种选择。

创建静态数组时,必须在编译时知道元素的数量。许多问题没有清楚地表达必须处理的元素数量,这意味着选择的大小是任意的。如果开发人员的猜测太大,就会浪费空间;如果猜测太小,这通常会导致崩溃,甚至更糟。这里一个很好的选择是使用动态数组,因为通常,代码可以在需要容器之前确定元素的数量。开发人员必须记住管理容器 的内存——哦,动态数组也需要默认构造函数。

无论基于数组的容器有多少元素,添加过多的元素都是可能的。使用动态数组允许开发人员通过分配较大的数组并复制值来管理过短的数组。如果该数组被其他实体引用,开发人员通常会感到意外,因为他们不知道分配的内存已被转移到其脚下。

最后,尝试删除不再需要或有效的数组元素。使用数组的所有代码都必须知道如何确定无效(或已删除)元素。这个问题会将知识分散到使用该数组的任何函数中,并使工作量重复。由于数组是哑对象,因此无法询问它们哪些元素有效。开发人员必须确定一种方案,通过某种方式将各个元素标记为无效——读者可能不清楚此代码的含义或为什么它会与某些元素发生冲突。以下代码演示了静态数组的简单情况,该数组强制 Person 类包含伪默认构造函数。

\filename{清单5.22 静态数组强制使用默认构造函数}

\begin{cpp}
struct Person {
  std::string name;
  int age;
  Person(const std::string& n, int a) : name(n), age(a) {}
  Person() : name("", 0) {}
};
int main() {
  Person people[3]; // 1
  Person suzy("Susan", 25);
  people[0] = suzy;
  Person anna("Annette", 32);
  people[2] = anna;
  std::cout << people[1].name << '\n';

  int count = 5; // assume this is computed
  Person* others = new Person[count]; // 1
  return 0;
}
\end{cpp}

{\footnotesize
注释1:每个元素通过调用默认构造函数来初始化
}

\mySamllsection{分析}

第一个数组是静态的;元素数量必须在编译时知道。此示例表明元素数量太少。尝试了另一种方法,即在确定元素数量后分配动态数组。在这两种情况下,都需要默认构造函数,但实例变量没有合理的默认值。必须选择一些东西,无论其含义如何。如果五个元素中有四个被有意义地初始化,则缺少的元素仍将是“合法的” Person 元素;但是,它将包含非法信息,因为其数据不代表任何人。

\mySamllsection{解决}

用向量代替数组几乎总是正确的选择。 Stroustrup 博士(C++ 的发明者)推荐这种方法,所以我对此深信不疑。即使他没有这样说过,使用向量的理由本身也足够令人信服。

首先,向量是动态的;因此,选择太少的元素是不可能的。使用向量 在某些方面类似于数组。数组用于实现称为 backing array 的向量。当支持数组填满并且没有多余的元素时,就会发生神奇的事情。添加新元素时不会发生崩溃。向量将分配一个新的、更大的支持数组;从前一个数组复制元素;并将新元素添加到第一个未使用的索引。开发人员只需享受轻松使用的乐趣,并且不需要用户编写内存管理代码即可获得此结果。请参阅第三个好处,以了解过于频繁地扩展支持数组的注意事项。请不要假设向量不会被滥用或使用而不会产生后果;错误的代码可能会导致异常 - 确实如此!

其次,可以将元素推送到向量中,而无需考虑特定的索引值。如果索引有效,则可以添加元素。对于开发人员而言,删除元素就是删除该元素;无需设置尴尬的标志来指示无效元素。

第三,如果对后备数组进行多次重新分配,则可能会出现严重的性能下降,这是由于元素被复制的次数所致。最好估算所需元素的数量并调用 reserve 函数。reserve 函数会为该数量的元素分配足够的内存。选择正确的值意味着不会重新分配或浪费空间遇到这种情况时,即使猜测错误,向量也会正确运行。如果猜测太低,则会发生重新分配;如果猜测太高,则会浪费空间。但是,开发人员不必管理内存或担心向量的机制。

以下代码改进了数组实现,使编码更加流畅,错误更少。开发人员不 会白白得到任何东西;只要明智地使用,向量就可以接近这个不可能的梦想。

\filename{清单5.23 使用vector替换笨拙的数组}

\begin{cpp}
struct Person {
  std::string name;
  int age;
  Person(const std::string& n, int a) : name(n), age(a) {}
};

int main() {
  std::vector<Person> people;
  Person suzy("Susan", 25);
  people.push_back(suzy);
  Person anna("Annette", 32);
  people.push_back(anna);
  std::cout << people[people.size()-1].name << '\n';

  int count = 5; // assume this is computed
  std::vector<Person> others;
  others.reserve(count);
  return 0;
}
\end{cpp}

\mySamllsection{建议}

\begin{itemize}
\item
在大多数情况下,用向量代替数组。

\item
仔细阅读并理解使用vector 的空间和时间含义;在许多情况下,它们不会出现问题,并且具有出色的性能。

\item
vector有许多可用的方法;研究它们以了解它们的威力和可能性。
\end{itemize}













