此错误分别影响正确性、可读性和有效性。C 要求在函数顶部声明所有变量。

\mySamllsection{问题}

许多语言要求在函数顶部声明所有变量,有时也定义所有变量。这是对编译器的让步,而不是开发人员的需求,这种做法存在一些问题。首先,当所有变量都在函数顶部声明时,读者和开发人员必须参考顶部来查看名称、记住它们的含义,并查看变量的使用位置。这种方法需要一定的认知负荷,这可能会很繁重。其次,开发人员必须在使用变量之前确定变量的初始化;方便起见,此初始化可能是一个简单的值(通常为0),但不一定反映后面的代码所依赖的正确值。但如果开发者花时间正确地初始化该值,读者可能要到很晚才明白为什么。维护开发者可能需要稍后更改此值,从而引入问题。最后,这些初始化值可能经常看起来像是神奇的数字;为什么 1 应该被初始化为 0、1 或(显然)任意值?

清单 5.1 演示了其中的一些问题。开发人员需要初始化 max 变量,但由于“膝跳”值为 0,由于初始化不当,它会计算出错误的答案。此外,pos 变量在循环之前的代码中使用,并保留了一些值。原始代码假设pos 将在(某处)初始化并在循环中正确使用。后来的开发人员忽略了这个未记录的假设,并将“完美无缺”的变量用于其他目的。

\filename{清单5.1 在函数的顶部声明变量}

\begin{cpp}
int maximum(const std::vector<int>& values) {
  int max = 0; // 1
  int pos; // 2

  // assume code here that uses pos ...
  pos = 1; // 3
  // assume more code here...
  for (; pos < values.size(); ++pos) // 4
    if (values[pos] > max)
      max = values[pos];
  return max;
}

int main() {
  std::vector<int> values;
  values.push_back(1);
  values.push_back(-2);
  values.push_back(-3);
  std::cout << maximum(values) << '\n';
  return 0;
}
\end{cpp}

{\footnotesize
注释1:变量的初始化选择不当

注释2:没有初始化;假定将在稍后发生

注释3:添加的一些计算的结束值

注释4:使用 pos 时假定其值有意义
}

\mySamllsection{分析}

变量声明与变量使用之间的分离,使得变量在预期使用之前会发生奇怪的事情。pos 变量对于一些计算来说是一个不错的选择,但它处于不适合循环的状态。max 变量的初始化使用了一个通常正确且合适的值,但每个问题都可以使用这个初始值。由于vector中所有测试的值都是负数,所以没有一个大于错误初始化的值。这个错误引入了一个容器中未包含的外部值作为其最大值。

\mySamllsection{解决}

考虑在使用变量之前声明它——尽可能缩短声明、定义和使用之间的距离。在某些情况下(for 循环),可以在结构范围内声明变量,确保它仅在那里可见(现代 C++ 允许在 if 语句中这样做)。

初始化变量是一项有趣的练习。许多学生倾向于始终使用0值来初始化变量。大多数情况下,这非常合适,但并非全部。这部分将0视为错误的问题将导致微妙的问题。清单 5.2 中的 maximum 函数通过在一组负值中搜索最大值来演示这一点。出于某种原因,太多开发人员在编写代码时只考虑正值。整数和实数有一个令人讨厌的习惯,就是经常为负数,不容忽视。在扫描值集合之前,一种简单的初始化方法是将第一个元素的值复制到变量,并比较其他值。这种方法可确保使用实际数据,而不是开发人员假设的值初始化变量。如前所述,使用开发人员选择的值初始化,可能会使用数据集中不存在的值,从而导致错误。

\filename{清单5.2 需要时声明变量}

\begin{cpp}
int maximum(const std::vector<int>& values) {
  // assume code here...
  int max = values[0]; // 1
  for (int pos = 1; pos < values.size(); ++pos) // 2
    if (values[pos] > max)
      max = values[pos];
  return max;
}

int main() {
  std::vector<int> values;
  values.push_back(1);
  values.push_back(-2);
  values.push_back(-3);
  std::cout << maximum(values) << '\n';
  return 0;
}
\end{cpp}

{\footnotesize
注释1:使用集合值之一初始化变量

注释2:将循环控制变量的范围限制在循环内
}

记住变量的含义和值,对于代码的可读性和推理至关重要。在需要的地方准确声明变量,并在使用前初始化它们,可以减轻读者的认知负担。

\mySamllsection{建议}

\begin{itemize}
\item
限制每个变量的范围;声明后立即用有意义的值进行初始化。

\item
可在 for 循环范围和允许这种方法的其他构造中声明变量(现代 C++ 增加了更多机会)。
\end{itemize}












