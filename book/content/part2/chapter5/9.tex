这个错误与正确性有关。使用 exit 函数可能会导致有限或动态资源泄漏,从而影响系统的稳定性。

\mySamllsection{问题}

假设开发人员需要遵循一些可疑的要求。如果发现有员工解雇,程序必须在工资单处理期间立即崩溃。开发人员理解这一要求,并记得以前的 C 程序会立即使用 exit 调用来终止;因此,开发人员决定在 C++ 程序中使用此技术。后来,在出现一些奇怪的问题后,系统开发者会询问开发人员,为什么这个程序会导致系统稳定性问题。

\filename{清单5.20 使用exit结束程序}

\begin{cpp}
class Connection {
private:
  int conn;
public:
  Connection(const std::string& name) : conn(0) {
    conn = 1; // assume: database connection resource is returned
  }
  ~Connection() {
    if (conn)
      conn = 0; // assume: destroys database connection resource
    std::cerr << "Connection destroyed\n";
  }
};

struct Employee {
  bool isTerminated() { return true; }
  double computePay() { return 42.0; }
};

int main() {
  Connection c("payroll");
  Employee emp;
  if (emp.isTerminated())
    exit(-1); // 1
  std::cout << emp.computePay() << '\n';
  return 0;
}
\end{cpp}

{\footnotesize
注释1:程序必须在此终止
}

\mySamllsection{分析}

程序分配数据库连接、处理员工,并在发现解雇的人员时崩溃。清理数据库后重新运行程序,可能又崩溃了几次。处理结束时,数据库清理干净,程序运行完成,每个人都拿到了工资——快乐的一天!但为什么会出现系统稳定性问题?问题出在 Connection 类,其构造函数访问了有限的资源,而析构函数是为了释放这些资源而设计的。正常情况下,此代码可以正常工作。但问题会在特殊情况下出现,exit 调用强制程序立即终止。Connection 对象将不会进行清理,所以其析构函数无法运行。

动态资源已分配,但没有使用——典型的资源泄漏。

\mySamllsection{解决}

解决这个问题需要两个步骤。首先,不要使用 exit 调用;而是抛出异常。其次,将最外层的代码包装在一个通用的 try/catch 块中,其异常操作是重新抛出异常。当然,catch 子句可以更具体,但在解决这个问题时,代码的智能是有限的。

这种方法可以防止资源泄漏,因为析构函数有机会运行。当Connection 对象超出范围时,其析构函数会将数据库连接返回到池中,从而防止分配但未使用的资源。

代码很笨拙,但在许多情况下,开发人员发现自己需要改进整体结构或实现。这些情况下,“廉价”解决方案是必要的,即使设计很差。开发人员很少能负担得起,改进旧代码所需的时间和预算——这是一个在他们可以适应的时间和地点,并在有限的领域寻求改进的问题。现实有时很残酷。

\filename{清单5.21 使用异常终止程序}

\begin{cpp}
class Connection {
private:
  int conn;
public:
  Connection(const std::string& name) : conn(0) {
    conn = 1; // assume: database connection resource is allocated
  }
  ~Connection() {
    if (conn)
      conn = 0; // assume: returns database connection to pool
    std::cerr << "Connection destroyed\n";
  }
};

struct Employee {
  bool isTerminated() { return true; }
  double computePay() { return 42.0; }
};

struct TerminatedEmployee {
  std::string message;
  TerminatedEmployee(const std::string& msg) : message(msg) {}
};

int main() {
  try {
    Connection c("payroll");
    Employee emp;
    if (emp.isTerminated())
    throw TerminatedEmployee("Employee was terminated");
    std::cout << emp.computePay() << '\n';
  } catch (...) {
    throw; // 1
  }
  return 0;
}
\end{cpp}

{\footnotesize
注释1:此时终止程序
}

某些情况下,有必要提前退出程序。需要时,请确保对 exit 进行编码,以便在退出之前调用所有析构函数。只有在不需要堆栈展开(调试)的情况下,程序才应该 exit ,而不清理资源。请注意不要泄漏系统资源,例如:套接字、数据库连接和类似的实体或对象。

\mySamllsection{建议}

\begin{itemize}
\item
避免调用 exit 。

\item
抛出异常,一般性地(或具体地)捕获它,然后重新抛出,以便让析构函数有机会运行。

\item
尽其所能地改进;很可能不会有很好的机会来改进大量代码。
\end{itemize}









