这个错误主要与性能有关。操作系统通常会缓冲输入和输出数据,以尽量减少输入/输出 (I/O) 操作。误解缓冲可能会对性能产生重大影响。

\mySamllsection{问题}

下面的代码很简单,循环迭代一千次,每次输出循环控制变量的值。最后输出终止消息——很简单,对吧?

\filename{清单6.3 不加区分地使用std::endl}

\begin{cpp}
int main() {
  for (int i = 0; i < 1000; ++i)
    std::cout << i << std::endl; // 1
  std::cout << "finished!" << std::endl;
  return 0;
}
\end{cpp}

{\footnotesize
注释1:输出到 std::cout 缓冲区并刷新到设备
}

\mySamllsection{分析}

这里没有什么花招!代码按预期工作,只是性能可能比预期的更差。简单的问题是 std::endl 输出行尾字符并刷新输出缓冲区。

更少的 I/O 操作等同于更高的吞吐量。std::cout 流是使用更少 I/O 操作的缓冲流。其缓冲区会累积输出值和换行符,直到填满为止,然后将缓冲区作为一个大块刷新。使用 std::endl 会排除这种缓冲优化,因为无论缓冲区包含多少数据,它都会刷新缓冲区。在多个系统上,各种优化可以抵消终端输出的这种明显差异,但不能抵消文件输出的这种明显差异,因此请考虑对此分析进行更细致的理解。

\mySamllsection{解决}

如果将 std::endl 更改为 \verb|'\n'|,缓冲将按预期工作。请记住,在某些情况下,应刷新部分填充的缓冲区以完成 I/O 操作。清单 6.4 中的代码优化了缓冲区填充,直到输出所有值和换行符。然后,终止消息刷新缓冲区以确保任何缓冲数据都被写出,从而干净地完成输出操作。在这种情况下,程序在刷新后终止,因此这里并不是严格需要的; 通常以这种方式实现它是一种很好的做法。

\filename{清单6.4 谨慎地使用std::endl}

\begin{cpp}
int main() {
  for (int i = 0; i < 1000; ++i)
    std::cout << i << '\n'; // 1
  std::cout << "finished!" << std::endl; // 2
  return 0;
}
\end{cpp}

{\footnotesize
注释1:输出到 std::cout 缓冲区

注释2:输出并将缓冲区刷新到设备
}

在几乎所有情况下都使用缓冲输出,因为它可以与操作系统配合使用以提供最佳性能。如果要在不使用缓冲的情况下输出数据,请考虑使用 std::cerr 流。

\mySamllsection{建议}

\begin{itemize}
\item
尽可能将 std::endl 更改为 \verb|\n|,以防止不必要的缓冲区刷新。

\item
当需要刷新输出缓冲区时,请使用 std::endl。

\item
如果输出大量数据,则对除最后一个操作之外的所有操作使用 \verb|\n|;对最后一个操作使用 std::endl。

\item
考虑使用 std::cerr 进行无缓冲输出。
\end{itemize}






