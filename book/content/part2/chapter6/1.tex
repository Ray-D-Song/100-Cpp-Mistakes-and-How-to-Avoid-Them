这个错误主要关注有效性,其次是可读性。当只有 scanf 和 printf 类型的函数时,必须使用。然而,它们很难正确使用,而且阅读起来很别扭。

\mySamllsection{问题}

许多问题需要从键盘或文件输入文本数据,并将其转换为各种数据类型。scanf 和 printf 函数使用格式说明符来确定数据类型。此说明符通常是一个字符,以符号形式表示数据类型。如果使用了错误的说明符,则会导致未定义行为。某些情况下,错误可以容忍;其他情况下,会发生崩溃,甚至更糟。如果错误的说明符不会导致崩溃,则数据很可能是错误的。

只有当开发人员了解格式说明符的可能性时,才有可能读取格式说明符。清单 6.1 对 sscanf(字符串扫描)函数使用了复杂的格式说明符。有时,该说明符比正则表达式更难读,读者必须停下来,在脑海中解析各个部分。如果说明符不为人知,开发人员必须通过搜索互联网,或其他资源来了解其含义和正确用法。编程的许多方面都是如此,而不仅仅是格式说明符。关键是让我们停下阅读,并进行一些研究的事情,都是改进的机会。格式说明符就是这样的机会。

\filename{清单6.1 使用复杂的格式说明符确定转换}

\begin{cpp}
int main() {
  const char* str = "3.14159 042 boxes .3";
  double pi;
  int cats;
  int mice;
  char buffer[5];

  int count = sscanf(str, "%lf%*c%i%s%d", &pi, &cats,
    buffer, &mice); // 1
  if (count != 4) // 2
    std::cout << "error reading value " << count+1 << '\n';

  printf("%f being eaten by %d cats in %s along with %d mice\n",
    pi, cats, mice, mice); // 3
  return 0;
}
\end{cpp}

{\footnotesize
注释1:所有输入规范都在一次操作中应用;这……正确吗?

注释2:开发者必须确定正确的数字

注释3:文本、规范和变量混合在一起很别扭——其中有一个错误
}

\mySamllsection{分析}

格式说明符有一些奇怪的地方。double 变量上可以使用 d 字符进行扫描和转换,但这样做会导致错误。此处的 d 字符指的是decimal,而不是 double。f 字符表示浮点值,直观上是 float 数据类型。要将文本读入 double 变量,需要使用 long 浮点说明符 lf。我们当然希望看到更直观的东西。

另一个奇怪的是,第一个整数变量使用 i 说明符(表示整数?),而第二个整数变量使用 d 说明符。第一个整数变量的输出结果是 34,而不是 42。

只要理解了 d 表示以 10 为基数的整数,谜团就解开了。相比之下,i 表示一个整数,其基数由数据的前导输入字符决定(0 表示八进制,0 x 表示十六进制,否则为十进制),前导0表示文本是八进制值,这些细节很容易被忘记。

最后,buffer 变量表示 C 样式字符串并使用 s 说明符 — 一个具有直观含义的说明符了!开发人员知道容器是五个字符,因此为数组分配了该数量的元素。很好,但scanf 类型函数将相关字符传输到目标中并添加终止空字符。如果使用 c 说明符,则不会添加该空终止符。这些差异必须记住,或每次都查找都进行。

printf 输出也存在一些问题。某些情况下,其说明符与 scanf 的说明符不同。此示例中,double 变量使用 f 说明符,但没有使用 lf,就像在 scanf 中一样。可以理解将输出文本与说明符混合在一个字符串中,但它确实引入了一些阅读的复杂性,因为变量是在格式字符串之后列出的。从左到右阅读,还需要在字符串和变量之间来回跳转(并且必须严格保持它们的顺序)。

最后,如果说明符和变量类型之间出现变量不匹配,会发生什么情况?清单 6.1 中的示例两次错误指定了 mice;第一次应该是 buffer。当使用整数变量作为 C 风格字符串?肯定不会有什么好事发生,所以未定义行为才是正解。我的系统发出了段错误,提醒我有些东西坏了,但不会继续执行。

这些函数系列不是类型安全的,使用起来存在风险。很容易出现预期格式说明符,与实际数据类型不匹配的情况。鉴于这些有限的错误检测能力,有意义的恢复代码非常复杂。

\mySamllsection{解决}

C 语言为提供了 scanf 和 printf 功能,但我们使用 C++ 进行编程,它有更好的选项。将字符数据移入或移出程序时,应使用流。流提供了非常有用的插入和提取操作符(分别为 <{}< 和>{}>)。这些操作符确定如何将文本转换为正确的数据类型(假设输入数据与数据类型一致),并使开发人员不再需要数据类型说明符。

清单 6.2 中的错误编码(其中 mice 输出两次)不会导致运行时,错误或歪曲数据——输出不正确,并且没有像清单 6.1 中那样从一种类型到另一种类型的错误转换。如果数据一致,则将文本数据提取到各种变量中,可以正常工作;但当数据不一致时会失败。scanf 的问题在于输入和转换是一次性发生的;如果发生错误,开发人员必须从返回值中确定哪个转换失败了。清单 6.1 中的检测逻辑更加比清单 6.2 更简单,但至少第二个版本为开发人员提供了对错误检测的更细粒度的控制。

\filename{清单6.2 使用类型确定转换}

\begin{cpp}
int main() {
  std::istringstream str("3.14159 042 boxes .3");
  double pi;
  int cats;
  int mice;
  std::string buffer;

  str >> pi; // 1
  if (str.fail())
    std::cout << "error reading value 1\n";
  str >> cats;
  if (str.fail())
    std::cout << "error reading value 2\n";
  str >> buffer;
  if (str.fail())
    std::cout << "error reading value 3\n";
  str >> mice;
  if (str.fail())
    std::cout << "error reading value 4\n";

  std::cout << pi << " being eaten by " << cats << " cats in " << mice
    << " along with " << mice << " mice\n"; // 2
  return 0;
}
\end{cpp}

{\footnotesize
注释1:输入和转换工作正常或检查fail()函数

注释2:转换基于实际数据
}

整数值 042 正确转换为值 42;前导0不会影响其含义。如果数据确实是八进制,则可以将输入流设置为 8 进制以进行输入操作。此代码将输入并将 cats 转换为八进制:

\begin{cpp}
str >> std::setbase(8) >> cats;
\end{cpp}

\mySamllsection{建议}

\begin{itemize}
\item
尽可能用更适合流式传输的提取和插入操作符,替换 scanf 和printf 调用。

\item
避免使用复杂的输入和输出格式字符串;它们即难用,又难阅读。
\end{itemize}
