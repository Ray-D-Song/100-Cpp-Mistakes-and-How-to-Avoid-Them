这个问题主要集中在正确性上,其次是有效性。内存分配自 C 语言以来已经取得了长足的进步,但这并不是每个人都能接受这些变化。

\mySamllsection{问题}

C 提供 malloc 和 free 内存分配和释放操作符对。与使用 new 和delete 的 C++ 动态资源分配一样,使用 malloc 获得的任何内容都应使用 free 释放。这听起来很简单,很多情况下也确实如此。但 C 和 C++ 代码可能会使这个“简单”的过程变得复杂,尤其是当资源所有权发生变化时。

清单 6.5 展示了 malloc 代码中出现的两个常见问题:

\begin{itemize}
\item
大小计算不正确

\item
无法初始化获取的内存
\end{itemize}

\filename{清单6.5 不规范地使用malloc和free}

\begin{cpp}
struct Buffer {
  char* str;
  Buffer(int size) : str(new char[size+1]) { str[0] = '\0'; }
  ~Buffer() { free(str); }
};

int main() {
  double* val = (double*)malloc(sizeof(int)); // 1
  std::cout << val << '\n';
  Buffer* buf = (Buffer*)malloc(sizeof(Buffer)); // 2
  std::cout << buf->str << ", size " << strlen(buf->str) << '\n';
  free(buf);
  return 0;
}
\end{cpp}

{\footnotesize
注释1:睡眼惺忪的错别字

注释2:动态对象空间已分配,但未初始化
}

\mySamllsection{分析}

除了前面提到的两个 malloc 问题之外,上述代码至少还有三个其他错误。首先,获取了一个动态内存块来保存 double 值。由于熬夜和咖啡因水平下降,开发人员错误地请求了足够的内存来存储整数(通常是 double 值的一半大小)。编译器对这种不匹配保持沉默,这是一个典型的类型安全错误。对存储在这个太短的块中的数据的访问,都将访问其边界之外的数据。

第二个问题出现在创建 Buffer 对象时。这次,分配的大小是正确的,但缺少 char* 变量的初始化。构造函数应该处理获取动态内存块,以保存字符数据,并确保将终止字符写入第一个位置。然而,这种情况并未发生过。调用 malloc 时,不会执行构造函数,也不会初始化。

第三,val 变量没有相应的释放。内存的动态分配在主函数退出时泄漏;没有析构函数或其他管理实体在监视内存。在长期运行的程序中,几次这样的泄漏可能会导致严重的问题。

第四个问题是,虽然 buf 对象已正确释放,但该对象获取的动态内存并未释放。诚然,从未分配,但更好的代码会进行分配,只会在封闭实体释放时泄漏。free 调用从不调用析构函数,导致动态资源都处于困境之中。

最后,通过 new 分配内存并调用构造函数获得的 Buffer 对象的动态内存,与通过调用 free 释放不匹配,这种不匹配的行为没有定义。使用复制和赋值(和移动)语义将缓解其中许多问题。

\mySamllsection{解决}

new 和 delete 操作符并非只是为了互逆而开发,malloc 和free 的不足之处足以说明需要一种新的方法。

如清单 6.6 所示,第一个问题由 new 处理,以确保获得的对象可以接收变量的正确类型——不会出现类型安全错误。第二个问题通过确保每次分配时,都调用构造函数来处理。正确设计的代码将实现 RAII 模式,以确保分配与正确的释放配对。第三,仍然有可能无法删除new 对象;开发人员必须确保其配对。如果设计正确(RAII 模式),则在调用析构函数时会解决第四个问题。第五,也是最后,使用 new 和 delete 操作符时就能很好的匹配。混合使用 new/ delete 与 malloc/free 不仅是不好的形式,而且还会产生严重后果:更多未定义的行为。

\filename{清单6.6 以规范的方式使用new和free}

\begin{cpp}
struct Buffer {
  char* str;
  Buffer(int size) : str(new char[size+1]) { str[0] = '\0'; }
  ~Buffer() { delete[] str; }
};
int main() {
  // double* val = new int; // 1
  double* val = new double; // 2
  std::cout << val << '\n';
  Buffer* buf = new Buffer(25); // 3
  std::cout << buf->str << ", size " << strlen(buf->str) << '\n';
  delete buf; // 4
  return 0;
}
\end{cpp}

{\footnotesize
注释1:不再可能,编译器会报错

注释2:类型安全得到保证

注释3:调用构造函数

注释4:调用析构函数;记得删除它!
}

Buffer 类可以删除或隐藏复制构造函数和复制赋值操作符,以使其实现更加健壮。这一点并未得到演示,但在使用动态资源时应始终考虑。

\mySamllsection{建议}

\begin{itemize}
\item
将 malloc 调用替换为 new,将 free 调用替换为 delete。

\item
每个 new 都有对应的 delete,每个 new[] 都有匹配的delete[]。
\end{itemize}
