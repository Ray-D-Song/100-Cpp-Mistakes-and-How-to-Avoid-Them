这个问题影响正确性。大多数情况下,容器是数组的首选替代方案,但这一事实并不能消除使用容器的问题。

\mySamllsection{问题}

通常,编程问题涉及通常处理或处理的数据组。数组是将单个数据元素作为一个单元,保存和管理的经典解决方案,但很少使用。标准模板库提供一些容器比数组更加智能,并且提供了值得使用的功能。

假设开发人员想要打印某个活动的人员列表。其中会混合使用Persons 和 Employees,但必须全部列出。清单 6.14 中的代码展示了这一尝试。由于其所有权语义,开发人员被鼓励在编程期间使用自动指针 (auto\_ptr)。大胆前进,代码开发和测试,其结果可能会更令人欣喜。

\filename{清单6.14 滥用多态元素的容器}

\begin{cpp}
struct Person {
  std::string name;
  Person(const std::string& n) : name(n) {}
  virtual std::string toString() { return name; }
};

struct Employee : public Person {
  double salary;
  Employee(const std::string& n, double s) : Person(n), salary(s) {}
  std::string toString() {
    std::stringstream ss;
    ss << Person::name << " gets paid " << salary;
    return ss.str();
  }
};

int main() {
  Person p("Sue");
  Employee e("Jane", 123.45);

  std::vector<Person> people;
  people.push_back(p);
  people.push_back(e);
  for (int i = 0; i < people.size(); ++i)
    std::cout << people[i].toString() << '\n'; // 1

  std::vector<std::auto_ptr<Person> > persons;
  // persons.push_back(p); persons.push_back(e); // 2

  std::vector<Person> peeps; // 3

  peeps.push_back(p);
  for (int i = 0; i < peeps.size(); ++i)
    std::cout << peeps[i].toString() << '\n';

  std::vector<Employee> emps; // 3
  emps.push_back(e);
  for (int i = 0; i < emps.size(); ++i)
    std::cout << emps[i].toString() << '\n';
  return 0;
}
\end{cpp}

{\footnotesize
注释1:多态行为不起作用

注释2:无法编译;具有不同的元素类型

注释3:单独的vector解决了该问题
}

\mySamllsection{分析}

尝试使用 std::vector<std::auto\_ptr<Person>{}>无法通过编译。开发人员对此代码进行了评论,希望以后对其进行研究。由于开发人员迫切希望代码能够正常工作,因此编写了单独的容器来保存每种数据类型。这种方法有效,但需要优化。

单个vector的问题在于保存的是基类型的对象。每个元素只分配了足够容纳基类对象的空间,从而切断了派生类数据。使用数组解决同一问题时也会出现此问题。多态行为就是这样!

第二次尝试是使用新奇的 auto\_ptr 类型,独占容器内的实例。这种方法很有意义,但会遇到麻烦。只有可复制构造和可分配的数据类型才能插入容器中。auto\_ptr 类型两者都不是,它支持独占所有权语义,所以无法将实例添加到vector中——新奇的想法就这么多。(现代 C++ 有一个更比使用 auto\_ptr 更好的方法;出于充分的理由,此类型在 C++11 中已弃用)。开发人员的最后一次尝试解决了该问题;但若没有单个容器的灵活性,解决方案可能会更好。

\mySamllsection{解决}

容器应包含值类型,可以是基本元单元、不与派生类实例混合的类型或指针(在本例中为原始指针)。现代 C++ 已经解决了原始指针问题,因此此建议仅应在前现代 C++ 中使用。选择原始指针作为容器的元素类型,是为了保持开发人员想要的灵活性、消除切片问题,并允许复制构造和赋值。以下代码展示了解决方案,并展示了开发人员的所有首选行为。

\filename{清单6.15 正确使用容器——多态元素}

\begin{cpp}
struct Person {
  std::string name;
  Person(const std::string& n) : name(n) {}
  virtual std::string toString() const { return name; }
  virtual ~Person() {}
};

struct Employee : public Person {
  double salary;
  Employee(const std::string& n, double s) : Person(n), salary(s) {}
  std::string toString() const {
    std::stringstream ss;
    ss << Person::name << " gets paid " << salary;
    return ss.str();
  }
};
int main() {
  Person p("Sue");
  Employee e("Jane", 123.45);
  std::vector<Person*> people;
  people.push_back(&p); // 1
  people.push_back(&e); // 1
  for (int i = 0; i < people.size(); ++i)
    std::cout << people[i]->toString() << '\n'; // 2
  return 0;
}
\end{cpp}

{\footnotesize
注释1:每一个都可以添加;相同的元素类型

注释2:多态行为有效
}

容器在使用多态元素时很有用,但必须正确管理。确保基类添加虚析构函数,容器在构造和析构期间会复制和删除元素。代码安全可确保运行时稳定性!

\mySamllsection{建议}

\begin{itemize}
\item
确保添加到容器中的内容都是可复制构造和可分配的;消除其中一个或两个的类不符合纳入条件。

\item
容器应保存值类型或指针类型,以避免切片或其他问题。

\item
如果可能,请考虑使用现代 C++。
\end{itemize}










