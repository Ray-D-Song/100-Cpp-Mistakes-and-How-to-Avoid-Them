这个错误主要针对性能和效率。初始化列表是为了增强初始化实例变量的性能。

\mySamllsection{问题}

清单 6.11 中的代码显示了尝试从提供的参数构建实例。完整标题是title、first、middle 和 last 名称的组合。在构造函数中用这些名称初始化完整标题是合理的,因为所有其他变量都是从提供的值初始化的。在这种情况下,可以在访问器中构建标题。尽管如此,为了挽救轻微的性能差异(假设从未访问过标题),所做的努力是微不足道,并迫使开发人员编写更多代码。这种方法影响了可读性,却没有提供足够的好处。

当代码在我的系统上运行的时候,会抛出一个 std::bad\_alloc 异常。开发人员会修复这个错误。这个结果是这个问题的一个可能的影响;你的系统可能会有不同的行为。无论如何,代码存在一个导致未定义行为的问题。如果碰巧这个代码在给定的系统上运行,可能会产生奇怪的输出,调试可能会很困难。

\filename{清单6.11 混淆初始化顺序}

\begin{cpp}
struct Person {
  std::string full;
  std::string first;
  std::string middle;
  std::string last;
  std::string title;
  Person(std::string f, std::string m, std::string l, std::string t) :
    first(f), middle(m), last(l), title(t),
    full(title + ' ' + first + ' ' + ' ' + middle + ' ' + last ) {} // 1
};

int main() {
  Person judge("Hank", "M.", "Hye", "Hon.");
  std::cout << judge.full << '\n';
  return 0;
}
\end{cpp}

{\footnotesize
注释1:从各个部分构建完整的标题
}

\mySamllsection{分析}

这段代码大部分都是正确的,但请注意实例变量的声明顺序和初始化顺序。首先声明了 full 实例变量,然后是其余变量。程序员对此进行了推理,并决定最后初始化这个实例变量,因为各个实例变量(似乎)已经被初始化了。

开发人员在这个假设上犯了一个错误;尽管完整的实例变量似乎最后才初始化,但编译器会编写代码按声明顺序初始化变量。因此,full 会先初始化,然后才是其他变量。这些未初始化变量的连接会导致异常或其他不良行为。

在理解了这个强制顺序之后,开发人员决定通过将代码放在构造函数主体中来控制初始化。清单 6.12 显示了结果。

\filename{清单6.12 将初始化推入构造函数体}

\begin{cpp}
struct Person {
  std::string full;
  std::string first;
  std::string middle;
  std::string last;
  std::string title;
  Person(std::string f, std::string m, std::string l, std::string t) { // 1
    first = f; middle = m; last = l; title = t;
    full = title + ' ' + first + ' ' + ' ' + middle + ' ' + last;
  }
};
int main() {
  Person judge("Hank", "M.", "Hye", "Hon.");
  std::cout << judge.full << '\n';
  return 0;
}
\end{cpp}

{\footnotesize
注释1:将所有初始化都放在构造函数主体中(性能较差)
}

此解决方案有效,但需要提高性能。这可能并不明显,但所有实例 变量仍然使用初始化列表形式进行初始化,就像代码一样

\begin{cpp}
Person(std::string f, std::string m, std::string l, std::string t) :
  full(), first(), middle(), last(), title() {}
\end{cpp}

编译器确保所有类实例都使用其默认构造函数进行初始化。因此,所有实例变量都初始化为初始化列表中的默认值。初始化之后,将执行 构造函数主体,并将参数值分配给现有实例。这是每次操作成本的两倍(我们不要挑剔复制构造函数和复制赋值运算符之间的性能差异) 。这种方法不是最优的。

\mySamllsection{解决}

由于编译器将使用初始化列表形式来初始化此类声明中的每个基于类的实例变量(而非基元变量),因此如果顺序很重要,理想的方法是按照变量应初始化的顺序对其进行排序。此示例对顺序很敏感,但应尽可能避免这种情况。如果顺序不敏感,请选择任何有意义的顺序( 按字母顺序、按类型分组等)。以下代码考虑了顺序并确保只有在其所有组成部分都完全初始化后才初始化 full。

\filename{清单6.13 严格使用初始化顺序}

\begin{cpp}
struct Person {
  std::string first;
  std::string middle;
  std::string last;
  std::string title;
  std::string full;
  Person(std::string f, std::string m, std::string l,
         std::string t) : // 1
    first(f), middle(m), last(l), title(t),

  full(title + ' ' + first + ' ' + ' ' + middle + ' ' + last ) {}
};

int main() {
  Person judge("Hank", "M.", "Hye", "Hon.");
  std::cout << judge.full << '\n';
  return 0;
}
\end{cpp}

{\footnotesize
注释1:重新排序变量以确保首先初始化组件部分
}

\mySamllsection{建议}

\begin{itemize}
\item
如果可能的话,避免使用复合实例变量;如果不能,请对它们进行排序,以便首先初始化所有组成部分。

\item
对所有实例变量使用初始化列表;如果不使用它们,编译器将无论如何都会生成它们。

\item
确保编译器警告已打开;这是最早、最广泛检测到的问题之一( 考虑在编译时始终使用 -Wall)。
\end{itemize}













