任何作者都希望自己能够正确评估读者的需求,并帮助弥补现有文献中的缺陷。读完本书后,你将能够发现这些错误和其他错误;了解它们为什么是问题;知道如何解决它们;从而避免犯类似的错误。

\mySubsubsection{1.3.1.}{留意错误}

意识是提高编程水平的第一步。当我解释一个例子时,我的学生眼神呆滞。这些学生试图理解这些概念,但尚未取得突破。为了减轻他们的痛苦,我经常说我不指望他们独自发展这些概念。我展示和解释这些例子是为了让他们对具体问题敏感,识别它们,然后看看如何解决它们。

无论我有多专业,这都要归功于那些教我和给我举例的人。虽然我知识渊博,看起来很聪明,但我必须不断学习(和其他人一样)。模式识别对于成长至关重要;一旦我们注意到熟悉的模式,我们就有了处理问题的工具。这本书将帮助你识别并解决现有代码中的错误,而不是延续它们。

\mySubsubsection{1.3.2.}{理解错误}

认识到错误是关键的一步,但要了解其本质,我们必须学会更好地理 解为什么这是一个错误。我们长期以来的一些最佳实践如今已不再被视为最佳做法。编程遵循科学模型;我们提出解决方案,使用它们,并评估其有效性。有时,我们赢了;有时,我们输了。

没有人会对这种进步感到惊讶;这是我们所有人的学习方式。当我们犯错时,我们可以填补知识上的空白(或缺乏知识)。这种填补涉及其他领域,我们扩展了我们的知识和对编程功能之间关系的理解。我想传达的是,每一个错误都可以是一个智力和情感驱动的改进机会。只有当我们放弃或拒绝学习时,才会失败。这本书将帮助您理解为什 么某种特定的编码方法是错误的。

\mySubsubsection{1.3.3.}{修复错误}

纠正错误是写这本书的真正目的。虽然纠正错误必不可少,但这只是更广泛过程中的一步。如果没有对错误造成问题的原因的认识和理解,我们只不过是修改源代码的机械单元(假设猴子也可以接受同样的训练)。

我们想要改正错误,因为我们知道这些错误为什么会产生问题。这种理解对编程团队来说是有益的。我们都会犯错误,但在将错误推送到生产之前发现并改正它们将确保开发人员、客户和公司获得更好的体验。这本书将帮助您学习如何自己改正和避免写作错误。

\mySubsubsection{1.3.4.}{前车之鉴‌}

检测、理解和修复错误对于编写良好的代码至关重要;然而,从这个过程中可以获得更多。随着经验的增长,其他有问题的编程代码将更 容易分析。模式检测和理解使人们能够发现其他错误和问题。这可以被认为类似于在健身房锻炼——它训练身体适应各种运动,而不一定专注于任何特定的运动。

您对这些错误的了解将使您成为其他人的有用资源,无论是在开发还是代码审查期间。当您分享您的知识时,您将帮助其他人成长并发展类似的技能。

本书将帮助您做好准备,为他人做出超出您当前能力的贡献。



