本书按照时间倒序组织内容,首先探讨现代 C++ 中的问题,随后分析影响现代 C++(C++11 及更高版本)与前现代 C++(C++98 和 C++03)之间的过渡问题。最后,我们将聚焦于前现代 C++,这部分内容涉及许多受限于现代方法的遗留代码库。

书中展示的代码主要基于 C++11 标准之前的代码(C++98 和 C++03 的结合体)。生成这些代码所使用的编译器是 GNU C++ 工具集(版本 11.3.0),运行环境为 Ubuntu 22.04 LTS(通过 Windows Subsystem for Linux [WSL] 2 实现)。选择这一编译器版本是为了确保在该平台上实现简单性和稳定性,尽管后续版本已发布并将继续更新。较新的编译器版本都可以重现本书中讨论的问题和编码缺陷,但使用较旧版本的编译器可能会导致结果有所不同。

业务限制可能迫使部分读者继续使用 C++98 进行开发,从而错失现代 C++ 功能带来的巨大优势。这确实令人遗憾,但却是现实情况。我的大部分代码库目前仍局限于 C++03,原因在于企业优先考虑稳定性,而操作系统及工具团队对更高版本的审查过程往往漫长且痛苦。

尽管以下列举的许多错误出现在经典 C++(即 C++11 之前的标准)中,但其中一些问题同样存在于现代 C++ 中。由于这些问题对现代开发者具有重要意义,因此会安排在较早的章节中进行讨论。

在本书的后半部分,所有问题及其解决方案均采用经典 C++ 方法进行阐述。尽管某些问题也可能出现在更现代的代码中,但应优先考虑使用现代技术解决它们。然而,对于那些无法使用现代编译器的开发者而言,掌握如何在受限环境中处理这些问题至关重要。这种情况虽然令人遗憾,但确实是当前实际开发环境中的挑战。
