
有经验的 C++ 开发者都会告诉你,C++ 中可能出现的错误远不止 100 种。我所选择的这些错误,是我亲身经历、深入研究或通过其他方式发现的典型问题。可以将这份列表视为我的“前 100 名”,它们大致可分为以下几类。

\mySubsubsection{1.4.1.}{类型设计}

C++ 的核心目标之一是将面向对象编程范式引入流行的 C 语言。类型设计本质上是对新数据类型的定义过程,目的是让编译器能够像处理内置类型一样理解和操作这些新类型。开发者需要完全负责确定新类型的意义、行为,以及其语义。

尽管编译器会严格遵循语言规则,但它对新创建的数据类型几乎没有任何限制。这种自由度虽然强大,但也成为了错误滋生的温床。因此,开发者需要深刻理解这些问题,并主动采取措施加以缓解,而不是单纯依赖语言本身的约束。

\mySubsubsection{1.4.2.}{程序实现}

在编程过程中,设计或实现中的不足往往会导致各种困难。C++ 的灵活性是一把双刃剑,它要求开发者必须深入了解语言特性,才能有效避免或改进潜在的问题。本部分中的一些错误与类型设计相关,但由于其适用范围更广,因此归类在此标题下。正确运用语言特性是生产高质量软件的关键所在。

\mySubsubsection{1.4.3.}{库的问题}

开发者的代码通常只是整个程序的一小部分,而大部分功能依赖于库和函数的支持。此外,某些语言特性(如模板)可以更通用的方式解决问题。例如,模板可以通过描述通用解决方案并让编译器生成最终代码,从而解决一系列相关问题。

库函数提供了丰富的功能,如果开发者尝试自行实现这些功能,不仅耗时耗力,还容易引入错误。然而,正确使用这些库函数的前提是,开发者必须充分理解其用法。如果误解了如何正确调用这些库函数,则可能导致严重问题,甚至影响程序的稳定性和性能。

\mySubsubsection{1.4.4.}{现代C++}

C++ 的一个重要目标是确保语言在不断发展的同时,仍保持显著的向后兼容性。这种兼容性使得现代编译器能够顺利编译旧代码,但这也并不意味着旧代码一定是高质量或理想的。随着语言的演进,许多过去的错误和问题得到了缓解,并出现了更优的解决方案。

现代 C++ 提供了许多改进,使开发者能够编写更正确、更灵活且更简洁的代码,同时完成与过去相同的功能。这些改进通常会影响多个关键方面:正确性、可读性、效率和性能。然而,为了充分利用这些改进,开发者需要投入时间学习新特性,并理解它们如何改变了我们的思维方式和编码习惯。

\mySubsubsection{1.4.5.}{旧标准和用法}

在这些新特性出现之前编写的代码,显然无法享受到这些改进带来的好处。即使在这些特性推出之后,许多开发者可能仍然沿用旧的编程风格,导致代码功能陈旧、表达能力受限,甚至更容易出错。例如,C++ 字符串可以有效缓解 C 风格字符串的诸多问题,但开发者只有真正使用 C++ 字符串,才能获得这些优势,并避免潜在的错误。

\mySubsubsection{1.4.6.}{陈旧的经验和过时的知识}

随着经验丰富的 C++ 开发者逐渐退出行业,新一代开发者需要填补他们的位置。然而,这些新人不得不面对一些较旧、质量较差且更具挑战性的代码。遗憾的是,许多学术机构提供的现代 C++ 培训并不完善,进一步加剧了这一问题。一些最新的 C++ 教科书中的示例和建议仍然过时,与当前的最佳实践脱节,甚至在实际开发中无效。

有些错误源于过时的方法,有些则是因为糟糕的教学方式,还有一些是因为过去的最佳实践已不再适用。无论这些错误的根本原因是什么,保持对这些问题及其潜在解决方案的敏感性,都将有助于 C++ 开发者编写出更加正确、可读、高效且性能优异的代码。


