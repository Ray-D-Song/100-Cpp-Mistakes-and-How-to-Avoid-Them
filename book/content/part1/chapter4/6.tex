这个错误主要针对可读性和有效性,但需要经过一段陡峭的学习曲线才能解决。一次性且不常调用的函数往往会使源码变得混乱,并使查找变得困难。

\mySamllsection{问题}

假设,开发人员编写代码来管理美国一所学校。有几名学生在读,每个学生都必须在数据库中。与任何由多人组成的组织一样,必须进行排名或排序。我们的开发人员要求按姓名、年龄和 GPA 对学生进行排名。

为了便于处理,学生名单保存在容器中。必须使用某种方法来迭代容器并确定首选顺序,可以修改容器内的顺序。

开发人员编写了三个独立函数来实现基于给定标准的排名,代码库比我们想的要大得多。这些新函数必须放在某个地方,因此它们在许多其他代码中的文件中定义,并且在其他地方调用这些函数来执行其行为。

\filename{清单4.13 几个独立函数}

\begin{cpp}
class Student {
private:
  std::string name;
  int age;
  double gpa;
public:
  Student(const std::string& n, int a, double g) : name(n), age(a), gpa(g) {}
  std::string getName() const { return name; }
  int getAge() const { return age; }
  double getGpa() const { return gpa; }
};

// assume lots of code

bool by_name(const Student& s1, const Student& s2)
{ return s1.getName() < s2.getName(); } // 1
bool by_age(const Student& s1, const Student& s2)
{ return s1.getAge() < s2.getAge(); } // 1
bool by_gpa(const Student& s1, const Student& s2)
{ return s1.getGpa() < s2.getGpa(); } // 1
void output_by_student_name(const std::vector<Student>& s) {
  for (int i = 0; i < s.size(); ++i)
    std::cout << s[i].getName() << '\n';
  std::cout << '\n';
}
// assume even more code

int main() {
  std::vector<Student> s;
  s.push_back(Student("Susan", 23, 3.85));
  s.push_back(Student("James", 24, 3.35));
  s.push_back(Student("Annette", 25, 3.75));
  s.push_back(Student("Wilson", 26, 3.8));

  std::sort(s.begin(), s.end(), by_name); // 2
  output_by_student_name(s);
  std::sort(s.begin(), s.end(), by_age); // 2
  output_by_student_name(s);
  std::sort(s.begin(), s.end(), by_gpa); // 2
  output_by_student_name(s);
  return 0;
}
\end{cpp}

{\footnotesize
注释1:几个隐藏在代码中的函数,只使用一次

注释2:调用函数,但没有参数或实现的提示
}

\mySamllsection{分析}

命名函数使得描述其行为的句柄易于记忆,但当混入其他代码时,其实现就很难找到。由于定义和调用站点相距甚远,开发人员必须记住其实现细节,才能充分理解函数的预期行为。

现在,做一个大胆的假设:开发人员最近有机会学习函数式编程。一个想法突然出现在他们的脑海中,认为这些函数可以重新设计成一种更具功能性的方法。一个新的需求出现了,开发人员认为这是尝试创建函子的绝佳机会。

函子是实现 operator() 的结构(或类)。此操作符有多种名称,我们将其称为 apply 操作符 (以下函数术语),其是一个执行行为的普通函数 — 本例中,确定最大元素。为了实现这一点,创建了一个 Maximum 结构,其应用操作符用于查找vector中的最大值。将apply 视为将函数应用于数据,也许感觉很落后,但这就是函数式思维方式。传统的命令式编程更多地考虑,将数据传递给函数并执行其行为;函数式编程考虑,将函数印记或应用于数据——一种以数据为中心的模型。

清单 4.14 中的代码展示了,确定最大值的两种实现。第一种是开发人员在创建函子后想出的。函子的实例化会创建一个函数对象 f,f 对数据的应用是常规实现的;函数对象后面跟着一个参数列表,并返回一个结果(如果有)。开发人员的团队负责人提供了另一种方法,如下所示。团队负责人建议使用 Lambda 来简化代码,消除命名函子及其实例化,并让读者理解调用点的行为。

\filename{清单4.14 演示就地函数定义}

\begin{cpp}
template <typename T>
struct Maximum { // 1
  T operator()(const std::vector<T>& vals) { // 2
    T max = vals[0];
    for (const auto& v : vals)
      if (v > max)
        max = v;
    return max;
  }
};

int main() {
  std::vector<int> v {3, 9, 6, 2 -1, 0, 8};
  Maximum<int> f; // 3
  std::cout << f(v) << '\n'; // 4
  auto max = std::accumulate(v.begin(), v.end(), v[0],
    [](auto a, auto b){return std::max(a, b); }); // 5
  std::cout << max << '\n';
  return 0;
}
\end{cpp}

{\footnotesize
注释1:定义函数对象类(函子)

注释2:定义应用操作符

注释3:实例化函数对象类,生成函数对象

注释4:将函数对象应用于数据;调用应用操作符

注释5:使用带有局部最大值 Lambda 函数的数值算法
}

\mySamllsection{解决}

团队负责人展示了 Lambda 的一个优势,能够在需要的地方准确定义行为。这种方法消除了回滚和寻找函数定义,以及定义函子的尴尬。清单 4.15 中的代码演示了按姓名、年龄或 GPA 对学生进行排名的原始三个函数。此示例中,行为是在使用时定义的,清楚地传达了正在执行的操作。诚然,最初几次编写和阅读 Lambda 需要更长的时间,但理解了Lambda就会变得简单且自动化。

由于 Lambda 是函数对象,因此可以将它们分配给变量,并在调用点传递该变量。GPA 对学生进行排名的地方就证明了这种方法。

\filename{清单4.15 原位函数定义(Lambda)}

\begin{cpp}
class Student {
private:
  std::string name;
  int age;
  double gpa;
public:
  Student(const std::string& n, int a, double g) : name(n), age(a), gpa(g) {}
  std::string getName() const { return name; }
  int getAge() const { return age; }
  double getGpa() const { return gpa; }
};

// assume lots of code

void output_by_student_name(const std::vector<Student>& s) {
  for (int i = 0; i < s.size(); ++i)
    std::cout << s[i].getName() << '\n';
  std::cout << '\n';
}

int main() {
  std::vector<Student> s;
  s.push_back(Student("Susan", 23, 3.85));
  s.push_back(Student("James", 24, 3.35));
  s.push_back(Student("Annette", 25, 3.75));
  s.push_back(Student("Wilson", 26, 3.8));

  std::sort(s.begin(), s.end(),
    [](const Student& s1, const Student& s2){
      return s1.getName() < s2.getName(); }); // 1
  output_by_student_name(s);

  std::sort(s.begin(), s.end(),
    [](const Student& s1, const Student& s2){
      return s1.getAge() < s2.getAge(); }); // 1
  output_by_student_name(s);

  auto f = [](const Student& s1, const Student& s2) {
    return s1.getGpa() < s2.getGpa(); }; // 2
  std::ranges::sort(s, f); // 3
  output_by_student_name(s);
  return 0;
}
\end{cpp}

{\footnotesize
注释1:在需要的地方定义函数

注释2:在需要之前定义函数

注释3:使用范围并将函数对象作为参数传递
}

GPA 的排序在姓名和年龄方面略有不同,展示了使用范围 API 的替代方法,其更紧凑、更易读且更易于编写。那么为什么要等到开始使用范围呢?

Lambda 具有广泛的适用性;当一次性且不常调用的函数在编写,或研究的代码上下文中,难以找到和记住时,Lambda 函数提供了在需要时准确定义函数的机会,使开发人员和维护人员无需寻找其实现。

Lambda 函数最好只使用一次;如果多次调用,则将 Lambda 赋给变量。当然,在需要的地方定义函数的好处就消失了;与几乎所有事情一样,必须考虑竞争值的权衡。

\mySamllsection{建议}

\begin{itemize}
\item
将 Lambda 用于简单、不常调用的函数;提供的文档可显著提高理解力,而不会占用过多的短期记忆。

\item
考虑在现有代码使用简单函数的地方添加 Lambda。

\item
标准模板库有几个函数可用于代替编写实现(例如,std::max\_element 和 std::less、 std::sort)。

\item
考虑使用范围 API 来简化编码并方便阅读。开发者不必指定 begin 或 end 迭代器;范围会计算出这一点 — 只需提供容器和一个函数。
\end{itemize}









