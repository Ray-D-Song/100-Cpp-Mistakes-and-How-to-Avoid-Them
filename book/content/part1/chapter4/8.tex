这种错误会影响有效性和正确性,但对可读性也有负面影响。字符串数据中的模式匹配很常见;模式匹配代码费力又复杂,很难做到恰到好处。

\mySamllsection{问题}

一个常见问题是文本数据的解析。输入以字符串数据的形式输入,而从文件读取的信息通常是文本。当所需的数据类型是字符串时,这种方法很好用,但在许多情况下,需要验证文本或将文本分解为组成部分。除了最简单的情况外,分析数据格式、内容或类似内容的代码在所有情况下都具有挑战性。清单 4.18 演示了电子邮件地址的验证。规则是单个单词(用户名)由 at 符号 (@) 分隔,后跟另一个单词(域) ,由句点分隔,后跟另一个单词(顶级域)。代码检查这两个分隔符和三个单词,其中一个单词是一个或多个字母(不包括 at 符号或句点)。如果有任何不妥,则电子邮件地址被视为无效。

\filename{清单4.18 手工编码的电子邮件验证函数(有bug)}

\begin{cpp}
bool isValidEmail(const std::string& e) {
  size_t pos_at = e.find('@'); // 1
  if (pos_at == std::string::npos) // no '@' found
  return false;
  if (pos_at == 0) // no first word // 2
    return false;
  size_t pos_period = e.find('.', pos_at); // 1
  if (pos_period == std::string::npos) // no '.' found
    return false;
  if (pos_at + 1 >= pos_period) // no middle word // 2
    return false;
  if (pos_period == e.length() - 1) // no last word // 2
    return false;
  return true;
}
int main() {
  std::cout << isValidEmail("prof@nu.edu") << '\n';
  return 0;
}
\end{cpp}

{\footnotesize
注释1:解析分隔符

注释2:解析单词
}

\mySamllsection{分析}

代码确定组成用户名、域和顶级域的字符(任何字符,甚至是无效字符)的存在。必须用 at 符号和句点将它们分开。这种匹配存在不精确性,但它可以正确解析表单,而无需检查其他有效性。恶意或错误的代码可能会让错误的电子邮件地址逃过此检查。想象一下,增强此代码以确保用户名的字母、数字和标点符号正确,同样,两个域部分( 没有标点符号)的字母、数字和标点符号正确。如前所述,编写解析器很困难。此外,要理解代码,必须对其进行大量文档记录带有注释。注释有一个令人讨厌的习惯,即在第一次维护代码后变得陈旧或错误。

\mySamllsection{解决}

通过包含 regex 标头,现代 C++ 中添加了文本模式匹配。正则表达式 是一种计算机,是公认的最简单的类型。通常,它们由状态机实现,每个字母(符号)都会导致从一个状态到另一个状态的转换。如果模式匹配,则认为已接受。清单 4.19 显示了该行 为的实际效果。我们看到了五个基本部分,从模式的左侧开始向右移动:

\begin{enumerate}
\item
用户名,由一个或多个字母、数字和有限标点符号组成

\item
at符号分隔符

\item
域名,由一个或多个字母或数字组成

\item
句点(注意句点前的反斜杠转义符)

\item
顶级域名,由两个或多个字母组成
\end{enumerate}

此解析器模式比清单 4.17 中的代码更精确,因为它确保顶级域名至少有两个字母(从技术上讲,顶级域名可以有一个字符,但尚未注册;请参阅 \url{http://data.iana.org/TLD/tlds-alpha-by-domain.txt} 以获取当前列表)。此外,用户名和域名经过验证,仅包含允许的符号。这种精度对正确性有很大影响。

\filename{清单4.19 使用正则表达式进行电子邮件验证}

\begin{cpp}
bool isValidEmail(const std::string& e) {
  std::regex pattern(
      R"([a-zA-Z0-9._%+-][CA}+@[a-zA-Z0-9.-]+\.[a-zA-Z]{2,})"; ❶
  return std::regex_search(e, pattern);
}

int main() {
  std::cout << isValidEmail("prof@nu.edu") << '\n';
  return 0;
}
\end{cpp}

{\footnotesize
注释1:用于匹配电子邮件地址的五段正则表达式
}

另一个演示的功能是原始字符串。原始字符串以 R"( 开头,以 )" 结尾。其目的是表达(通常是正则表达式模式)而无需转义任何字符。眼尖的读者会注意到正则表达式中有一个转义字符;转义是正则表达式的重要组成部分。通常,句点匹配单个字符(换行符除外)。必须对其进行转义才能告诉正则表达式匹配句点。

当需要转义(非原始字符串)时,正则表达式的阅读难度更大,因为必须将模式与通常控制 C++ 格式的字符区分开来。一个简单的示例清楚地显示了其好处。假设匹配的是双反斜杠。使用转义字符,模式将如下所示:

\begin{cpp}
std::string pattern = "\\\\";
\end{cpp}

使用原始字符串,它将是

\begin{cpp}
std::string pattern = R"(\\)"
\end{cpp}

后者精确地显示了匹配的内容,而前者将其隐藏在转义字符中。相信了吗?

正则表达式旨在匹配文本模式。它们的语言很复杂,需要花费精力 去学习,但开发人员发现这种努力是值得的。一旦理解了,模式就很容易编写、测试简单且表达紧凑。使用正则表达式可以获得更强大的功能。可以提取匹配模式的部分,例如用户名或所有三个部分。可以使用正则表达式用新值替换文本。

\mySamllsection{建议}

\begin{itemize}
\item
学习和使用正则表达式是值得的;确保定制的文本模式匹配代码的正确性是一项挑战。

\item
正则表达式表达能力很强,但读写却很困难;请使用您能想到的每种数据变体对它们进行彻底测试。
\end{itemize}






