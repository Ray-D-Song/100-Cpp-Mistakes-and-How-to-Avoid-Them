这个错误会影响可读性、有效性,并且在一定程度上影响正确性。常量是必不可少的,但表达它们通常会导致代码难以阅读、值被截断以及使用不灵活。

\mySamllsection{问题}

许多程序必须定义常量或使用将单位按常量转换的函数。传统上,可以使用 \#define 或 const 定义来指定常量或函数的名称和值,我们使用它们将值从一种单位类型转换为另一种单位类型。

下面的代码展示了使用定义常量的三个例子。第一个是一年中的秒数(为简单起见,假设为非闰年)。但是,这个定义有一个问题:它太大了 10 倍。

\filename{清单4.16 常数值的传统用法}

\begin{cpp}
const int SEC_PER_YEAR = 315360000; // 1

const double pi = 3.1415927; // 2
struct Circle {
  double radius;
  Circle(double r) : radius(r) {}
  double perimeter() throw() { return 2 * pi * radius; }
  double area() throw() { return pi * radius * radius; }
};

int main() {
  Circle c(3);
  std::cout << "perimeter " << c.perimeter() << ", area " << c.area() << '\n';

  double rads = 90 * pi / 180; // 3
  std::cout << "90 degrees is " << rads << " radians\n";

  std::cout << "There are " << SEC_PER_YEAR << " seconds in a year\n";
  return 0;
}
\end{cpp}

{\footnotesize
注释1:这是正确的吗?还是这 10 倍太大了?

注释2:不准确的近似值

注释3:不灵活的方法,仅适用于 90 度
}

\mySamllsection{分析}

定义第一个常量时很容易犯错误。开发人员很难区分零的数量,错误地多加了一个。第二个是 pi 的截断值。开发人员为许多用途编写了 一个合理的近似值,但考虑到数据类型可以表示的内容,这个近似值相对不精确。最后,将 90 度转换为弧度不够灵活——它只适用于 90 度。如果能用函数来表示会更好。

\mySamllsection{解决}

现代 C++ 为这三个问题提供了解决方案。首先将解决 pi 的值。numbers 标头定义了此值和其他几个值,而不是由开发人员创建一些近似值。系统架构会考虑这些值,并选择最佳值;无需再次手写它们。

使用千位分隔符改进了每年秒数的定义。撇号将常数值分成几组,这些组与书籍或文章中的读法相近。逗号已被定义为运算符,因此选择了类似的符号。使用它简化了零的数量计算(在本例中),并使错误更加明显。

最后,将度数转换为弧度使用用户定义文字 (UDL)。operator"" 是一个接受参数值的函数并对其进行转换。结果是一个可读的用法,可以处理任何合法的值。

\filename{清单4.17 改进了常数值的使用}

\begin{cpp}
constexpr double operator"" _deg_in_rad(long double d) { // 1
  return d * std::numbers::pi / 180;
}

constexpr int SEC_PER_YEAR = 31'536'000; // 2

struct Circle {
  double radius;
  Circle(double r) : radius(r) {}
  double perimeter() noexcept { return 2 *
  std::numbers::pi * radius; } // 3
  double area() noexcept { return std::numbers::pi * radius * radius; }
};

int main() {
  Circle c(3);
  std::cout << "perimeter " << c.perimeter() << ", area " << c.area()
    << '\n';

  std::cout << "90 degrees is " << 90.0_deg_in_rad << " radians\n";

  std::cout << "There are " << SEC_PER_YEAR << " seconds in a year\n";
  return 0;
}
\end{cpp}

{\footnotesize
注释1:执行转换的用户定义文字

注释2:易于阅读的精确分隔的数字

注释3:这是该架构所能提供的最好的近似
}

定义常量通常是必要的。可用的现代 C++ 技术使代码更易读且更易于实现。

\mySamllsection{建议}

\begin{itemize}
\item
使用千位分隔符来分隔长数字序列。

\item
使用 operator"" 轻松地将一个单位转换为另一个单位;可以为任何数据类型创建文字 - 查看 std::chrono,,它具有强类型 minutes、hours 等,展示了这一强大的概念。

\item
使用 numbers 标头中定义的常量来简化使用并获取系统的最佳近似值。
\end{itemize}
