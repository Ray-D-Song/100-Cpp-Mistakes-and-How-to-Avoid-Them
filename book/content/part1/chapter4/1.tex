这个错误影响了效率和可读性。对于许多问题来说,变长参数列表是很自然的,但处理它们会增加开销和复杂性。

\mySamllsection{问题}

可变参数列表非常常见,C 语言很早就提供了一种处理它们的方法。C++ 继承了这种方法(请参阅“另请参阅”部分),但事实证明这种方法存在问题。

C++ 提供了一种更简洁的方法来处理这种情况,但并非没有复杂性和开销。清单 4.1 将参数捆绑到容器中,并将容器传递给 sum 函数(可以使用 std::accumulate 算法更好地解决此问题)。没有特殊的函数调用(例如,va\_list),处理容器很简单。此解决方案模板化以处理加法运算符有意义的任何类型。但是,这种好处是有代价的:开发人员必须创建容器,将参数打包到其中,并将其作为参数传递给函数。所有这些努力都是解决问题的附带工作;打包数据只是必要的日常工作。阅读此代码需要分散注意力,专注于解决问题和开销。

\filename{清单4.1 使用vector实现可变形参列表}

\begin{cpp}
template <typename T>
T sum(T initial, const std::vector<T>& vals) { // 1
  T sum = initial;
  for (int i = 0; i < vals.size(); ++i)
    sum += vals[i];
  return sum;
}
int main() {
  std::vector<int> intvalues; // 2
  for (int i = 1; i < 10; ++i) // 3
    intvalues.push_back(i);
  std::cout << sum(0, intvalues) << '\n'; // 4
  std::vector<double> doublevalues;
  for (int i = 1; i < 4; ++i)
    doublevalues.push_back(i);
  std::cout << sum(5.0, doublevalues) << '\n';
  return 0;
}
\end{cpp}

{\footnotesize
注释1:选择模板化表单,实现更大的灵活性

注释2:创建容器

注释3:填充容器

注释4:最后,解决问题
}

\mySamllsection{分析}

清单 4.1 中所示的解决方案是上一节推荐的(参见“另请参阅”部分),它给开发人员的效率留下了一个重大问题,但也为改进提供了机会。程序员负责构建一个包含要处理的值列表的结构。此结构的构建纯粹是开销;它不是要解决的问题的一部分,而是将值聚合为可用形式的必要步骤。代码越不直接关注问题,开发人员的效率就越低,出错的机会就越大。此外,它很无聊。

\mySamllsection{解决}

现代 C++ 提供了一种使用递归方法解决此问题的绝佳方法。其思想是为通用模板函数提供长度可变的参数列表。模板将可变参数列表解析为第一个参数和其余参数形式的对。对粘贴到上述对中的其余参数进行递归。此过程持续到最后一个参数独立而没有任何剩余参数为止。编写了第二个接受一个参数的模板函数;这是通用模板的特化。模板的一个显着优势是它们在编译时进行评估,从而节省了运行时成本。

清单 4.2 展示了这两个模板函数。特化(或基础)模板很简单;它接受一个参数,返回它。递归(或通用)模板也很简单。每个参数对都是最左边的参数和其余参数的可变参数包。当前模板(概念上)被推送到堆栈上,其剩余参数成为对模板的下一次递归调用的参数列表。此模板调用将参数列表分为最左边的参数及其剩余参数。当剩余参数恰好包含一个参数时,将调用专用模板。然后(概念上)使用它们的单个参数弹出堆栈模板调用,并从每个参数构建函数调用。

要实现这种模板化的可变参数模式,首先编写基本情况,然后编写递归情况。就是这样!当编译器解析函数调用时,它会查找可以处理多个参数的 sum 函数。它将发现具有单个参数和可变参数包的模板。如果它可以生成参数对(最左边和剩余),它就知道继续调用该模板版本。当只剩下一个参数时,可变参数模板不适用。编译器将选择具有单个参数的 sum 模板。如果没有该基本情况函数,这种技术将无法工作。但它确实有效,而且效果非常出色。

\filename{清单4.2 使用模板化的可变参数列表}

\begin{cpp}
template <typename T> // base case
T sum(T t) { return t; }

template <typename T, typename... Pack> // general (recursive) case
T sum(T t, Pack... remaining) { return t + sum(remaining...); }

int main() {
  std::cout << sum(3, 4) << '\n'; // 1
  std::cout << sum(3, 4, 5) << '\n'; // 1
  std::cout << sum(3, 4, 6, 7) << '\n'; // 1
  return 0;
}
\end{cpp}

{\footnotesize
注释1:以自然的形式解决了问题
}

考虑使用折叠表达式(C++17 及更高版本)来实现扩展目标。清单 4.2 中的模板可以压缩为

\begin{cpp}
template <typename T, typename... Pack>
T sum(T t, Pack... remaining) {
  return (t + ... + remaining);
}
\end{cpp}

\mySamllsection{建议}

\begin{itemize}
\item
使用带有可变参数包的模板来干净地处理可变长度的参数列表。

\item
避免使用前现代 C++ 方法处理可变参数列表。
\end{itemize}
