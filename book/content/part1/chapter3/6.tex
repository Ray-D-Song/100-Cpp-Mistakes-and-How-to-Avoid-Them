这个错误主要针对性能。向容器中添加大型或复杂元素时,成本会很高。

\mySamllsection{问题}

管理对象集合时,应优先使用vector之类的容器。通常创建对象并将其复制到容器,如清单 3.12 所示。创建对象需要使用构造函数,之后将其内容复制到容器的元素中;如果使用复制赋值操作符,需要开销也相当于构造函数。临时变量的成本会很高,但这无法避免。

就地创建旨在在特定内存位置创建对象。局部变量分配在运行时栈上,动态创建的对象分配在堆上。开发者不会影响这些变量的位置,只需确保它们存在即可。另外,就地创建可让开发人员指定创建对象的位置。

以下代码演示了创建对象,并将其添加到vector的典型方式。通常,创建对象时会调用一次构造函数调用。将对象添加到vector时,vector会将其复制到其元素中,从而再调用一次构造函数。

\filename{清单3.12 创建和添加元素到容器}

\begin{cpp}
class Resource {
private:
  std::string name;
  int instance;
  static int handle;
public:
  Resource(const std::string& n) : name(n), instance(++handle) {}
  int id() const { return instance; }
};
int Resource::handle = 0;

int main() {
  std::vector<Resource> resources;
  resources.push_back(Resource("resource 1")); // 1
  resources.push_back(Resource("resource 2"));
  for (int i = 0; i < resources.size(); ++i)
    std::cout << resources[i].id() << '\n';
  return 0;
}
\end{cpp}

{\footnotesize
注释1:创建一个临时变量,然后将其复制到 vector 的元素中
}

\mySamllsection{分析}

创建 Resource 对象需要调用构造函数。要将对象添加到vector,容器必须进行复制或赋值,以将源对象的内容传输到目标容器元素。然而,将创建对象的成本增加一倍实在不应该。

\mySamllsection{解决}

现代 C++ 改变了规则,以避免两步将对象插入容器的成本过高。如清单 3.12 所示,实例的创建只是为了插入容器;否则,不会使用。这种情况下,可以直接在容器的元素中构造对象,如清单 3.13 所示。就地创建使用单个构造函数在vector元素的位置创建对象。这种技术减少了一个构造函数或赋值调用,从而提高了性能。还需注意的是,定义移动构造函数和移动赋值操作符,可将不可复制的对象放入容器中。

\filename{清单3.13 直接在容器中创建对象}

\begin{cpp}
class Resource {
private:
  std::string name;
  int instance;
  static int handle;
public:
  Resource(const std::string& n) : name(n), instance(++handle) {}
  int id() const { return instance; }
};
int Resource::handle = 0;

int main() {
  std::vector<Resource> resources;
  resources.emplace_back("resource 1"); // 1
  resources.emplace_back("resource 2");
  for (int i = 0; i < resources.size(); ++i)
    std::cout << resources[i].id() << '\n';
  return 0;
}
\end{cpp}

{\footnotesize
注释1:在 vector 中创建一个对象
}

并非每个类都可以使用这种技术。如果对象不可复制或不可分配,则编译器在编译代码时会大声抱怨。如果这些对象定义了移动语义,则编译器会报错。容器中的对象可能必须在容器管理期间移动或复制。如果vector必须增长,则必须将前一个vector的元素复制到新vector中,所以就地创建仅适用于某些数据类型。

\mySamllsection{建议}

\begin{itemize}
\item
当对象不独立于容器使用时,请考虑使用就地创建。

\item
指针容器(最好是智能指针容器!)可能比值容器更好。
\end{itemize}













