这个错误主要针对可读性和有效性。元组是一种方便的模板,可以传递或返回多个值。

\mySamllsection{问题}

错误 16 中的示例展示了清单 3.16 中的代码,消除了编写结构体或类的麻烦,并依赖于 std::tuple 模板。这种方法的简单性是其最显著的吸引力,但访问值时必须使用函数模板 std::get。此代码编写和阅读起来很不方便,这引起了人们对其实用性的担忧。

\filename{清单3.16 使用函数模板提取元组值}

\begin{cpp}
using Student = std::tuple<std::string, int, double>;

int main() {
  Student s("Susan", 23, 3.85);
  std::cout << "student " << std::get<0>(s) << ", " << std::get<1>(s) <<
        " years old, carries a " << std::get<2>(s) << '\n'; // 1
  return 0;
}
\end{cpp}

{\footnotesize
注释1:使用带有索引的std::get模板函数
}

\mySamllsection{分析}

Student 元组很容易初始化, 看起来像是对构造函数的调用。我们认为这种简单性将继续存在,但对字段的访问却证明并非如此。元组的字段按定义顺序排列。由于它们没有名称,因此访问的唯一选项是 get 函数使用索引值。这种方法效果很好,但字段的“名称”与索引之间的脱节的确令人不安。

\mySamllsection{解决}

std::tuple 模板的灵活性因其访问字段的方法过多,且不连贯而减弱。编码正确的索引值时很容易出错,或者稍后重新排列字段,并忘记已经编码的访问函数。

结构化绑定将元组,或其他具有公共数据成员的对象,分解为单个值。有些语言将此称为“解构”。其语法很简单:以 auto 关键字开头,添加括号,并为每个数据成员添加一个变量。然后,将元组实例分配给它。以下代码显示了使用结构化绑定的简易性。

\filename{清单3.17 使用结构化绑定提取元组值}

\begin{cpp}
using Student = std::tuple<std::string, int, double>;
int main() {
  Student s("Susan", 23, 3.85);
  auto [name, age, gpa] = s; // 1
  std::cout << "student " << name << ", " << age <<
      " years old, carries a " << gpa << '\n'; // 2
  return 0;
}
\end{cpp}

{\footnotesize
注释1:使用结构化绑定来分解元组

注释2:使用变量名代替索引值
}

如果结构中字段的顺序发生变化,结构化绑定将存在严重的正确性和可读性问题。如果重新排列了清单 3.17 中的变量名称,则它们与字段的顺序将不一致。如果在结构中重新排列了 age 和 gpa,则变量名称将具有错误的值。如果重新排列使得类型可转换,则清单 3.17 中的代码可以工作(没有编译器或运行时错误),但值将不正确。因此,请注意确保结构字段和结构化绑定变量保持协调一致。

\mySamllsection{建议}

\begin{itemize}
\item
使用结构化绑定来分解 POD 和简单结构或类。

\item
对象必须具有公共数据成员,结构化绑定才能正常工作。
\end{itemize}
