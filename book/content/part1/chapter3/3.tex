
这个错误会影响可读性和有效性,类型别名通常用于从现有数据类型定义新数据类型。新类型更能表达类型的用途,并提高可读性。

\mySamllsection{建议}

清单 3.6 中的代码使用 typedef 来定义一个可以更好地表达概念的新类型。新类型是一个接受两个整数并返回一个整数的函数,MathFunction 的类型是一个函数指针,因此可以像函数一样分配和调用符合要求的函数。这时的别名嵌入在定义中,既不明显,也不直观。

\filename{清单3.6 使用typedef定义别名}

\begin{cpp}
typedef int (*MathFunction)(int, int); // 1

int add(int a, int b) {
  return a + b;
}

int sub(int a, int b) {
  return a - b;
}
int main() {
  MathFunction f = add; // 2
  std::cout << "Result of addition: " << f(5, 3) << std::endl;
  f = sub; // 2
  std::cout << "Result of subtraction: " << f(5, 3) << std::endl;
  return 0;
}
\end{cpp}

{\footnotesize
注释1:以函数形式定义新类型

注释2:赋值给新类型的变量
}

\mySamllsection{分析}

别名定义的尴尬之处在于,是从 C 中继承下来的,C++ 仍支持这种语法。如果新类型基于内置类型(例如 int),源类型会位于目标类型之前,这有些反直觉。例如,将别名 age 定义为无符号整数:

\begin{cpp}
typedef unsigned int age;
\end{cpp}

\mySamllsection{解决}

现代 C++ 引入了 using 关键字,这在很大程度上简化了别名类型。此关键字遵循更直观的方法,其中别名名称位于左侧,其定义位于右侧。赋值操作符使人们认为别名是从定义中分配的,就像常规算术赋值一样,增强了可读性。此外,编写别名更有效,其遵循更直观的语法。

以下代码与清单 3.6 几乎完全相同;唯一的区别在于别名定义。这种相似性表明了两点:首先,使用别名类型与 typedef 版本相同;其次,没有理由继续使用 typedef 别名。

\filename{清单3.7 使用typedef为类型创建别名}

\begin{cpp}
using MathFunction = int (*)(int, int); // 1
int add(int a, int b) {
  return a + b;q
  }
  int sub(int a, int b) {
  return a - b;
}
int main() {
  MathFunction f = add; // 2
  std::cout << "Result of addition: " << f(5, 3) << std::endl;
  f = sub; // 2
  std::cout << "Result of subtraction: " << f(5, 3) << std::endl;
  return 0;
}
\end{cpp}

{\footnotesize
注释1:类型名称与类型定义不同

注释2:用法相同
}

如果别名基于内置类型(例如 int),则可读性会增强。例如,可以将别名 age 定义为无符号整数:

\begin{cpp}
using age = unsigned int;
\end{cpp}

不要使用全局命名空间

\begin{cpp}
using namespace std;
\end{cpp}

很多代码中(在源文件的顶部),包括大多数教科书和许多在线教师都忽略了这条建议,使用这种全局命名空间显然是为了避免冗余的编码,比如:

\begin{cpp}
std::cout << ...
\end{cpp}

std(标准)命名空间相当大,包含许多标识符。如果使用全局包含,代码具有同名标识符的可能性很高。猜猜编译器会使用哪一个?我也不知道。冲突和歧义的可能性很大,可能会导致未定义的行为。

\mySamllsection{建议}

\begin{itemize}
\item
始终使用 using 关键字而不是 typedef。

\item
将 typedef 替换为using。

\item
值得为代码的可阅读性付出努力。

\item
避免在代码中包含全局命名空间。
\end{itemize}
