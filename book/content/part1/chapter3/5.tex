这种错误可以提高正确性、可读性和有效性;某些情况下,也会影响性能。初始化变量的方法有很多种,每种方法都有优点和缺点;但是,这些方法必须一致。

\mySamllsection{问题}

C++ 使用从 C 继承的相同技术来初始化变量。但在某些情况下,C++ 超越了 C 的限制,引入了其他形式的初始化。已经引入了几种方法,但没有一种方法能以相同的方式完成其他方法所做的事情。最终结果是 C++ 代码有多种初始化变量的方法,增加了认知负担。请记住,编程思维应该专注于正在解决的问题,并使用语言来表达解决方案——跟踪语言细节会增加不必要的负担,而不一致会降低效率。

以下代码是初始化整数vecotr的简单示例。经典 C++ 没有其他方法可以做到这一点,因此需要代码来设置。真是无语!

\filename{清单3.10 容器和简单变量的典型初始化}

\begin{cpp}
int main() {
  std::vector<int> v;
  for (int i = 0; i < 10; ++i)
    v.push_back(i); // 1
  v.push_back(33); // 2
  v.push_back(42); // 2
  int count = v.size(); // 3
  std::cout << "the vector has " << count << " elements\n";
  return 0;
}
\end{cpp}

{\footnotesize
注释1:使用循环来计算一致的值

注释2:不一致的值需要单独插入

注释3:通过赋值进行初始化
}

\mySamllsection{分析}

如果只有 push\_back,那么每个初始化问题看起来都像 push\_back 问题。这种情况表明 C++ 有动力添加其他初始化方法;代码很笨重,需要付出很多才能完成一个简单的任务。

\mySamllsection{解决}

使用统一初始化有几个原因;最优秀的是其一致性,在所有情况下都易于理解和适用。特别是对于容器,使用统一初始化的能力都非常出色。清单 3.11 中的代码演示了对vector和独立变量使用这种方法。当初始化值放在括号内时,就会成为首选方法。

统一初始化使用括号将初始化值括起来。vector 使用此方法,count 变量也使用此方法。随着开发人员变得更加熟练,这种统一性使编程变得更容易。

\filename{清单3.11 容器和简单变量的统一初始化}

\begin{cpp}
int main() {
  std::vector<int> v {0, 1, 2, 3, 4, 5, 6, 7, 8, 9, 33, 42}; // 1
  auto count {v.size()}; // 2
  std::cout << "the vector has " << count << " elements\n";
  return 0;
}
\end{cpp}

{\footnotesize
注释1:容器(复合变量)的统一初始化

注释2:变量的统一初始化(简单变量)
}

统一初始化使用一种形式来初始化变量,这种一致性使读写代码更容易。

使用此方法还有其他优点:

\begin{itemize}
\item
不同变量类型之间的一致性

\item
不包括窄化转换 - 例如,double 类型的值不能初始化 int 类型的变量

\item
复杂的数据结构更容易初始化

\item
构造函数和容器支持初始化列表
\end{itemize}

\mySamllsection{建议}

\begin{itemize}
\item
统一初始化可用于初始化变量,甚至是复杂变量。

\item
理解并使用这种方法,一致且简单。
\end{itemize}
