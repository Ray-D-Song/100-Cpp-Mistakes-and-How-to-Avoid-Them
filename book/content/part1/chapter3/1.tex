这种错误可能会对效率产生积极影响,但在某些情况下可能会对可读性产生负面影响。for 循环可以使用范围进行初始化,但 if 语句还不行。

\mySamllsection{问题}

需要搜索一个容器,以查找相应的键值。这会创建一个索引循环并用于索引每个值,直到找到键或到达容器末尾,且最好使用标准模板库中提供的算法。以下代码显示了一种典型的方法。预计需要编写或使用一个函数来搜索并返回一个布尔值,指示是否找到;这可以在条件中完成。if 语句使用返回的值在两个选项之间进行选择。这种情况下,不会返回布尔值,但会返回迭代器。需要测试迭代器,以确定它是否有效。

\filename{列表 3.1 基于索引的关键字搜索}

\begin{cpp}
int main() {
  std::vector<int> values;
  for (int i = 0; i < 10; ++i)
    values.push_back(i*2);
  int key = 4;
  std::vector<int>::iterator it =
  std::find(values.begin(), values.end(), key); // 1
  if (it != values.end()) // 2
    std::cout << key << " was found\n";
  else
    std::cout << key << " was not found\n";
  return 0;
}
\end{cpp}

{\footnotesize
注释1:确定是否找到了键

注释2:测试迭代器以确定是否找到了相应键
}

\mySamllsection{分析}

设置搜索、搜索本身,以及处理搜索返回结果,所需的代码冗长且包含大量关键字和结构。虽然这些都是必需的,但还可以更简单。感觉就像用 C 语言在编写代码(除了vector)。

\mySamllsection{解决}

现代 C++ 增强了 if 语句以允许初始化,可将变量限制在与 for 语句非常相似的语句中,此功能称为“带初始化器的 if 语句”。它可以减少代码的行数,并且不会引入可能在函数中意外重用的变量。此外,逻辑包含在条件中,读者能更容易找到它。

以下代码演示了如何在初始化程序中查找键值。此代码进一步演示了,如何使用标准模板库算法进行搜索。迭代器变量在分号之前声明和初始化 - 这是初始化程序部分。分号后面的条件代码检测迭代器是否引用结尾(未找到键);如果没有,则找到键值。条件值决定了语句其余部分所采用的执行路径。

使用带有初始化器的 if 语句,比传统的 if 语句更紧凑、更具表现力。在使用某些函数式语言时,也是一种温和的过渡。

\filename{列表 3.2 使用带有初始化器的 if 语句来搜索关键字值}

\begin{cpp}
int main() {
  std::vector<int> values;
  for (int i = 0; i < 10; ++i)
    values.push_back(i*2);
  int key = 4;
  if (auto it = std::find(values.begin(),
        values.end(),key); it != values.end()) // 1
    std::cout << key << " was found\n";
  else
    std::cout << key << " was not found\n";
  return 0;
}
\end{cpp}

{\footnotesize
注释1:初始化迭代器并判断是否找到键
}

\mySamllsection{建议}

\begin{itemize}
\item
使用带有初始化程序的 if 语句,减少设置决策所需的代码行数。

\item
变量应仅在其使用范围内,而不能超出该范围;带有初始化程序的 if 语句的好处是,限制变量的范围。
\end{itemize}















