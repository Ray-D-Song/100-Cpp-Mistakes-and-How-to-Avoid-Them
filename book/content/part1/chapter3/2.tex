这个错误会影响效率。可读性通常也会得到增强,但当开发人员希望了解确切数据类型时,这些信息不透明。

\mySamllsection{问题}

以下清单中的代码展示了,索引循环和测试搜索结果的典型方法。这些传统用法已刻在我们的大脑中,但传统的并不总是正确的。

\filename{清单3.3 典型的基于索引的循环和测试变量}

\begin{cpp}
int main() {
  std::vector<int> values;
  for (int i = 0; i < 10; ++i) // 1
    values.push_back(i*2);
  int key = 4;
  std::vector<int>::iterator it =
  std::find(values.begin(), values.end(), key) // 2
  if (it != values.end())
    std::cout << key << " was found\n";
  else
    std::cout << key << " was not found\n";
  return 0;
}
\end{cpp}

{\footnotesize
注释1:带有循环控制变量的循环

注释2:数据类型正确,但直观否?
}

\mySamllsection{分析}

许多解决日常问题的传统方法(例如:循环和变量类型)反映了语言的原始状态。过去,编译器不太擅长与开发人员交互,并且依赖于静态方法来声明变量 - 编译器要么批准,要么拒绝开发人员的选择。基于索引的循环非常典型,甚至不会引起注意,但其循环控制变量的方式,在技术上是错误的。

容器的索引始终从0开始,增加到容器到达末尾。当然,负值肯定不正确,但 int 变量可以模拟正值、零值和负值。如果吹毛求疵一下,整数不可能是正确的数据类型。用于测试 find 算法的结果,要求我们了解容器、其元素及其迭代器的类型——这三部分需要去确认。更糟的是,这些与正在解决的问题无关;这就是无谓是开销。这种开销会对效率产生负面影响,并影响代码的表达力。

\mySamllsection{解决}

现代 C++ 为编译器提供了更多的智能和灵活性。编译器知道数据类型是否正确,可以利用这些来确定类型并将其插入到代码中。清单 3.4 解决了前面提到的两个问题。

首先,编译器确定循环控制变量的有效类型 — 理想情况下是无符号的。不过,由于其初始化器为零将推断为 int。无符号不仅会给出两倍的范围,而且还会排除负索引值。从数学上讲是正确的类型,实际类型是 size\_t,通常实现为unsigned long 或 unsigned long long。可以表示任何对象的大小,请放心使用。使用 auto ,并让编译器确定其类型会更好。

其次,类型的复杂性可能会分散对所要解决的问题的注意力,并引入其他错误。编译器有确定正确数据类型的能力,可以减轻这种复杂性并避免引入异常。使用 auto 关键字可以利用编译器推断正确数据类型的能力,还可以将这些信息添加到代码中。开发人员专注于所要解决的问题,语言负责处理日常事务,这是一种协同作用——专业的人做专业的事。

\filename{清单3.4 使用auto让编译器推断出正确的类型}

\begin{cpp}
int main() {
  std::vector<int> values;
  for (auto i = 0; i < 10; ++i) // 1
    values.push_back(i*2);
  int key = 4;
  if (auto it = std::find(values.begin(),
          values.end(), key); it != values.end()) // 2
    std::cout << key << " was found\n";
  else
    std::cout << key << " was not found\n";
  return 0;
}
\end{cpp}

{\footnotesize
注释1:编译器会为循环控制变量选择正确的数据类型

注释2:让编译器找出确切的类型
}

auto 关键字有一个注意事项:倒计时循环。以下代码展示了以相反顺序显示元素的简单方法。

\filename{清单3.5 对auto的误用}

\begin{cpp}
int main() {
  std::vector<int> values;
  for (int i = 0; i < 10; ++i)
    values.push_back(i*2);
  for (auto i = values.size() - 1; i >= 0; --i) // 1
    std::cout << values[i] << '\n';
  return 0;
}
\end{cpp}

{\footnotesize
注释1:反向实现循环
}

此代码在系统上运行时会崩溃,出现了段错误。为什么?循环控制变量的推导类型 size\_t 是 unsigned。是否有不等于或大于0的无符号值?没有。因此,当 i 变为零时,下一个值是数据类型的最大值 — 超出了容器的边界。这种情况最容易解决的方法是将数据类型改为 int,而不是 auto。这时使用 size\_t 是错误的,这是类型推导可能引入的错误之一。由于应用程序二进制接口 (ABI) 的稳定性和向后兼容性,标准不会对此进行修改。

另一个影响性能的问题出现在使用 auto 推断值类型时:

\begin{cpp}
auto name = student->getName();
\end{cpp}

name 变量是学生姓名的副本,但引用会更好。引用代表学生姓名 std::string 的别名,而不是副本,从而节省了对复制构造函数的调用。代码如下所示:

\begin{cpp}
auto& name = student->getName();
\end{cpp}

这种区别也出现在使用 auto 在基于范围的 for 循环中推导循环控制变量时。变量始终是元素的副本。虽然这鼓励元素的只读特性,但如果元素类型大于指针,则效率会很低。此代码演示了对 students 列表进行迭代的首选方式:

\begin{cpp}
for (auto& student : students) ...
\end{cpp}

\mySamllsection{建议}

\begin{itemize}
\item
使用 auto 进行变量声明。

\item
使用 auto 关键字可使模板受益匪浅。

\item
在数据类型大于指针的情况下,请使用 auto 引用,避免复制。

\item
开发人员很少需要知道确切的类型 - 如果需要,应将其写出来。
\end{itemize}
