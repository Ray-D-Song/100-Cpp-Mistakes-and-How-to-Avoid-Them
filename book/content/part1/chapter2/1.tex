这个错误主要针对性能。因为数据通常会被复制,所以将信息从一个区域传输到另一个区域可能代价高昂,从而产生两个(或更多)的数据版本。

\mySamllsection{问题}

数据必须在一个地方创建或构建,而在另一个地方进行分析或操作。将数据从一个地方复制到另一个地方效率很低,复制少量数据无关痛痒,但复制大量数据就会造成问题,尤其是频繁复制的情况下。

如果多个对象必须拥有数据,则可将数据共享。可以通过传递指针副本来实现大量共享数据,从而允许各种代码段访问数据,而无需复制数据。并且,数据有独占性。因为指针是一种共享机制,所以传递这种数据的指针可能很危险。现代 C++ 为这种情况提供了独享所有权的智能指针。

数据通常由值,而非指针拥有和管理。清单 2.1 中,TextSection在一个实例中创建并初始化,然后传递给另一个实例。p1 对象独占 TextSection,所有权和数据“转移”到 p2,这是意图。结果是 p2 拥有数据的单个副本,但成本高昂。请注意,源对象的数据和headers被销毁,保留了唯一所有权要求,但打破了传递 const 引用的做法。

\filename{代码清单2.1 保持唯一性的赋值操作符,但代价高昂}

\begin{cpp}
class TextSection {
  // 假设有一段巧妙的实现
};

class Page {
private:
  TextSection* headers;
  TextSection* body;
public:
  Page(TextSection* h) : headers(h), body(new TextSection()) {}
  Page(Page& o);
  ~Page() { if (body) delete body; }
};

Page::Page(Page& o) {
  if (this == &o)
    return;
  body = new TextSection(*o.body);
  delete o.body;
  o.body = 0;
  headers = o.headers;
  o.headers = 0;
}

int main() {
  Page p1(new TextSection());
  Page p2 = p1; // 1
  return 0;
}
\end{cpp}

{\footnotesize
注释1:未定义行为的赋值操作
}

\mySamllsection{分析}

数据的副本归 p2 所有,但并非原始数据。这会导致产生两份副本,并且需要花时间来复制。如果数据量很大,复制的时间可能会相当长。复制构造函数会销毁复制的数据,从而将副本数减少为一份,但无法最大限度地降低复制成本。转移所有权的努力值得称赞,但可以更高效。我们需要的是一种无需复制,即可转移数据的方法。

\mySamllsection{解决}

现代 C++ 提供了一种称为 右值 引用的新引用类型(“右引用”),引用操作符右侧的值。右值是赋值的源,不能是赋值的目标。尽管右侧不是赋值目标,也可以对其进行修改。临时变量也是右值,所以修改它们是没有意义的,但“窃取”它们的值是合法的。新语法将包含一个双引用字符 (\&\&) 来指示移动语义。

该语言提供了添加移动构造函数和移动赋值操作符的功能。这些新功能旨在在资源从源转移到目标时保留独占所有权,并增加了移动数据而不是复制数据的好处。由于目标在移动后拥有转移的数据,所以源中数据失效,并避免错误访问。

\filename{清单2.2 开销低且保持唯一性的赋值操作符}

\begin{cpp}
class TextSection {
  // 假设有一段巧妙的实现
};

class Page {
private:
  TextSection* headers;
  TextSection* body;
public:
  Page(TextSection* h) : headers(h), body(new TextSection()) {}
  Page(const Page& o) : headers(o.headers), body(new // 1
  TextSection(*o.body)) {}
  Page(Page&& o) : headers(o.headers), body(o.body) { // 2
    o.headers = nullptr; // 3
    o.body = nullptr;
  }
  ~Page() { delete body; }
};

int main() {
  Page p1(new TextSection());
  Page p2 = std::move(p1); // 4
  return 0;
}
\end{cpp}

{\footnotesize
注释1:可移动的复制构造函数

注释2:源资源重新分配给目标。

注释3:源资源失效。

注释4:这看起来有点奇怪,但语义正确。
}

移动意味着源对象将其内容释放到目标对象。一些实现使用交换函数将源数据移动到目标,将目标数据移动到源。由于数据重新分配,源对象不再拥有其原始数据,但拥有目标的数据通常可以接受。如果不使用交换操作,最好的方法是确保源中所有移动的数据无效。

源对象未破坏的情况下,其先前的数据不能访问。因此,无效化源对象中的数据,以避免在移动后的无意使用。有时,可以将指针设置为 nullptr(或 0)。

另一个不错的功能是,当提供移动构造函数和赋值操作符时,编译器会尽一切努力使用它们,开发人员不必决定何时可以使用。当源是 左值(操作符左侧的可赋值值)时,将使用标准复制构造函数和赋值操作符。

std::move 函数模板对于正确移动至关重要。业界对此的流行语是“移动不动”(Move doesn’t move.)。那么,移动为何如此重要?移动将左值转换为右值,这使得其符合移动语义。如果不这样做,将发生标准赋值,而不是移动赋值。使用复制和移动操作符,类可以提供共享赋值(或构造)或独占赋值(或构造),这使得语义更加清晰。

\mySamllsection{建议}

\begin{itemize}
\item
向类添加移动构造函数和赋值操作符,以独占方式转移资源所有权。

\item
通过包含移动构造函数和移动赋值操作符,将三规则(复制构造函数、赋值操作符、析构函数)扩展为 五规则。
\end{itemize}
















