本章内容

\begin{itemize}
\item
移动语义

\item
类成员

\item
基于范围的循环

\item
智能指针
\end{itemize}

随着 C++11 标准的出现,C++ 增加了一些更改以增强语言功能、改进标准模板库 (STL)、提高性能,以及简化语法和表达能力。这些改进包括针对并发性的重大更新,例如线程和任务管理、错误检测机制、时间顺序与日历功能的增强,以及编译时计算的支持。

C++11 中引入的 auto 关键字实现了类型推断,从而简化代码并提高了可读性和有效性。基于范围的 for 循环使集合上的迭代更加简便,提升了代码清晰度,并减少了错误的发生。nullptr 文字通过明确区分空指针和整数零,增强了代码的安全性。

STL 的功能得到了进一步增强。智能指针有助于实现安全的内存管理,并有效减少内存泄漏的风险。移动语义优化了资源处理流程,通过减少不必要的复制操作显著提高了性能。新容器的引入丰富了数据结构选项,提升了数据处理的灵活性和效率。Lambda 表达式以其简洁而富有表现力的特点,进一步提高了代码的可读性和有效性。

性能方面的改进显著提升了程序的效率和资源利用率。右值引用能够更高效地处理临时对象,最大限度地减少不必要的开销,从而缩短执行时间并提高资源管理效率,进而提升 C++ 程序的整体性能。

C++11 在语法简化和表达能力方面也取得了重要进展,使代码更具可读性和可写性。Lambda 表达式支持内联函数定义,提高了代码清晰度并减少了对辅助函数的需求。可变参数模板可以灵活处理可变数量的模板参数,实现了更通用和可重用的代码。基于范围的 for 循环为迭代容器提供了更直观的语法。这些改进使开发人员能够编写更具表现力、更高效和更灵活的代码。

本章和接下来的两章讨论了这些改进中的大部分。未涵盖重大改进和增强,但可以在许多书籍、课程和互联网搜索中找到。C++11 标准中的增强范围非常值得研究,它们为 C++17 和 C++20 标准提供了基础,添加了增强、添加、弃用、删除和新语言功能。随着时间的推移,C++ 在不断演进和完善。

现代 C++ 无法避免开发人员犯错,但它支持一些使错误变得更加困难的功能。例如,基于范围的 for 循环不受 off-by-one\footnote{译者注:一种特殊的溢出漏洞} 的影响。以下所列的优点旨在通过改进语言特性来解决经典错误,即使只是其中的一小部分改进,也能在开发过程中带来显著的好处。







