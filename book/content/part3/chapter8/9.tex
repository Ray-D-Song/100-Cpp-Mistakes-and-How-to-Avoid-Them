这个错误主要针对正确性。在虚函数中误用构造函数和析构函数会严重影响其他特性。

面向对象编程的显著优势之一是能够创建相关类的层次结构。这些类通过扩展其基础类,使其具有额外的特定行为,从而与其基础类相关联。这种能力源于根据定义特定行为的接口来处理相关类。专门的类会修改接口函数,从而为其特定类型产生有意义的结果。

这种专门化允许对一般函数的行为进行细粒度控制。例如,Shape 基类可能具有 Circle 派生类。一般而言,所有形状都应该具有 area 方法,但如何实现计算完全取决于正在处理的形状类型。这种定义一般概念(在基类中)并用具体内容(在派生类中)覆盖该行为的想法称为 runtime polymorphism。多态性是面向对象编程的三大公认支柱之一。(我也主张抽象!)了解特定编译器如何实现多态性没有帮助;从概念上了解如何实现多态性有助于理解为什么在构造函数中调用虚函数是不合适的。

考虑一下继承是一种自上而下的信息流。基类知道什么(状态,如实例变量)和做什么(行为,如方法)向下流向派生类。多态行为是自下而上搜索特定函数定义。从概念上讲,对适当虚拟函数的搜索从特定的派生类类型开始,并沿着层次结构向上移动。使用该函数的第一个定义版本。

编译器确定何时需要非虚拟实例方法的地址,并留出空间用于插入该方法的地址。当实例调用这些方法之一时,编译器或链接器会将地址插入可执行代码中。在运行时,该地址用于定位方法的代码。但是,对于虚拟函数,需要的不仅仅是这种方法。

虚函数的存在会导致编译器为该类建立一个新表,该表保存每个虚函数的地址在其中声明。这些地址不会像非虚拟方法那样在编译或链接期间直接插入到代码中。创建实例时,每个虚拟函数的指针用于引用正确的虚拟函数。实际细节很复杂,但以下概念模型是一个可行的想法。构造函数确定哪个类表代表正确的虚拟函数并调整指针以引用它。
构造顺序是精确定义的。假设一个三级层次结构:基类是 A,中间类是 B,最外层派生类是 C。当创建 C 的实例时,将调用其构造函数。

构造函数做的第一件事是调用 B 的相应基类构造函数。B 构造函数通过调用其基类构造函数来初始化 A 部分,从而开始执行。在 A 构造函数完成初始化 A 部分后,B 构造函数继续执行并完成初始化 B 部分。
只有在 B 构造函数返回后,C 构造函数才会恢复执行并初始化对象的C 部分。图 8.1 显示了此构造顺序,其中构造函数首先调用其基类构造函数。

对于具有虚函数的类,认为构造函数负责将指针调整到正确的类表; 这在技术上并不正确,但它解释了为什么派生类有最终决定权。由于A 构造函数首先运行完成,因此将使用定义虚函数的 A 类表初始化表。如果 B 也定义了虚函数,则 A 类的表指针将被 B 类表的位置覆盖。C 类也是如此。因此,当通过基类指针处理 C 实例时,用于查找正确虚函数的表将是最近调整的表。因此,即使 C如果实例被视为通用的 A 对象,则写入实例的最后一个表指针就是调用虚拟函数时调用的指针。因此,当调用虚拟函数时,通用的 A 指针或引用可以表现得像 C 实例。

\myGraphic{0.5}{content/part3/chapter8/images/1.png}{图 8.1 显示构造顺序的三级层次结构}

这种构造顺序的结果是,实际对象(本例中为 C 实例)将仅在其构造的某个时刻调整其类表指针。在此调整之前的任何时刻,对虚拟函数的调用都会调用不正确且意外的版本。当构造函数或析构函数正在执行时,对象要么正在构造,要么正在销毁。由于对象仍然需要完全构造,不能认为已经准备好,并且不能保证类不 变性。

\mySamllsection{问题}

假设定义了一个继承层次结构,其中不同的人被赋予不同的尊称。在这所学校,研究生受到高度尊敬,学生受到一定程度的尊敬,而普通人则几乎不受到尊敬。由于这个层次结构需要表现得一般化,因此 GradStudent 和 Student 实例将根据需要替换为 Person 实例。但是,在这种情况下,这些被替换的实例必须通过确定适当的尊称来具体表现。

清单 8.17 中的示例代码展示了一个案例,本来应该调用实际类型的getHonorific 虚拟方法并输出正确的标题。但是,由于这个错误,它失败了。

此示例显示了一个三级层次结构。每个构造函数都会初始化其实例变量,然后调用 print 方法。当此方法执行时,它会调用虚拟getHonorific 方法。输出显示每次调用 print 都会输出不同的消息。getHonorific 虚拟方法在每个部分的构造过程中将其函数指针添加到表中,这是 print 方法调用的特定函数。

\filename{清单8.17 从构造函数调用虚函数}

\begin{cpp}
class Person {
private:
  std::string name;
public:
  Person(const std::string& name) :
  name(name) { print(); }
  void print() const { std::cout << getHonorific() << name << '\n'; }
  const std::string getName() const { return getHonorific() + name; }
  virtual std::string getHonorific() const { return ""; }
};

class Student : public Person {
private:
  std::string year;
public:
  Student(const std::string& name, const std::string& year) : Person(name),
    year(year) { print(); }
  std::string getHonorific() const { return year + " "; }
};

class GradStudent : public Student {
private:
  bool candidate;
public:
  GradStudent(const std::string& name, const std::string& year,
    bool candidate) : Student(name, year), candidate(candidate)
  { print(); } // 1
  std::string getHonorific() const {
    return candidate ? "candidate " : "";
  }
};

int main() {
  GradStudent aimee("Aimee", "second year", true);
  return 0;
}
\end{cpp}

{\footnotesize
注释1:在构造函数中调用虚方法;函数版本很可能不正确
}

\mySamllsection{分析}

代码的结果表明,在构造每个部分后,将调用其版本的getHonorific,而不是实际类型的版本,因为实例尚未完全构造。因此,虚拟函数的版本和结果完全取决于对象的哪个部分正在调用该方法。以下输出显示了构造对象时的各种敬语,每个类类型一行:

\begin{shell}
Aimee
second year Aimee
candidate Aimee
\end{shell}

另一个问题是虚拟方法可能依赖于给定部分的状态数据。如果构造函数尚未完成该部分,则无法保证数据处于有效状态。很可能不是。如果调用虚拟函数并且它依赖于此不完整的数据,则结果可能是不确定的。

\mySamllsection{解决}

解决方案是在对象完全构造后调用虚拟函数。只有这样,虚拟函数表才会被正确初始化,行为才会正确实现多态。以下代码演示了此修复。

\filename{清单8.18 没有虚函数调用的构造函数}

\begin{cpp}
class Person {
private:
  std::string name;
public:
  Person(const std::string& name) : name(name) {}
  void print() const { std::cout << getHonorific() << name << '\n'; }
  std::string getName() const { return getHonorific() + name; }
  virtual std::string getHonorific() const { return ""; }
};

class Student : public Person {
private:
  std::string year;
public:
  Student(const std::string& name, const std::string& year) : Person(name), year(year) {}
  std::string getHonorific() const { return year + " "; }
};

class GradStudent : public Student {
private:
  bool candidate;
public:
  GradStudent(const std::string& name, const std::string& year,
      bool candidate) :
    Student(name, year), candidate(candidate) {}
  std::string getHonorific() const { return candidate ? "candidate " : ""; }
};

int main() {
  GradStudent aimee("Aimee", "second year", true);
  aimee.print(); // 1
  return 0;
}
\end{cpp}

{\footnotesize
注释1:对象构造后调用虚函数
}

对于正确设计的类,析构函数的工作顺序与构造函数相反。当对指针或引用调用析构函数时,最先调用派生类的析构函数。在其执行结束时,它会调用其基类的析构函数。这会一直渗透到层次结构的顶部,从而保证对象以与构造相反的顺序被销毁。

如果在析构过程中调用虚拟方法,则最外层派生类中的状态信息可能无效,具体取决于其析构函数的行为。但是,虚拟方法会期望这些数据处于有效状态。由于无法保证这一点,因此应始终避免在析构函数中使用虚拟方法。

\mySamllsection{建议}

\begin{itemize}
\item
确保在实例完全构造之前未调用类的任何虚拟函数;切勿从其构造函数中调用类的虚拟函数。

\item
请记住,构造顺序会影响状态信息的有效性以及将调用哪个版本的虚拟函数;在调用任何虚拟函数之前,让构造完成。

\item
请记住,销毁顺序会影响虚函数可用状态信息的有效性,其中一些信息可能会被销毁或以其他方式无效;一旦销毁开始,就永远不要调用虚函数。
\end{itemize}


