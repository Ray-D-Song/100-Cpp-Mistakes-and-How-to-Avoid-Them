本章内容

\begin{itemize}
\item
如何确保在程序设计中维护类的不变量

\item
由于不加区别地使用旧的面向对象设计建议而产生的困难

\item
复制构造函数和复制赋值运算符之间的区别

\item
无法初始化内置类型变量的普遍问题
\end{itemize}

在完成第 7 章中讨论的建立类不变量的艰苦工作后,开发人员必须小心维护它。程序开发的几个方面提供了违反不变量的机会。重要的是要意识到其中一些可能性并保持警惕以防止它们的使用。

每当使用数据初始化或修改状态变量时,都有可能违反类不变性。C ++ 构造的灵活性和细粒度为影响数据提供了充足的机会,有时甚至会以令人惊讶的方式发生。最明显违反不变性的地方是任何状态数据发生变化时。然而,只有在这种情况下才会有时很明显。对象的构造和删除在某种程度上是一个隐蔽的过程,但在确定对象的内容(即不 变性)方面起着至关重要的作用。

继承提供了更多滥用数据和影响不变量的机会。C++ 对对象的创建、 每个部分的构造顺序以及在构造(或销毁)的哪个阶段可以使用哪些状态数据有着严格的规定。使用错误可能会带来灾难性的后果。数据复制函数提供了一些弄乱类实例的机会。不当使用可能很难被发现,但影响可能是巨大的。复制有两个方面:浅层和深层;不当使用这些方面会潜移默化地威胁类的不变量。

使用数组和其他数据容器取决于复制语义。正确使用它们将确保对象保持原始状态并保留对象数据的含义。正确使用虚拟函数对于多态性至关重要。此外,正确使用它们对于正确形成对象至关重要。使用这些函数时可能犯的错误各不相同,每次错误使用都会对正确的操作、 状态或两者产生不利影响。
