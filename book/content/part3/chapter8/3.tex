这个错误过于注重正确性而忽视了有效性,许多编程教科书延续了一种古老而熟悉的反模式。解释如何构建类时,作者经常建议为每个实例变量提供一个访问器和修改器。面向对象编程的早期,这认为是一种很好的做法。当前的教科书作者倾向于将这种做法延续。然而,这种做法往往会使类不变量失效;所以,我们需要一种更好的方法。

\mySamllsection{问题}

让我们考虑具有单个实例变量 radius 的 Circle 类,如以下清单所示。构造函数初始化变量,访问器返回其副本,而修改器修改它。

\filename{清单8.3 非验证的可修改类}

\begin{cpp}
class Circle {
private:
  double radius;
public:
  Circle() : radius(1) {}
  Circle(double radius) : radius(radius) {}
  double getRadius() { return radius; }
  void setRadius(double radius) { this->radius = radius; } // 1
};

int main() {
  Circle c1(2);
  std::cout << c1.getRadius() << '\n';
  c1.setRadius(-1);
  std::cout << c1.getRadius() << '\n';
}
\end{cpp}

{\footnotesize
注释1:由于没有验证检查,值的范围不受限制
}

根据教科书的方法,这是正确的设计;但存在三个问题:

\begin{itemize}
\item
未经验证的输入值

\item
坚持所有实例变量可变

\item
设置器和构造函数参数间的重复
\end{itemize}

遗留代码中充满了这种模式,了解这些问题将有助于确定何时进行代码改进。

脆弱的设置器允许对实例变量进行无限制的修改,从而显著影响正确性。用户不应负责了解变量的正确值范围。类必须保持类的这一部分不变,并提醒用户注意不正确的值。引入无关的方法可能会影响可读性,从而掩盖重要的方法。使实例变量可变会影响正确性,某些类在构造后不应更改其实例变量。

坚持要求开发人员编写不必要的方法,并引入破坏类不变性的代码会影响有效性。有效值范围的知识通常在构造函数和变量之间重复,很容易出现分歧。

有效的编码试图在一个地方消除和隔离这些重复。

\mySamllsection{分析}

虽然脆弱的访问器会分散注意力但相对无害,但脆弱的修改器可能很危险。修改器负责维护构造函数建立的类不变量。任何错误的值引入类都会导致实例不一致,并可能导致未定义行为。所以,修改器必须验证输入值的正确性。

此外,实例变量可能不需要修改。虽然我们可能倾向于改变实例变量,但这条路比看上去更曲折。在编写变量之前,必须采取两个基本步骤。首先,确定实例变量是否应可变。其次,确定合法值的范围。

许多情况下,第一个问题的答案是否定的。如果是这样,就没有理由编写一个变量设置器;这样的变量器是危险的和多余的。开发人员应该确定半径在 Circle 类中是否正确可变。这种情况下,不太可能;如果需要具有不同半径的圆,用户应该创建一个新实例。

对于剩余的变量,请仔细考虑变量的适当值范围。在将参数值分配给实例变量之前,编写验证代码以确保满足这些界限。考虑如何解决超出范围的参数值问题。使值无效,这通常是最好的方法。另一种选择是忽略无效参数值,而不执行任何操作;但当构造函数调用变量来设置初始值时,这种方法不起作用。

\mySamllsection{解决}

以下代码展示了一种更好,但并不理想的方法。它试图通过确保构造函数和变量在用户传递无效值时,抛出异常来解决验证问题。没有解决可变性或重复问题。

\filename{清单8.4 具有重复验证的类}

\begin{cpp}
class Circle {
private:
  double radius;
public:
  Circle() : radius(1) {}
  Circle(double radius) {
    if (radius < 0) // 1
      throw std::invalid_argument("radius is negative");
    this->radius = radius;
  }
  double getRadius() const { return radius; }
  void setRadius(double radius) {
    if (radius < 0) // 1
      throw std::invalid_argument("radius is negative");
    this->radius = radius;
  }
};
int main() {
  Circle c1(2);
  std::cout << c1.getRadius() << '\n';
  c1.setRadius(-1);
  std::cout << c1.getRadius() << '\n';
}
\end{cpp}

{\footnotesize
注释1:重复确定有效值
}

通常,构造函数和修改器有共同的代码。构造函数负责建立类不变量,包括初始化实例变量(当前未初始化)的值。修改器负责修改实例 变量(现在已初始化)的值。可以通过让构造函数调用修改器来消除这种重复,都必须保持类不变量,因此知识重复是不可避免的。误读“不要重复自己”(DRY)原则可能会导致人们仅关注代码重复;开发人员应该关注知识重复。范围检查代码应放在修改器中,构造函数应调用修改器。

以下代码解决了前面讨论的三个问题(未验证的参数值、不加区分的可变性和重复知识),并以最小的努力解决了每个问题。

\filename{清单8.5 一个具有验证和单一知识源的类}

\begin{cpp}
class Circle {
private:
  double radius;
  static double validateRadius(double radius) { // 1
    if (radius < 0)
      throw std::invalid_argument("radius is negative");
    return radius;
  }
public:
  Circle() : radius(1) {}
  Circle(double radius) : radius(validateRadius(
    radius)) {} // 2
  double getRadius() const { return radius; }
  void setRadius(double r) { radius =
  validateRadius(r); } // 3
};

int main() {
  Circle c1(2);
  std::cout << c1.getRadius() << '\n';
  Circle c2(-1); // 4
  std::cout << c2.getRadius() << '\n';
}
\end{cpp}

{\footnotesize
注释1:验证知识的单一来源

注释2:代码依赖于知识来源

注释3:代码依赖于知识来源;如果需要不变性,请删除此方法

注释4:这会引发异常,提醒开发人员注意问题
}

私有的 validateRadius 方法隔离了对适当半径值的知识,构造函数通过引用此方法初始化实例变量,并实例化具有不同半径的圆。每个不可变的实例变量必须没有变量;消除现有的变量。完成此步骤是一种练习,可以摆脱思维混乱,并让代码免除不必要的维护成本。

\mySamllsection{建议}

\begin{itemize}
\item
使尽可能多的实例变量不可变,并消除变量。

\item
验证每个输入参数,以确保其值在类不变量定义的范围内。

\item
如果可能,抛出无效值的异常;否则,忽略无效值。让构造函数和变量调用一个通用的验证方法,将知识隔离到一个地方。

\item
确保每个实例变量都初始化为有意义的值。
\end{itemize}


















