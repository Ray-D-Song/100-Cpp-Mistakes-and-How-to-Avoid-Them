这个错误会影响正确性和可读性,编码不当会对类不变量和代码的适用性产生不利影响。

经常会将信息从一个类的实例复制到另一个类的实例。此操作的常用方法是使用复制赋值操作符,但也必须考虑复制构造函数。如果误解了两者的语义,很容易认为两者在做同样的事情,而其中一个是多余的。然而,虽然操作非常相似,但含义却不同。

\mySamllsection{问题}

创建类时,必须考虑赋值和从现有对象复制的语义。如果开发者既没有编写复制构造函数,也没有编写复制赋值操作符,则编译器默认提供这两者。如果提供了其中之一,则另一个自动生成。

两者的默认实现都是对每个实例变量,从源到目标对象进行成员级复制。某些情况下,这已经足够了,而且是正确的。但是,开发人员仍然必须仔细考虑其中的含义。忽略它们并轻率地让编译器完成繁重的工作是一种糟糕的方法。

使用动态资源(如内存、句柄、连接、套接字等)时,应使用编译器默认方法的替代方法。开发人员必须确保在复制操作期间,正确处理动态资源。

当创建新对象且当前未初始化时(无论是基于堆栈还是基于堆),将调用复制构造函数。当使用源对象值覆盖现有的已初始化目标对象时,将调用复制赋值操作符。区别在于目标对象是未初始化的,还是已初始化的——两者的含义略有不同。

清单 8.8 中的代码使用动态资源来处理读取文件(省略大部分代码),并使用现有的文件读取器作为初始化源。编译器提供了其默认版本,因为开发人员没有提供复制构造函数或复制赋值操作符。每个版本都使用浅复制语义,将值从源实例变量复制到目标实例变量。

\filename{清单8.8 由默认复制构造函数和复制赋值操作符处理的资源}

\begin{cpp}
class InputReader {
private:
  std::string* filenames; // 1
  int count;
public:
  InputReader() : filenames(new std::string[5]), count(0) {}
  ~InputReader() { delete[] filenames; }
  void addFileName(const std::string& n) {
    if (count == 5)
      throw std::runtime_error("too many files specified");
    filenames[count++] = n;
  }
  int getCount() const { return count; }
  const std::string& operator[](int index) {
    return filenames[index];
  }
};
int main() {
  InputReader ir1;
  ir1.addFileName("statistics.txt");
  InputReader ir2(ir1); // 2
  ir2.addFileName("numbers.txt");
  InputReader ir3 = ir2; // 3
  ir3.addFileName("categories.txt");
  ir1.addFileName("questions.txt");

  for (int i = 0; i < ir1.getCount(); ++i)
    std::cout << ir1[i] << '\n';
  for (int i = 0; i < ir2.getCount(); ++i)
    std::cout << ir2[i] << '\n';
  for (int i = 0; i < ir3.getCount(); ++i)
    std::cout << ir3[i] << '\n';
  return 0;
}
\end{cpp}

{\footnotesize
注释1:为了简单而采用的糟糕做法;STL 提供了更好的方法

注释2:使用浅复制语义调用编译器默认的复制构造函数

注释3:使用浅复制语义调用编译器默认的复制赋值操作符
}

\mySamllsection{分析}

第一个读取器添加 statistics.txt 文件,然后用于初始化第二个读取器; 第二个读取器添加 numbers.txt 文件,然后用于初始化第三个读取器;第三个读取器添加 categories.txt 文件,预计有三个文件需要读取。第一个读取器然后添加一个名为 questions.txt 的新文件。以下输出显示事情没有按预期进行(为清晰起见添加了空行):

\begin{shell}
statistics.txt
questions.txt

statistics.txt
questions.txt

statistics.txt
questions.txt
categories.txt
Segmentation fault
\end{shell}

第一个读者应该是唯一可以访问 questions.txt 文件的人;但是,输出显示二级和三级的读者也可以访问 questions.txt。由于复制和赋值仅复制指针值,因此每个都指向相同的共享数据。ir2 读者已失去对numbers.txt 文件的访问权限。当 ir1 添加另一个文件时,它认为该数组有一个有效元素。因此,当添加 questions.txt 文件时,它将成为共享数组中的第二个元素,覆盖 numbers.txt 元素。

默认复制构造函数和复制赋值操作符对 filenames 指针进行了浅拷贝,共享了数组数据。每个读取器都会独立于其他读取器更新下一个元素,从而导致数据损坏。这种情况会导致意外输出和危险的错误行为,包括重复删除。

\mySamllsection{解决}

要使此示例正常工作,必须编写复制构造函数和复制赋值操作符,来正确处理文件名的动态数组。复制构造函数将分配一个新数组,执行现有元素的复制,并设置计数变量。

对于复制构造函数,新实例未初始化,必须在初始化列表中创建一个新的动态数组。现有值的复制在构造函数主体中完成。经过此更改后,代码可以正常工作并符合预期。复制赋值操作符必须采用不同的方法,尽管大部分功能相同。由于目标对象已初始化,已经具有其非共享动态数组,无需分配新的动态数组。只需要复制元素并初始化计数值,如以下清单所示。

\filename{清单8.9 正确编写的复制构造函数和赋值操作符}

\begin{cpp}
class InputReader {
private:
  std::string* filenames;
  int count;
public:
  InputReader() : filenames(new std::string[5]), count(0) {}
  ~InputReader() { delete[] filenames; }
  InputReader(const InputReader& r) : filenames(new std::string[5]),
      count(r.count) { // 1
    for (int i = 0; i < r.count; ++i)
      filenames[i] = r.filenames[i];
  }
  InputReader& operator=(const InputReader& r) { // 2
    for (int i = 0; i < r.count; ++i)
      filenames[i] = r.filenames[i];
    count = r.count;
    return *this;
  }
  void addFileName(const std::string& n) {
    if (count == 5)
      throw std::runtime_error("too many files specified");
    filenames[count++] = n;
  }
  int getCount() const { return count; }
  const std::string& operator[](int index) {
  return filenames[index];
}
};
int main() {
  InputReader ir1;
  ir1.addFileName("statistics.txt");
  InputReader ir2(ir1); // 3
  ir2.addFileName("numbers.txt");
  InputReader ir3 = ir2; // 4
  ir3.addFileName("categories.txt");
  ir1.addFileName("questions.txt");

  for (int i = 0; i < ir1.getCount(); ++i)
    std::cout << ir1[i] << '\n';
  for (int i = 0; i < ir2.getCount(); ++i)
    std::cout << ir2[i] << '\n';
  for (int i = 0; i < ir3.getCount(); ++i)
    std::cout << ir3[i] << '\n';
  return 0;
}
\end{cpp}

{\footnotesize
注释1:分配一个新数组并复制元素,而非指针

注释2:复制元素,而非指针

注释3:调用用户编写的复制构造函数

注释4:调用用户编写的复制赋值操作符
}

通过这些更改,行为得到纠正。以下输出符合预期(为清晰起见添加了空行):

\begin{shell}
statistics.txt
questions.txt

statistics.txt
numbers.txt

statistics.txt
numbers.txt
categories.txt
\end{shell}

对于某些类,这种复制赋值操作符方法可能需要修改。清单 8.9 中的代码展示了一种情况,其中所有实例的数组大小都相同。假设有一个类,其中数组大小不同。复制构造函数的工作方式相同,会创建一个与源对象大小相同的新数组。复制赋值操作符的目标对象,将具有不同大小的数组,因此需要删除现有数组,并且需要根据源的大小和复制的每个元素创建一个新数组。请保持警惕,以正确销毁动态元素。如果元素是类实例,则需要销毁。尽管它可能会自动发生,但在极少数情况下可能需要调用析构函数。

清单 8.10 中的代码展示了解决此问题的变体,其中假设类添加了一个表示最大数组大小的容量实例变量(这是比硬编码大小更好的设计)。此示例有一个潜在的问题;如果进行自赋值,此代码将失败。

\filename{清单8.10 处理不同大小的源数组和目标数组长度}

\begin{cpp}
InputReader& operator=(const InputReader& r) {
  if (capacity != r.capacity) {
    delete[] filenames;
    filenames = new std::string[r.capacity];
    capacity = r.capacity;
  }
  for (int i = 0; i < r.capacity; ++i)
    filenames[i] = r.filenames[i];
  count = r.count;
  return *this;
}
\end{cpp}

现代 C++ 修改了默认成员的生成方式。如果仅提供了移动构造函数,则编译器将不会自动生成复制构造函数、复制赋值操作符或移动赋值操作符。如果仅提供了移动赋值操作符,则编译器将不会自动生成复制构造函数、复制赋值操作符或移动构造函数。期望编译器填充缺失的成员时,请务必了解这些限制。以下代码是对清单 8.9 中代码的补充,演示了移动构造函数和移动赋值操作符:

\begin{cpp}
InputReader(InputReader&& r) : filenames(r.filenames), count(r.count) {
  r.filenames = nullptr;
  r.count = 0;
}

InputReader& operator=(InputReader&& r) {
  if (this != &r) {
    delete[] filenames;
    filenames = r.filenames;
    count = r.count;
    r.filenames = nullptr;
    r.count = 0;
  }
  return *this;
}
\end{cpp}

\mySamllsection{建议}

\begin{itemize}
\item
仔细考虑复制赋值操作符和复制构造函数的机制,并决定逐个成员复制是否足够。

\item
记住处理自我赋值;如果忽略,可能会出现未定义的行为。

\item
复制构造函数针对未初始化的内存,而复制赋值操作符针对已初始化的内存。

\item
仔细考虑如何在初始化期间,正确处理动态资源。
\end{itemize}










