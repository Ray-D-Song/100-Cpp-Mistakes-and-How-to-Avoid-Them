这个错误主要关注正确性、有效性和性能。理解这个习语之前,开发人员可能会说这会对可读性产生负面影响。然而,理解了这个习语,大多数人都会更喜欢这种方法。

类将具有两种类型的实例变量:原始(内置)和类实例。许多现代语言强制初始化所有变量,无论其位置或用途如何。C++ 并不总是确保初始化发生。某些情况下,实例变量已声明但没有定义的初始值。内存中的每个字节都有一些状态,因此这些变量将具有未定义的值。读取这些变量会导致未定义行为,先赋值然后读取就可以了。初始化的目的是,确保状态由程序明确定义。

初始化原始类型的三种常用方法:使用文字值分配变量;输入分配给变量的值;从另一个已初始化的变量分配变量,这些方法确保变量已声明并初始化。开发人员必须确保在第一次使用局部变量之前都保证这一点。

对于类,初始化是构造函数的职责;其主要工作是初始化变量,建立类不变量。构造函数必须接收每个参数的参数值,并使用其初始化实例变量或为其分配默认值。如果不是在每种情况下都这样做,则可能导致未定义行为。

\mySamllsection{问题}

许多问题都以某种方式与人有关。假设需要用 name 和 age 来建模 Person 类。这个简单类的以下实现编译后似乎运行良好,只是它使用了未定义的数据。除非手动检查人员的年龄,否则它不会使其错误显现出来。

\filename{清单8.21 带有未初始化内置实例变量的类}

\begin{cpp}
class Person {
private:
  std::string name; // 1
  int age; // 2
public:
  Person(const std::string& name) { this->name = name; } // 3
  void setAge(int age) { this->age = age; }
  friend std::ostream& operator<<(std::ostream&, const Person&);
};

std::ostream& operator<<(std::ostream& o, const Person& p) {
  o << p.name << " is " << p.age << " years old";
  return o;
}

int main() {
  Person joey("Joey");
  std::cout << joey << '\n';
  return 0;
}
\end{cpp}

{\footnotesize
注释1:一个实例变量,是一个类实例

注释2:一个实例变量,是一个原始类型

注释3:看似无辜的构造函数
}

开发人员应该坚持在构造函数中提供年龄(忽略变量是错误的格式)。目的是让用户代码通过调用 setAge 方法初始化年龄数据,从而确保 age 实例变量有效。当开发者需要记住这样做时,可能会出现问题。编译器无法警告未初始化问题,需要开发者自己找出问题,而这个问题很可能会在代码使用一段时间后出现。

有些编译器在调试模式下运行时会初始化变量,但在发布模式下不会。当然,“在我的机器上可以运行”的说辞并不会让用户满意。

\mySamllsection{分析}

类中的两个实例变量属于两个类别。std::string 对象是类实例,如果未尝试将值传递给它的构造函数。其默认初始化确保字符串为空,是一个合法的对象,但不包含对开发者的问题有用的值。等一下,名称怎么会是空字符串呢?构造函数主体将其初始化为参数值,该参数值不太可能是空字符串。确实如此,但这超出了实际操作的范围。

构造函数主体中的代码是赋值,而非初始化。此错误的介绍部分中,术语 initialization 的使用有些含糊,其中赋值用于初始化。

构造函数确保在调用构造函数主体之前,初始化所有类实例变量。在进入 Person 构造函数主体之前会调用默认的 std::string 构造函数,在构造函数主体中,赋值将参数值的副本绑定到实例变量。非常低效,变量初始化并赋值。执行了两个操作,但实际上只需要一个操作。更糟糕的是,如果参数不引用调用者的参数值,则会进行另一次复制以初始化参数。

age 实例变量是内置类型。构造函数在进入构造函数主体之前会做什么?什么也不做。(谢谢 C,你的遗产在这里非常受用。)age 实例变量将是恰好位于其所在内存中的随机位模式。允许访问该值,但其内容未知 - 其含义和用途未定义。这种情况可能会导致漫长,而(并不)有趣的调试。

\mySamllsection{解决}

编写构造函数的正确方法是确保,每个实例变量都明确初始化。为每个实例变量提供一个参数值,或在初始化列表中提供一个有意义的默认值。坚持这种方法将确保每次都初始化每个实例变量。

消除多余的类型实例变量初始化的解决方案是,使用参数值以初始化列表形式初始化实例变量。将构造函数主体中的代码,用于赋值给实例变量的情况作为少数例外。

\filename{清单8.22 坚持为每个实例变量提供参数的类}

\begin{cpp}
class Person {
private:
  std::string name;
  int age;
public:
  Person(const std::string& name, int age) :
  name(name), age(age) {} // 1
  friend std::ostream& operator<<(std::ostream&, const Person&);
};

std::ostream& operator<<(std::ostream& o, const Person& p) {
  o << p.name << " is " << p.age << " years old";
  return o;
}

int main() {
  Person joey("Joey", 27); // 2
  std::cout << joey << '\n';
  return 0;
}
\end{cpp}

{\footnotesize
注释1:每个实例变量都必须初始化

注释2:每个构造函数参数都必须有一个实参
}

如果需要验证(例如变量),请编写一个私有验证方法,该方法返回已验证的参数值或引发异常。在初始化列表中调用验证方法。以下清单中的代码已更新,以包含 age 实例变量的私有验证器,该验证器在构造函数的初始化列表中调用。

\filename{清单8.23 在初始化列表中使用私有验证器验证参数}

\begin{cpp}
class Person {
private:
  std::string name;
  int age;
  static int validateAge(int age) { // 1
    if (age < 0)
      throw std::invalid_argument("negative age");
    return age;
  }
public:
  Person(const std::string& name, int age) :
  name(name), age(validateAge(age)) {} // 2
  friend std::ostream& operator<<(std::ostream&, const Person&);
};

std::ostream& operator<<(std::ostream& o, const Person& p) {
  o << p.name << " is " << p.age << " years old";
  return o;
}

int main() {
  Person joey("Joey", 27); // 3
  std::cout << joey << '\n';
  return 0;
}
\end{cpp}

{\footnotesize
注释1:返回有效值或引发异常的私有验证器方法

注释2:验证器在初始化列表中调用;不需要多余的赋值

注释3:因验证而安全地创建实例
}

\mySamllsection{建议}

\begin{itemize}
\item
确保每个实例变量在每个构造函数中都已初始化,无论是通过参数值还是默认值。

\item
内置类型不会在构造函数主体之外隐式初始化。

\item
不要依赖开发者记住初始化实例变量;在构造函数中提供参数,以便编译器提醒他们提供每个所需的参数。

\item
调用参数验证代码以确保正确建立类不变量;将验证器设为私有,需要本地化实例变量的知识,并返回有效值或引发异常。
\end{itemize}
