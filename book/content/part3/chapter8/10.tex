这个错误集中在正确性上,以带有基类元素的数组(或容器)作为参数的函数似乎允许多态行为。可以将派生元素的数组传递给函数 —— 不会发出编译时错误。这提高了效率,但隐藏的危险会让粗心的用户绊倒。

对象容器是收集相关对象,并根据基类公共接口对其进行通用操作的自然方式——这是多态性的核心。但当将派生对象数组传递给函数时,使用可能会出现问题。数组是易于编写和使用的内置容器,其使用方式简单而直观。

\mySamllsection{问题}

接受指向基类对象的指针的函数,也可以传递派生类指针或引用。LSP 在这里成立,派生类可以在需要基类的地方替换。这是一种强大的技术,可以编写正确处理传递对象的通用代码。

很容易想到,如果指向对象的指针有效,那么指向对象数组的指针也应该有效。在 C++ 中,这并不意味着派生元素数组可以替换为基类元素数组,并且仍然有效。

C++ 代码使用指针算法访问数组中的元素。假设有一个名为 arr 的整数数组,其中包含多个元素;第三个元素由 arr[2] 索引。此表示法是语法糖,因为访问实际上是 *(arr+2)。指向第一个元素 arr 的const 指针添加了整数 2,以跳过两个元素并指向第三个元素,而要跳过的字节数是 2*sizeof(element\_type)。正是这种指针算法导致传递数组时出现问题。

考虑清单 8.19 中的代码,其中打印了派生类元素的数组。类 D 派生自基类 B,而 D 在需要打印其元素时使用 B 中定义的 getN 方法。B 将 n 实例变量默认为 0,而 D 将其 m 实例变量默认为 1。这些易于区分的值使输出分析变得清晰。

\filename{清单8.19 传递带有派生元素的数组}

\begin{cpp}
const int SIZE = 4;
class B {
private:
  int n;
public:
  B(int n=0) : n(n) {}
  ~B() { std::cout << "destroying B\n"; } // 1
int getN() const { return n; } // 2
};
class D : public B {
  int m;
public:
  D() : B(0), m(1) {}
  ~D() { std::cout << "destroying D\n"; }
};

void printArray(const B a[]) {
  for (int i = 0; i < SIZE; ++i)
    std::cout << a[i].getN() << '\n'; // 3
}

void deleteArray(const B a[]) { // 4
  delete [] a;
}

int main() {
  B* bs = new B[SIZE];
  printArray(bs);
  deleteArray(bs);

  D* ds = new D[SIZE];
  printArray(ds);
  deleteArray(ds);
  return 0;
}
\end{cpp}

{\footnotesize
注释1:此处故意不写成 virtual,是为了更好地演示整个问题。通常情况下,请不要这样做!

注释2:数组元素的通用访问函数

注释3:输出值时使用通用访问函数

注释4:使用通用删除函数
}

执行此代码时,B 元素数组按预期工作;n 部分的输出为 0,销毁为 B 元素。但对于 D 元素数组,输出可能会更好。getN 函数交替输出D 元素的 n 和 m 部分,由于 n 和 m 部分都是整数(大小相同),因此输出显示值的索引是增加 B 元素的大小,而不是必要的 D 元素的大小。此外,元素销毁时,只会调用 B 析构函数,所以会返回到操作系统的内存块可能与实际数组大小不一致,这是内存泄漏。虽然预计不同的系统可能会产生不同的结果,但无论如何这都是未定义的行为。

以下输出表明使用派生元素无法按预期工作。首先,析构函数是基类版本独有的。其次,getN 的结果在 n 和 m 实例变量之间有所不同:

\begin{shell}
0
0
0
0
destroying B
destroying B
destroying B
destroying B
0
1
0
1
destroying B
destroying B
destroying B
destroying B
\end{shell}

\mySamllsection{分析}

printArray 函数将索引数组元素并调用 getN 函数。第一次调用该函数会传递一个 B 元素数组。函数中的索引从第一个元素开始,到下一个元素添加一个 B 元素的大小。此过程持续进行,直到每个元素都已迭代完毕。

第二种情况下,数组包含 D 个元素。由于 D 派生自 B,因此传递数组成功,编译器没有报错 — LSP 在这里正常工作。printArray 函数接收指向数组的 const 指针作为其初始点;第一个元素从该点开始。在索引数组并调用 getN 函数时,访问从此元素开始,并使用 B 部分从中提取值;到目前为止一切顺利。它找到 1 的值并输出它。

仔细注意第二个打印输出;它是 m 变量的值。B 元素使用一个整数的空间(通常为四个字节),但 D 元素使用两倍的空间。当循环更新索引值时,会将指针增加一个 B 元素的大小(一个整数),但实际值是第一个 m 变量,恰好是指针的 sizeof(B)(或*(ptr+sizeof(B))。D 元素的大小是 B 元素的两倍,索引以交替模式引用它们。因此,索引 1 的指针指向第一个元素的 D 部分,而不是第二个元素的 B 部分。如果这听起来不对,因为这是错的。

指针算法对于 B 元素的数组正确运行,但对于派生类(例如 D)则不正确。没有编译器警告,运行时也没有发现段错误(访问无法访问的内存)。此错误与访问越界内存相反,其没有足够的内存可进行访问!

注意:对操作和结果的解释基于同一环境中的非优化编译和执行,优化或编译器更改可能会导致不同的行为。更重要的是,这种行为未定义。

现在,考虑删除数组。当使用包含 B 个元素的数组调用 deleteArray 时,此代码可以正常工作。不明显的是,delete 操作符还将索引数组,并在每个元素上调用 B 版本的删除操作 — 输出为“destroying B ”。当数组包含 B 个元素时,此方法可以正常工作。

当使用 D 个元素调用此函数时,循环将调用元素的 B 部分的析构函数,并根据 B 元素的大小进行索引。第二个 delete 引用第一个元素的 D 部分。这种情况下发生的情况尚待推测,只有 B 大小的元素销毁。指针算术问题最终导致了这个问题,所以虚析构函数无法解决问题。

现在,考虑一下派生元素比整数更复杂的情况。如果 D 元素包含std::string 引用或指向其他数据类型的指针,则不清楚会清理什么。需要澄清这一事实非常令人担忧;编程语言都不应该提供可能有效或可能无效的操作,这取决于无法控制的事物。

在最初的问题中,删除了一个包含 D 元素的数组(或者代码尝试删除),但只有前四个 B 大小的元素删除。如果数组位于堆栈中,最终,这个混乱局面将会清除,但如果内存基于堆,那么释放器将如何处理它,以及回收多少数据仍不清楚。

如果使用 valgrind(用于检测多种内存问题(尤其是泄漏)的出色工具)运行此代码,它会运行干净,因为分配和删除的数量正确配对。地址清理器可以发现一些问题;请参阅 Matt Godbolt 的 Compiler Explorer (\url{https://compiler-explorer.com/z/6d73hscoE}) 以获取示例。这个错误非常隐蔽,即使是复杂的工具似乎也无法完全检测到问题。开发者有责任永远不要使用基类数组,作为节省代码的通用技术。

\mySamllsection{解决}

正确的方法是,拒绝将指向基类和派生类对象数组的指针传递给函数。让函数只处理实际元素类型的数组,复制这些函数是一种简单的补救措施。使用添加函数模板的通用编程技术是确保这种复制的理想方法,这种技术既可读又有效。另一种可能性是将指针数组传递给基类。这种方法的缺点是,代码可能必须处理指向指针的指针,从而使调用和处理语法变得有点混乱——如果愿意,那就去做吧!

现代 C++ 提供了 std::array 类,也能解决这个问题。如果可以的话,请使用该功能来解决这个问题。

这时,代码重用不是一个好主意,如果过于严格地遵循这个概念,将会导致困难。以下代码通过根据模板建议,制作明确的非多态输出和删除函数,来解决前面提到的问题。

\filename{清单8.20 没有多态数组传递的函数}

\begin{cpp}
const int SIZE = 4;

class B {
private:
  int n;
public:
  B(int n=0) : n(n) {}
  ~B() { std::cout << "destroying B\n"; }
  int getN() const { return n; }
};

class D : public B {
  int m;
public:
  D() : B(1), m(2) {}
  ~D() { std::cout << "destroying D\n"; }
};

template <typename T> // 1
void printArray(T a[]) {
  for (int i = 0; i < SIZE; ++i)
    std::cout << a[i].getN() << '\n';
}

template <typename T> // 1
void deleteArray(T a[]) {
  delete [] a;
}

int main() {
  B* bs = new B[SIZE];
  printArray(bs); // 2
  deleteArray(bs);

  D* ds = new D[SIZE];
  printArray(ds); // 2
  deleteArray(ds);

  return 0;
}
\end{cpp}

{\footnotesize
注释1:函数模板处理所需的代码重复

注释2:不同元素类型的通用处理
}

以下代码显示修复后的代码的输出:

\begin{shell}
0
0
0
0
destroying B
destroying B
destroying B
destroying B
1
1
1
1
destroying D
destroying B
destroying D
destroying B
destroying D
destroying B
destroying D
destroying B
\end{shell}

元素显示和删除均正确,并且没有内存丢失。非常重要的是,B 和 D 元素的输出正确,并且 D 元素的每个部分均销毁。元素的销毁涉及正确的类型,不会留下剩余内存或奇怪的操作系统分配和释放不匹配。

\mySamllsection{建议}

\begin{itemize}
\item
如果必须将基类和派生类数组作为参数传递给函数,则编写单独的函数来处理;编写类似的数组可避免出现奇怪的行为。考虑使用函数模板让编译器编写重复的代码。

\item
没有运行时错误并不表示编程方法是正确的,也不表示没有潜在的内存问题。

\item
要知道,即使是复杂的工具也无法检测到所有问题;valgrind 是一个很棒的工具,但它并不是为检测这种奇怪的极端情况而设计的。

\item
正确编码是开发者的责任,并了解使用快捷方式可能引起的问题。
\end{itemize}












