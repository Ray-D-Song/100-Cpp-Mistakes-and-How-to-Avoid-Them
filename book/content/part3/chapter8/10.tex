这个错误集中在正确性上。以带有基类元素的数组(或容器)作为参数的函数似乎允许多态行为。可以将派生元素的数组传递给函数- 不会发出编译时错误。这提高了效率;然而,隐藏的危险会让粗心的用户绊倒。

对象容器是收集相关对象并根据基类公共接口对其进行通用操作的自然方式——这是多态性的核心。但是,当将派生对象数组传递给函数时,它们的使用可能会出现问题。数组是易于编写和使用的内置容器 ,因此它们的使用简单而直观。

\mySamllsection{问题}

接受指向基类对象的指针的函数也可以传递派生类指针或引用。LSP 在这里成立,也就是说,派生类可以在需要基类的地方被替换。这是一种强大的技术,可以编写正确处理传递的任何对象的通用代码。

很容易想到,如果指向对象的指针有效,那么指向对象数组的指针也应该有效。然而,在 C++ 中,这并不意味着派生元素数组可以替换为基类元素数组并且仍然有效。

C++ 代码使用指针算法访问数组中的元素。假设有一个名为 arr 的整数数组,其中包含多个元素;第三个元素由 arr[2] 索引。此表示法是语法糖,因为访问实际上是 *(arr+2)。指向第一个元素 arr 的const 指针添加了整数 2,以跳过两个元素并指向第三个元素。要跳过的字节数是 2*sizeof(element\_type)。正是这种指针算法导致传递数组时出现问题。

考虑清单 8.19 中的代码,其中打印了派生类元素的数组。类 D 派生自基类 B,而 D 在需要打印其元素时使用 B 中定义的 getN 方法。B 将 n 实例变量默认为 0,而 D 将其 m 实例变量默认为 1。这些易于区分的值使输出分析变得清晰。

\filename{清单8.19 传递带有派生元素的数组}

\begin{cpp}
const int SIZE = 4;
class B {
private:
  int n;
public:
  B(int n=0) : n(n) {}
  ~B() { std::cout << "destroying B\n"; } // 1
int getN() const { return n; } // 2
};
class D : public B {
  int m;
public:
  D() : B(0), m(1) {}
  ~D() { std::cout << "destroying D\n"; }
};

void printArray(const B a[]) {
  for (int i = 0; i < SIZE; ++i)
    std::cout << a[i].getN() << '\n'; // 3
}

void deleteArray(const B a[]) { // 4
  delete [] a;
}

int main() {
  B* bs = new B[SIZE];
  printArray(bs);
  deleteArray(bs);

  D* ds = new D[SIZE];
  printArray(ds);
  deleteArray(ds);
  return 0;
}
\end{cpp}

{\footnotesize
注释1:此处故意不写成 virtual,是为了更好地演示整个问题。通常不要这样做!

注释2:数组元素的通用访问函数

注释3:输出值时使用通用访问函数

注释4:使用通用删除函数
}

执行此代码时,B 元素数组按预期工作;n 部分的输出为 0,销毁为 B 元素。但是,对于 D 元素数组,输出可能会更好。getN 函数交替输出D 元素的 n 和 m 部分。由于 n 和 m 部分都是整数(大小相同),因此输出显示值的索引是增加 B 元素的大小,而不是必要的 D 元素的大小。此外,在元素被销毁时,只会调用 B 析构函数,这意味着返回到操作系统的内存块可能与实际数组大小不一致。这将是内存泄漏。虽然预计不同的系统可能会产生不同的结果,但无论如何这都是未定义的行为。以下输出表明使用派生元素无法按预期工作。首先,析构函数是基类版本独有的。其次,getN 的结果在 n 和 m 实例变量之间有所不同:

\begin{shell}
0
0
0
0
destroying B
destroying B
destroying B
destroying B
0
1
0
1
destroying B
destroying B
destroying B
destroying B
\end{shell}

\mySamllsection{分析}

printArray 函数将索引数组元素并调用 getN 函数。第一次调用该函数会传递一个 B 元素数组。函数中的索引从第一个元素开始,到下一个元素添加一个 B 元素的大小。此过程持续进行,直到每个元素都已迭代完毕。

在第二种情况下,数组包含 D 个元素。由于 D 派生自 B,因此传递数组成功,编译器没有任何抱怨 — LSP 在这里正常工作。printArray 函数接收指向数组的 const 指针作为其初始点;第一个元素从该点开始。在索引数组并调用 getN 函数时,访问从此元素开始,并使用 B 部分从中提取值;到目前为止一切顺利。它找到 1 的值并输出它。

仔细注意第二个打印输出;它是 m 变量的值。在这种情况下,B 元素使用一个整数的空间(通常为四个字节),但 D 元素使用两倍的空间。当循环更新索引值时,它会将指针增加一个 B 元素的大小(一个整数),但实际值是第一个 m 变量,它恰好是指针的 sizeof(B)(或*(ptr+sizeof(B))。D 元素的大小是 B 元素的两倍,因此索引以交替模式引用它们。因此,索引 1 的指针指向第一个元素的 D 部分,而不是第二个元素的 B 部分。如果这听起来不对,那是因为它是错误的。

指针算法对于 B 元素的数组正确运行,但对于派生类(例如 D)则不 正确。没有编译器警告,运行时也没有发现分段错误(访问无法访问的内存)。此错误与访问越界内存相反;它没有访问足够的内存!

注意:对操作和结果的解释基于同一环境中的非优化编译和执行。优化或编译器更改可能会导致不同的行为。

更重要的是,这种行为是未定义的。因此,对于确切结果不应下定论。

现在,考虑删除数组。当使用包含 B 个元素的数组调用 deleteArray 时,此代码可以正常工作。不明显的是,delete 运算符还将索引数组并在每个元素上调用 B 版本的删除操作 — 因此,输出为“destroying B ”。当数组包含 B 个元素时,此方法可以正常工作。

当使用 D 个元素调用此函数时,循环将调用元素的 B 部分的析构函数并根据 B 元素的大小进行索引。第二个 delete 引用第一个元素的 D 部分。在这种情况下发生的情况尚待推测。显而易见的是,只有 B 大小的元素被销毁。在这种情况下,虚拟析构函数无法解决问题,因为指针算术问题最终导致了这个问题。

现在,考虑一下派生元素比整数更复杂的情况。如果 D 元素包含std::string 引用或指向其他数据类型的指针,则不清楚会清理什么(如果有的话)。需要澄清这一事实非常令人担忧;任何编程语言都不应该提供可能有效或可能无效的操作,这取决于您无法控制的事物。

在我们最初的问题中,删除了一个包含 D 元素的数组(或者代码尝试删除它们),但只有前四个 B 大小的元素被删除。如果数组位于堆栈中,最终,这个混乱局面将会清除,但如果内存是基于堆的,那么释放器将如何处理它以及回收多少数据仍不清楚。

如果使用 valgrind(一种用于检测多种内存问题(尤其是泄漏)的出色工具)运行此代码,它会运行干净,因为分配和删除的数量正确配对。地址清理器可以发现一些问题;请参阅 Matt Godbolt 的 Compiler Explorer 网站 (\url{https://compiler-explorer.com/z/6d73hscoE}) 以获取示例。这个错误非常隐蔽,即使是复杂的工具似乎也无法完全检测到问题。程序员有责任永远不要使用基类数组作为节省代码和按键的通用技术。

\mySamllsection{解决}

正确的方法是拒绝将指向基类和派生类对象数组的指针传递给函数。让函数只处理实际元素类型的数组。复制这些函数是一种简单的补救措施。使用添加函数模板的通用编程技术是确保这种复制的理 想方法。这种技术既可读又有效。另一种可能性是将指针数组传递给基类。这种方法的缺点是代码可能必须处理指向指针的指针,从而使调用和处理语法变得有点混乱——如果您愿意,那就去做吧!

现代 C++ 提供了 std::array 类,它也能解决这个问题。如果可以的话,请使用该功能来解决这个问题。

在这种情况下,代码重用不是一个好主意,如果过于严格地遵循这个概念,将会导致困难。以下代码通过根据模板建议制作明确的非多态打印和删除函数来解决前面提到的问题。

\filename{清单8.20 没有多态数组传递的函数}

\begin{cpp}
const int SIZE = 4;

class B {
private:
  int n;
public:
  B(int n=0) : n(n) {}
  ~B() { std::cout << "destroying B\n"; }
  int getN() const { return n; }
};

class D : public B {
  int m;
public:
  D() : B(1), m(2) {}
  ~D() { std::cout << "destroying D\n"; }
};

template <typename T> // 1
void printArray(T a[]) {
  for (int i = 0; i < SIZE; ++i)
    std::cout << a[i].getN() << '\n';
}

template <typename T> // 1
void deleteArray(T a[]) {
  delete [] a;
}

int main() {
  B* bs = new B[SIZE];
  printArray(bs); // 2
  deleteArray(bs);

  D* ds = new D[SIZE];
  printArray(ds); // 2
  deleteArray(ds);

  return 0;
}
\end{cpp}

{\footnotesize
注释1:函数模板处理所需的代码重复

注释2:不同元素类型的通用处理
}

以下代码片段显示修复后的代码的输出:

\begin{shell}
0
0
0
0
destroying B
destroying B
destroying B
destroying B
1
1
1
1
destroying D
destroying B
destroying D
destroying B
destroying D
destroying B
destroying D
destroying B
\end{shell}

元素显示和删除均正确,并且没有内存丢失。非常重要的是,B 和 D 元素的输出正确,并且 D 元素的每个部分均被销毁。元素的销毁涉及正确的类型,因此不会留下剩余内存或奇怪的操作系统分配和释放不 匹配。

\mySamllsection{建议}

\begin{itemize}
\item
如果必须将基类和派生类数组作为参数传递给函数,则编写单独的函数来处理它们;编写类似的数组将防止出现奇怪的行为。考虑使用函数模板让编译器编写重复的代码。

\item
没有运行时错误并不表示编程方法是正确的,也不表示没有潜在的内存问题。

\item
要知道,即使是复杂的工具也无法检测到所有问题;valgrind 是一个很棒的工具,但它并不是为检测这种奇怪的极端情况而设计的。

\item
正确编码是您的责任,并了解使用快捷方式可能引起的问题。
\end{itemize}












