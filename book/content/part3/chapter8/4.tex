这个错误主要集中在正确性上。基类建立并维护了类不变性,但容易受到派生类所做的更改的影响。派生类有义务遵守基类不变性。可读性无疑得到了改善,有效性也可以得到改善。

作为最佳实践,所有实例变量都应为私有变量。这种方法可确保,没有外部代码可以在没有类严格控制的情况下,访问或修改它们。引入继承后,派生类会发现需要的基类信息无法访问。与用户代码一样,派生类必须使用基类访问器和修改器,但这种限制可能感觉过于严格。C++ 为基类提供了 protected 关键字,以允许派生类直接访问这些实例变量 - 外部代码仍然被禁止访问或修改这些变量。虽然这种放宽看起来不错,但也引入了违反类不变量的可能性。

\mySamllsection{问题}

基类通过严格控制每个数据成员的值,来维护其状态。构造函数和变量器应设计为初始化类不变量,将值保持在其适当的边界,并避免越界突变的企图。如果基类已将这些变量设为受保护,则派生类可以直接访问这些变量。

派生类应保持其类不变量,并尽力遵守其基类不变量。不幸的是,编译器无法保护不变量;开发者必须付出努力。此外,为了使派生类正确维护基类不变量,开发人员必须掌握如何正确处理基类限制的外部知识。这些知识已存在于基类中,不应在派生类中重复。

\filename{清单8.6 受保护实例变量的漏洞}

\begin{cpp}
class Person {
protected:
  std::string name;
  int age; // 1
public:
  Person(const std::string& name, int age) : name(name) {
    if (age < 0) // 2
      throw std::invalid_argument("negative age");
    this->age = age;
  }
  std::string getName() const { return name; }
  int getAge() const { return age; }
};
class Student : public Person {
private:
  double gpa;
public:
  Student(const std::string& name, int age, double gpa) : Person
  (name, age), gpa(gpa) {}
  double getGpa() const { return gpa; }
  void setAge(int age) { this->age = age; } // 3
};

int main() {
  Student jane("Jane", 26, 3.85);
  jane.setAge(-26);
  std::cout << "Jane is " << jane.getAge() << " years old\n";
  return 0;
}
\end{cpp}

{\footnotesize
注释1:易受攻击的实例变量

注释2:经过适当验证,并保持类不变量

注释3:危及基类不变量的权宜之计
}

\mySamllsection{分析}

对于不理解或未充分维护基类不变量的派生类,一个或多个受保护的基类变量可能会强制进入超出范围的状态,且不会发出警告,使拒绝无效值变得不可能。Student 的 setAge 方法不限制参数值,而是直接设置 age 实例变量;发生这种情况是因为该变量是受保护的,而非私有。

Student 类的设计者认为使用 Person 基类的受保护变量是有利的。Student 开发者决定添加一个 setAge 方法,其知道一些较新的学校规则会根据年龄而有所不同,并且学生可能会在学年期间过生日。

Person 类正确验证了年龄输入值并拒绝了无效值。派生类需要充分了解添加 setAge 方法的含义,但未能做到这一点。由于基类构造函数验证了 age,因此一切最初看起来都很好。但基类不会改变age 实例变量,需要一种快捷方式来直接改变变量。为了在Student 中正确实现 setAge 方法,派生类必须复制有关age基类变量。这种知识重复违反了 DRY 原则,并且给变量的封装带来了压力。

\mySamllsection{解决}

Person 基类负责维护其类的不变性。部分责任是通过拒绝其他类,在没有某些验证逻辑保护的情况下,直接访问其实例变量来处理的,如下面的清单所示。

\filename{清单8.7 通过将实例变量设置为私有来增强安全性}

\begin{cpp}
class Person {
private:
  std::string name;
  int age; // 1
public:
  Person(std::string name, int age) : name(name) { setAge(age); }
  std::string getName() const { return name; }
  int getAge() const { return age; }
  void setAge(int age) {
    if (age < 0) // 2
      throw std::invalid_argument("negative age");
    this->age = age;
  }
};

class Student : public Person {
private:
  double gpa;
public:
  Student(std::string name, int age, double gpa) :
    Person(name, age), gpa(gpa) {} // 3
  double getGpa() const { return gpa; }
};

int main() {
  Student jane("Jane", 26, 3.85);
  jane.setAge(-26);
  std::cout << "Jane is " << jane.getAge() << " years old\n";
  return 0;
}
\end{cpp}

{\footnotesize
注释1:不易受到派生类突变的影响

注释2:经过适当验证,并保持类不变

注释3:必须使用经过适当验证的基类方法
}

基类正确地验证了年龄值,该值从构造函数、派生类和用户代码中调用。派生类不需要知道有关有效年龄的信息,它将保证年龄正确这一责任委托给基类。

这次讨论并不意味着永远不使用受保护的变量;有些情况下它们已经存在,无法改变这一事实。但可以(并且必须)尊重基类不变性并使用其变量。如果基类无法更改且没有验证变量,请在派生类中编写一个并注释原因。记住里氏替换原则 (LSP);如果基类变量未抛出而派生类变量抛出,则不满足替换要求。

如果基类不打算向用户代码公开其某些实例变量,则可以为派生类提供受保护的方法,以供其进行变异。这些方法确保基类能够控制其状态的任何更改,同时为派生类提供一个接口,以用户代码无法实现的方式改变状态。

\mySamllsection{建议}

\begin{itemize}
\item
消除尽可能多的受保护变量。

\item
始终让变量器验证其参数。

\item
了解派生类使用访问器和修改器的轻微尴尬,完全可以通过基类保持其类不变性来抵消。

\item
考虑使用受保护的访问器和变量设置器,来提供用户代码无权使用的访问权限。

\item
不要将变量设为虚(从基类继承);它们只能在所属类中实现。
\end{itemize}













