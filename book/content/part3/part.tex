软件开发中,处理历史遗留代码是一项挑战,特别是对于 C++ 编程而言。C++ 的经典或前现代的特点是在现代语言功能建立之前开发的实践,通常缺乏当代功能提供的安全性、效率和简单性。本节深入探讨这些经典的编码问题,提供在遗留系统范围内提高代码质量和稳健性的见解和策略。

一个关键的关注领域是类不变量的设计和维护,这是创建稳健且稳定的应用程序的基本原则。建立强大的类不变量对于确保应用程序的稳定性和可预测性至关重要,但随着时间的推移,尤其是随着代码库的发展,维护这些不变量会带来挑战。本节探讨过去的陷阱,并提供通过精心设计来制定和维持类完整性的指导,帮助开发人员避免设计质量的逐渐下降和新代码的改进。

此外,本部分还讨论了有关类操作、资源管理和函数使用的问题。精心实现类操作(例如:构造函数、析构函数和赋值操作符)对于减少错误和优化性能至关重要。正确处理异常和系统资源管理,对于避免资源泄漏和确保应用程序可靠性至关重要。有必要改进函数和参数的设计,以避免前现代 C++ 实践中的低效。最后,对经典 C++ 中的一般编码实践的全面检查为开发人员提供了路线图,以提高其遗留代码的质量和可持续性,确保其保持扩展性以应对未来的需求。
