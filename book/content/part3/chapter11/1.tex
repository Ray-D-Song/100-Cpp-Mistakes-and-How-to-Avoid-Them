函数由名称、参数列表、返回值和函数体组成。编译器坚持所有四个部分,但提供了一种使参数列表和返回值看起来是可选的方法。有些书称构造函数(和隐含的析构函数)为函数;从技术上讲,它不是函数,但 C++ 标准称它们为“特殊成员函数”。构造函数和析构函数没有返回值,不符合函数性质。参数列表看起来是可选的,因为可以为空,但仍必须用空括号指定。返回值看起来是可选的,类型可以为 void。但这仍然是一种类型,是没有值的类型,或者其值为空。

命名是定义函数最难的部分。虽然可以使用旧名称,但有一个单一、简洁、易于沟通的名称通常很困难。必须以直观的方式解释函数的用途,使其使用起来简单明了。过长或复杂的函数几乎不可能命名。

理想情况下,函数应该是纯函数,不会访问其范围之外的变量或数据,完全根据其参数列表传递的数据执行其行为。有些函数应该在其范围之外影响系统,operator<{}< 和operator>{}> 方法就是很好的例子。由于这些函数必须存在,目标是将它们分为两类:纯函数和产生副作用的函数。

产生副作用的函数不应进行任何计算,逻辑最少,并且影响其作用域之外的实体。纯函数应执行计算,使用必要的逻辑,并通过返回值返回其结果。某些函数需要返回多个值,而使用其返回值无法实现这一点。返回结构体(或类)的实例或使用输出参数是选项。输出参数更难以快速理解,并影响可读性。返回实例更复杂,但可以更好地将输入与输出隔离。

值、指针或引用参数可以将数据输入传递给函数。本书使用在调用站点引用值的约定 argument,参数列表中的变量称为 parameter。(有些作者和老师也有使用 parameter 和 formal parameter 的配对)

函数的局部变量通过显式编程操作初始化,通常使用赋值操作符。参数是局部变量,但开发者不会显式地初始化它们。编译器生成代码来复制参数的值,该值是参数的初始化值。否则,参数在所有方面都是常规的局部变量。局部变量的一个重要结果是,当函数退出时(通常是通过抛出异常)局部变量会销毁,但这并不意味着销毁的变量一定无法访问。有些系统允许在销毁它们之后访问(通过指针或引用),从而导致难以诊断的错误;优秀的系统会在这种情况下会崩溃。

以下错误尝试解决函数设计和使用中的一些常见问题。虽然存在许多此类错误,但有些错误更常发生。
