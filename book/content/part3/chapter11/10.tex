这种错误会影响可读性和有效性,并可能对正确性产生负面影响。函数是编写可重用、可读代码的核心,但如果编程不当,它们可能会成为负担。

\mySamllsection{问题}

与上一个错误相关,过于负责的函数承载了太多功能,让人无法清楚理解。我们这位勇敢的程序员又走了一条捷径,结果如下代码。对数据进行清理的需求发展为两个要求:过滤低于零的数据和过滤高于零的数据。由于这些行为相似,因此编写了一个函数来执行 这两项操作。调用代码需要传递一个值向量和一个布尔标志,以确定哪些值被消除。

\filename{清单11.17 在一个函数中过滤0以上和0以下的数据}

\begin{cpp}
std::vector<double> cleanse(const std::vector<double>& values,
    bool less_than) {
  std::vector<double> new_values;
  for (int i = 0; i < values.size(); ++i)
    if (less_than) { ❶
      if (values[i] < 0)
        new_values.push_back(values[i]);
    } else
        if (values[i] > 0)
          new_values.push_back(values[i]);
  return new_values;
}

int main() {
  std::vector<double> values;
  values.push_back(3.14159);
  values.push_back(-1.23456);
  values.push_back(2.71828);
  values.push_back(-3.14159);

  std::vector<double> above = cleanse(values, false); // 2
  for (int i = 0; i < above.size(); ++i)
    std::cout << above[i] << ' ';
  std::cout << '\n';

  std::vector<double> below = cleanse(values, true); // 2
  for (int i = 0; i < below.size(); ++i)
    std::cout << below[i] << ' ';
  std::cout << '\n';
  return 0;
}
\end{cpp}

{\footnotesize
注释1:基于布尔标志的过滤器

注释2:指定用于过滤大于或小于零的布尔标志
}

\mySamllsection{分析}

cleanse 函数对向量进行迭代;确定是否过滤高于或低于的值;检查值的符号;如果匹配所需范围,则将其复制到结果向量中。虽然操作很简单,但很明显该函数正在完成两种行为,由标志切换。开发人员可能记得这种行为的微妙之处,但新开发人员必须花时间弄清楚像cleanse(values, false) 这样的调用意味着什么。这位开发人员必须找到该函数的源代码;阅读它;并且在没有有意义的注释(如本例 所示)的情况下,跟踪代码,直到清楚发生了什么。

这种编程风格需要付出额外的努力才能正确(有效)和理解(可读性)。这是一种糟糕的设计,希望节省精力和(天哪!)重复。现有的代码库中充满了例子。

\mySamllsection{解决}

类似函数中的代码重复是不可避免的,而消除重复的努力可能会导致更严重的问题。目标是尽量减少或消除多个函数或方法之间的知识传播。在这种情况下,过滤器下方和过滤器上方的代码之间没有共享知识,尽管它们之间存在相当多的重复。

当过滤之间的控制流(控制结构)相同时,通常会引入问题。这种情况导致开发人员将不同的部分(过滤功能)组合成一个通用结构,通常包含重复的部分。(哎哟!我们正在努力消除这种重复。)引入由布尔值选择的控制流路径肯定意味着要努力将不同函数中的控制流重复合并为一个函数。这不仅浪费开发时间并使测试更加困难,而且也无法解决编写它的目的。

几个简短的函数是理想的,不意味着丑陋的重复。简短的函数更易于命名,从而导致调用代码中的函数直观。这些函数易于测试和验证,并且易于阅读和编写。哦,它们往往会重复控制流逻辑。正如 Meatloa f 所说,“三分之二还不错。”

以下代码将过滤逻辑拆分为两个函数,并适当地命名它们。我们抵制了消除结构化代码的冲动,并将每个函数的目的(其逻辑)分离为一个单元。

\filename{清单11.18 使用单独的函数进行过滤}

\begin{cpp}
std::vector<double> filter_above(const
    std::vector<double>& values) { // 1
  std::vector<double> new_values;
  for (int i = 0; i < values.size(); ++i)
    if (values[i] > 0)
      new_values.push_back(values[i]);
  return new_values;
}

std::vector<double> filter_below(const
    std::vector<double>& values) { // 2
  std::vector<double> new_values;
  for (int i = 0; i < values.size(); ++i)
    if (values[i] < 0)
      new_values.push_back(values[i]);
  return new_values;
}

int main() {
  std::vector<double> values;
  values.push_back(3.14159);
  values.push_back(-1.23456);
  values.push_back(2.71828);
  values.push_back(-3.14159);

  std::vector<double> above = filter_above(values); // 3
  for (int i = 0; i < above.size(); ++i)
    std::cout << above[i] << ' ';
  std::cout << '\n';

  std::vector<double> below = filter_below(values); // 3
  for (int i = 0; i < below.size(); ++i)
    std::cout << below[i] << ' ';
  std::cout << '\n';
  return 0;
}
\end{cpp}

{\footnotesize
注释1:具有特定行为的单一用途函数

注释2:具有不同特定行为的另一个单一用途函数

注释3:使用简单,没有可疑参数
}

作为一项挑战,弄清楚如何使用函数模板来实现比较功能。编写一个接受向量和函数模板参数的单个过滤器函数。考虑在此示例中使用 std::greater<double> 和 std::less<double>。有关提示,请参阅以下 C++ 参考网页:\url{https://mng.bz/gaan}。

\mySamllsection{建议}

\begin{itemize}
\item
让每个函数在其职责上独立,并明确说明一个目的。

\item
不要害怕在函数之间复制控制结构 — 如果在一个函数中有效,那么在另一个函数中也能有效。

\item
不要过分担心跨函数重复 code;将 knowledge 隔离到一个函数或一组协调的函数(可能是个类)。
\end{itemize}
