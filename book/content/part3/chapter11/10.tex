这种错误会影响可读性和有效性,并可能对正确性产生负面影响。函数是编写可重用、可读代码的核心,但如果编程不当,则可能会成为负担。

\mySamllsection{问题}

与上一个错误相关,过于负责的函数承载了太多功能,让人无法清楚理解。开发者又走了一条捷径,结果如下代码。对数据进行清理的需求发展为两个要求:过滤低于0的数据和过滤高于0的数据。由于这些行为相似,因此编写了一个函数来执行这两项操作。调用代码需要传递一个值vector和一个布尔标志,以确定哪些值可消除。

\filename{清单11.17 在一个函数中过滤0以上和0以下的数据}

\begin{cpp}
std::vector<double> cleanse(const std::vector<double>& values,
    bool less_than) {
  std::vector<double> new_values;
  for (int i = 0; i < values.size(); ++i)
    if (less_than) { // 1
      if (values[i] < 0)
        new_values.push_back(values[i]);
    } else
        if (values[i] > 0)
          new_values.push_back(values[i]);
  return new_values;
}

int main() {
  std::vector<double> values;
  values.push_back(3.14159);
  values.push_back(-1.23456);
  values.push_back(2.71828);
  values.push_back(-3.14159);

  std::vector<double> above = cleanse(values, false); // 2
  for (int i = 0; i < above.size(); ++i)
    std::cout << above[i] << ' ';
  std::cout << '\n';

  std::vector<double> below = cleanse(values, true); // 2
  for (int i = 0; i < below.size(); ++i)
    std::cout << below[i] << ' ';
  std::cout << '\n';
  return 0;
}
\end{cpp}

{\footnotesize
注释1:基于布尔标志的过滤器

注释2:指定用于过滤大于或小于0的布尔标志
}

\mySamllsection{分析}

cleanse 函数对vector进行迭代;确定是否过滤高于或低于的值;检查值的符号;如果匹配所需范围,则将其复制到结果vector中。虽然操作很简单,很明显该函数正在完成两种行为,由标志切换。开发人员可能记得这种行为的微妙之处,但新开发人员必须花时间弄清楚像cleanse(values, false) 这样的调用意味着什么。这位开发人员必须找到该函数的源代码,并进行阅读;并且在没有有意义的注释(如本例 所示)的情况下,跟踪代码,直到清楚发生了什么。

这种编程风格需要付出很多努力才能正确(有效)和理解(可读性)。这是一种糟糕的设计,希望节省精力和避免重复。现有的代码库中充满了例子。

\mySamllsection{解决}

类似函数中的代码重复是不可避免的,而消除重复的努力可能会导致更严重的问题。目标是尽量减少或消除多个函数或方法之间的知识传播。过滤器下方和过滤器上方的代码之间没有共享知识,它们之间存在相当多的重复。

当过滤之间的控制流(控制结构)相同时,通常会引入问题。这种情况导致开发人员将不同的部分(过滤功能)组合成一个通用结构,通常包含重复的部分(我们正在努力消除这种重复)。引入由布尔值选择的控制流路径,则要努力将不同函数中的控制流重复合并为一个函数。这不仅浪费开发时间并使测试更加困难,而且也无法解决其编写目标。

几个简短的函数是理想的,但并非丑陋的重复。简短的函数更易于命名,从而导致调用代码中的函数直观。这些函数易于测试和验证,并且易于阅读和编写,往往会重复控制流逻辑。正如 Meat loaf 所说,“Two Out Of Three Ain't Bad(三分之二也不错)”\footnote{译者注:“Two Out Of Three Ain't Bad”是Meat Loaf演唱的歌曲,由Jim Steinman作词作曲,收录于《The Collection》专辑中。}。

以下代码将过滤逻辑拆分为两个函数,并适当地命名它们。消除结构化代码的冲动,并将每个函数的目的(其逻辑)分离为一个单元。

\filename{清单11.18 使用单独的函数进行过滤}

\begin{cpp}
std::vector<double> filter_above(const
    std::vector<double>& values) { // 1
  std::vector<double> new_values;
  for (int i = 0; i < values.size(); ++i)
    if (values[i] > 0)
      new_values.push_back(values[i]);
  return new_values;
}

std::vector<double> filter_below(const
    std::vector<double>& values) { // 2
  std::vector<double> new_values;
  for (int i = 0; i < values.size(); ++i)
    if (values[i] < 0)
      new_values.push_back(values[i]);
  return new_values;
}

int main() {
  std::vector<double> values;
  values.push_back(3.14159);
  values.push_back(-1.23456);
  values.push_back(2.71828);
  values.push_back(-3.14159);

  std::vector<double> above = filter_above(values); // 3
  for (int i = 0; i < above.size(); ++i)
    std::cout << above[i] << ' ';
  std::cout << '\n';

  std::vector<double> below = filter_below(values); // 3
  for (int i = 0; i < below.size(); ++i)
    std::cout << below[i] << ' ';
  std::cout << '\n';
  return 0;
}
\end{cpp}

{\footnotesize
注释1:具有特定行为的单一用途函数

注释2:具有不同特定行为的另一个单一用途函数

注释3:使用简单,没有可疑参数
}

作为一项挑战,弄清楚如何使用函数模板来实现比较功能。编写一个接受vector和函数模板参数的单个过滤器函数。可以在此示例中使用 std::greater<double> 和 std::less<double>。有关提示,请参阅以下 C++ 参考网页:\url{https://mng.bz/gaan}。

\mySamllsection{建议}

\begin{itemize}
\item
让每个函数在其职责上独立,并明确说明明确的目的。

\item
不要害怕在函数之间复制控制结构 — 如果在一个函数中有效,那么在另一个函数中也能有效。

\item
不要过分担心跨函数重复代码;可以将知识隔离到一个函数或一组协调的函数(可能是个类)。
\end{itemize}
