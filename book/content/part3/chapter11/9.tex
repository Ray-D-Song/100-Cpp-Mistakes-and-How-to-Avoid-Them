这种错误最显著的影响是可读性(可理解性),在每个函数中放入少量代码会对效率产生负面影响。不过,还是可以提出一个论点从长远来看,编写简短的函数可以提高开发人员的生产力。

\mySamllsection{问题}

许多遗留函数很长,有些甚至很长。项目包含超过 1,000 行的函数并不罕见。我曾经使用过一个有 1,200 行代码的函数,但我从未理解它的大部分功能。通常,函数开始时很短,但随着时间的推移,会积累越来越多的代码来满足新的需求和变化。清单 11.15 中的代码需要重新设计,随着时间的推移,添加了越来越多的代码。开发人员需要编写所有代码,并(错误地)认为既然它们一起工作,就应该一起编写。

假设标准偏差代码仅适用于非负值——需要一些数据清理的理由!在处理之前必须清理数据,应消除所有负值。此后,计算算术平均值。最后,计算标准偏差。代码从上到下逻辑流畅,该函数也应该更容易阅读。

更糟糕的是,函数名称 std\_dev 并未传达将涉及数据清理的信息,这种错误命名很容易让读者误以为只计算标准差。正确命名函数很复杂,每个名称都应该简明扼要地传达函数的用途。这个函数应有简短、易懂的名称,准确地传达其所有功能。像compute\_standard\_deviation\_after\_removing\_negative\_values 这样的内容传达效果很好,但冗长;坦率地说,也很难阅读。

\filename{清单11.15 一个包含三个部分的过长的函数}

\begin{cpp}
double std_dev(const std::vector<double>& values) { // 1
  std::vector<double> new_values;
  for (int i = 0; i < values.size(); ++i) // 2
    if (values[i] >= 0)
      new_values.push_back(values[i]);

      double sum = 0.0;
  for (int i = 0; i < new_values.size(); ++i)
    sum += new_values[i];
  double mean = sum/new_values.size();

  sum = 0.0;
  for (int i = 0; i < new_values.size(); ++i) // 3
    sum += std::pow(new_values[i] - mean, 2);
  return std::sqrt(sum / (new_values.size() - 1));
}
int main() {
  std::vector<double> values;
  values.push_back(3.14159);
  values.push_back(-1.23456);
  values.push_back(2.71828);
  std::cout << "standard deviation is " << std_dev(values) << '\n';
  return 0;
}
\end{cpp}

{\footnotesize
注释1:顾名思义,只计算标准差

注释2:数据清理令人意外;没有记录

注释3:只有此代码执行函数名称所暗示的操作
}

\mySamllsection{分析}

此函数有三种不同的行为,这些行为合乎逻辑;必须每做一次才能得出答案。很容易假设,这些行为是必需的,所以可以或应该逐个地写出来。

虽然本书没有解决可测试性问题,但这仍然是编写简短、易于理解的函数的重要动机。随着越来越多的需求接受,新功能可能会添加到这个函数中,从而加速问题的解决。

\mySamllsection{解决}

函数应该简短并执行一个明确定义的行为。如果需要一系列行为(一系列函数调用),则主要有两种选择:错误选择和正确选择。错误的选择是将 std\_dev 函数重写为三个函数,并让用户(main 函数) 按正确顺序调用这三个函数。

我的学生反复听到,main 函数应该视为老板代码。老板负责指挥操作,但不负责处理细节——从来不想要事无巨细的管理者。如此指挥的函数必须处理细节,老板代码应该只提供足够的数据来解决问题,而非更多。要求老板正确命令函数以获得标准偏差,是在推卸开发人员的责任。

正确的解决方案是编写包含所需功能的三个函数,并添加第四个函数来协调。以下代码展示了这种方法,boss 代码像以前一样调用std\_dev 函数,而无需了解执行此操作所需的详细信息计算结果。我们假设这个名字已经确立;否则,应该更完整地拼写出来——有时,必须接受我们无法改变的事情。

\filename{清单11.16 拆分长函数并添加编排器}

\begin{cpp}
std::vector<double> cleanse(const std::vector<double>& values) {
  std::vector<double> good;
  for (int i = 0; i < values.size(); ++i)
    if (values[i] >= 0)
      good.push_back(values[i]);
  return good;
}

double arith_mean(const std::vector<double>& values) {
  double sum = 0.0;
  for (int i = 0; i < values.size(); ++i)
    sum += values[i];
  return sum/values.size();
}

double std_deviation(const std::vector<double>& values, double mean) {
  double sum = 0.0;
  for (int i = 0; i < values.size(); ++i)
    sum += std::pow(values[i] - mean, 2);
  return std::sqrt(sum / (values.size() - 1));
}

double std_dev(const std::vector<double>& values) { // 1
  std::vector<double> new_values = cleanse(values);
  double mean = arith_mean(new_values);
  return std_deviation(new_values, mean);
}

int main() {
  std::vector<double> values;
  values.push_back(3.14159);
  values.push_back(-1.23456);
  values.push_back(2.71828);
  std::cout << "standard deviation is " <<
    std_dev(values) << '\n'; // 2
  return 0;
}
\end{cpp}

{\footnotesize
注释1:管理详细信息的编排器功能

注释2:用户(老板)代码没有变化
}

\mySamllsection{建议}

\begin{itemize}
\item
将扩展函数拆分为小的、易于理解的函数。

\item
添加协调器函数来协调新的小函数。

\item
为函数命名,使其行为清晰易懂。

\item
如果可以使用重构代码的自动化工具(例如 IDE 插件),请学会使用;有些工具会将多个参数转换为参数对象 — 太棒了!
\end{itemize}

