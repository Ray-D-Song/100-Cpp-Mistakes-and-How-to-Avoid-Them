这个错误关注的是正确性和有效性。多个构造函数通常作为单独的操作实现,尽管它们重复了许多代码。此外,计算方法经常重复计算代码。这种重复应该导致考虑确定基础。并非所有重复都意味着必须确定基础,但这是一个好兆头,表明可能找到基础。basis 是一个数学概念,它定义了一组最小函数 - 所有函数都是必不可少的,并且没有重叠 - 可以从中实现其他非基础函数。don’t repeat yourself(DRY)原则与本次讨论相关,但必须服务于比通常呈现的更细微的目的 - 稍后会详细介绍。

实现多个构造函数或函数的差异可能会导致错误的行为。在一个地方支持基本代码可以更有效地利用程序员的时间并减少认知负担。

本讨论分析了构造函数,但适用于方法和独立函数。主要想法是减少行为实现的数量,以防止代码随着时间的推移而出现分歧。当功能重复并以不同方式维护时,事情很快就会出错。

通常,类会有多个构造函数,每个构造函数接受不同数量的参数。如果实例化调用中未提供某些信息,则会提供默认值。在其他情况下,将应用默认行为来填补知识空白。

\mySamllsection{问题}

一个简单的例子可能有助于更好地理解基函数的含义。考虑一个包装内置 int 类型的 integer 类。这四种基本数学运算可能彼此之间有足够的差异,因此所有这四种运算都可以唯一地实现。然而,仔细分析表明事实并非如此。减法问题 a – b 在数学上等同于 a + -b。减法可以用加法和反相来实现。因此,减法不需要是单独的运算。乘法是重复的加法,除法是重复的减法——或者更好的是,重复加一个反相值。因此,只需要两个基函数来实现这四个算术函数——将两个整数值相加并对一个整数值取反。其他三个运算可以根据这两个来定义。

这种操作或函数以其他函数的形式实现的思想确立了如何区分基函数,并直接导致以这种方式实现它们。通过在尽可能少的地方编写计算代码来确保正确性的特征。以这些方式实现的其他方法将正常工作,并且不会重复可能在发生变化时变得不匹配的代码。通过为每个函数编写一次核心代码来保持有效性的特征,并且通过在其他非基函数中反复使用来证明其正确性。

Cylinder 代码示例(清单 7.5)展示了一种典型的方法,即在未适当考虑基函数的情况下。其构造函数和操作是单独且独立地编码的。在类开发的早期,很容易假设代码在未来将基本保持不变。在很多情况下,这被证明是不正确的。通过添加一两个新要求,实现这些构造函数和函数可能会很快彼此分歧。任何分歧都会影响正确性、可读性和有效性。

\filename{清单7.5 类中重复的知识}

\begin{cpp}
class Cylinder {
private:
  double radius;
  double height;
  double area;
  double volume;
public:
  const double PI = 3.1415927;
  Cylinder() {
    radius = 1;
    height = 1;
    area = PI;
    volume = PI;
  }
  Cylinder(double h) {
    radius = 1;
    height = h;
    area = PI;
    volume = PI * h;
  }
  Cylinder(double r, double h) {
    radius = r;
    height = h;
    area = PI * r * r; // 1
    volume = PI * r * r * h; // 2
  }
  double getBaseArea() const { return area; }
  double getVolume() const { return volume; }
};
\end{cpp}

{\footnotesize
注释1:使用标准公式计算面积

注释2:体积等于面积*高度;再次计算面积
}

\mySamllsection{分析}

这些构造函数和函数的实现会重复知识。DRY 原则旨在防止这种情况。然而,DRY 通常被实现为“不重复代码”方法,而不是更有帮助的“不重复知识”方法。随着时间的推移,功能类将容易受到新需求的影响,从而导致代码的添加和修改。这些变化是不可避免的,知识重复的可能性也会增加。当复制的知识在一个地方发生变化而另一个地方没有变化时,就会发生分歧。一段时间后,就不清楚哪个版本是正确的。

每个构造函数都会重复实例变量的初始化,有时使用默认值,有时使用参数。这种重复表明了一种更好的方法。代码展示了一种典型的模式,可以将其重构为单个辅助函数。辅助函数的好处是所有构造函数都可以使用它,并防止知识和代码的重复。当添加新需求时,更改会被隔离到辅助函数中,从而防止发散。

\mySamllsection{解决}

解决这个问题的最佳方法是了解如何根据其他函数来实现函数,然后将该依赖关系缩小到最低限度。构造函数应该将标准代码分解出来,并将其放在私有辅助方法中。此辅助方法将成为根据辅助方法实现的构造函数的基础。
getBaseArea 方法计算圆柱体圆形底面的面积。volume 方法计算圆柱体圆形底面的面积,然后将其乘以其高度。能否分解出通用代码并进行单一面积计算决定了必要的基础。在这种情况下,体积方法应根据 basearea 方法而不是辅助函数来实现。

方法重用也发生在类继承中,其中派生类继承基类方法功能并对其进行添加。在许多情况下,重写的派生类方法可以调用基类方法作为其计算的一部分并根据需要对其进行修改。这种方法可以防止功能重复,并允许派生类从基类中受益。设计得当的基类方法应该是派生类基础的一部分。

这些重构步骤已在清单 7.6 中执行。已创建一个名为 init 的辅助方法来处理构造函数中的先前重复。

volume 方法现在使用 basearea 来计算圆柱体的面积,并将该值乘以高度。体积以底面积的形式实现,以防止在不可避免的变化发生时出现未来分歧。这个例子很简单,但却是这个错误所解决的问题的典型例子。在简单的情况下,知识重复可能不是问题。然而,这个问题可以扩展到更复杂的类,其中知识重复的影响更大。不要让简单性欺骗你,让你认为不会有错误。

\filename{清单7.6 函数的最小集合}

\begin{cpp}
class Cylinder {
private:
  double radius;
  double height;
  void init(double r, double h) {
    radius = r;
    height = h;
  }
public:
  const double PI = 3.1415927;
  Cylinder() { init(1, 1); }
  Cylinder(double h) { init(1, h); }
  Cylinder(double r, double h) { init(r, h); }
  double basearea() { return PI * radius * radius; } // 1
  double volume() { return basearea() * height; } // 2
};
\end{cpp}

{\footnotesize
注释1:计算面积

注释2:计算体积
}

\mySamllsection{建议}

\begin{itemize}
\item
将通用构造函数代码分解为辅助函数,该函数将成为类的基础集的一部分。

\item
考虑如何根据其他基础函数实现函数,以简化编码、 防止发散并为该知识维护一个真实来源。

\item
最小化基础函数集,确保它们的功能不重叠。

\item
添加新函数时,以基础函数为基础,并重新评估是否需要新的基础函数。

\item
如果这种方法变得难以操作或过于笨拙,请放宽限制。请记住,原则有助于最大限度地减少技术债务并加快开发速度,但盲目执行可能会导致更糟糕的后果。

\item
寻找机会将覆盖的方法重写为基础函数或集合。
\end{itemize}























