
很多时候,课程设计都是从一个相对纯粹的想法开始的,并且这个想法得到了很好的实现,但随后却面临着不可避免的变化。变化通常是因为类有效但似乎不完整而需要。然后,需要进行额外的更改,设计受到损害,权宜之计胜过精确性,并且会发生混乱。类腐烂但仍然有用。代码异味(非理想技术或实现,通常是由于时间限制或对问题空间或代码影响的理解不足)不断增加,但时间和预算决定了其他优先事项。

检测这些问题通常比确定解决方案要简单。常见错误通常可以修复,对代码库的其他部分的影响很小。但是,这些错误仅代表现有问题的一小部分。关于类设计错误的内容可以写成一整本书。

良好的类设计是理解和维护类不变性的基础。类表示一种数据类型,必须保持一致性和正确性,实例才有意义。忽略任何不变性方面都会导致实例出现不一致和可能失败的情况。客户期望正确性,作为开发人员,我们有责任满足这一期望。

\mySubsubsection{7.2.1.}{类不变量}

class invariant 是类对象在构造之后以及在该对象上的任何方法调用序列之间必须始终满足的条件。它表示对象的一致、有效状态,通常通过构造函数、析构函数和成员函数强制执行。类不变量可确保对象数据的完整性和正确性。

每个类都应该代表一个概念或一个实体;类是一种新的数据类型。它有意义地描述其概念或实体,聚合其他数据类型,从不同的类到原始类型。良好的设计和理解类不变量是同一枚硬币的两面。抽象原则意味着,首先,类将概念或实体的细节减少到最低限度,用于定义类。其次,抽象意味着整个实例可以作为一个单元来处理。这种实体或概念的抽象应该由客户端代码有意义地表现。类应该避免以意想不到的方式表现而让客户端代码感到意外。

由于类表示概念或实体,因此它必须将其内部数据限制为对其表示有意义的特定值。如前所述,Student 应该具有非负的 age。类是有关所表示实体或概念的知识的来源。它不应依赖任何其他类或外部数据来传达该知识。封装原则在某种程度上意味着类所知道和所做的一切都应完全包含在类中。客户端代码应该能够与实例交互或从实例获取任何必要的数据,而无需操作或查询其他对象或变量。

这些特征伴随着责任。类必须确保其所有数据成员(无论是单独还是整体)都是合法的,并且始终有意义,并且与它的交互是可预测的,不会出现意外。此外,它还必须确保在程序执行期间这种敏感性和可预测性永远不会受到损害。这个类的责任称为 class invariant 。不变量是实例的一个属性,它始终无论如何初始化或修改实例,都是 true。类绝不能创建不遵守不变量的实例,并且保护现有实例免受违反不变量的更改。

正确性特征对于程序的正确行为最为重要。如果程序代码不正确,则无法保证任何事情;无论多大的性能都无法弥补此错误,结果也是可疑的。保持类不变是确保正确性的重要手段。当类的行为受控且可预测时,客户可以放心地使用它,其结果也是有意义的。遗留代码中的类通常设计时考虑了不变量以外的其他因素。存在大量机会来清理许多现有类。遗憾的是,修改(即使是使类更可预测和更强大的修改) 可能会导致意外行为。一如既往,理想和理论必须尊重现实。目标是改进,但道路可能有些(或非常)坎坷。

\mySubsubsection{7.2.2.}{建立类不变量}

类表示一致、有凝聚力的概念或实体。类要求必须分别和总体指定每个实例变量的界限。总体表示实例的状态。一个重要的目标是确保实例的状态一致。继续使用 Person 的示例,如果 age 的值为负数,则Person 的含义将受到威胁——负数年龄代表什么?

建立类不变量是构造函数的工作;维护不变量是修改器或其他改变状态的函数的责任。每个实例变量必须初始化并保持有意义且正确的值。实例的解释取决于它是否保持不变量。变量类型的值范围可能会超出不变量的范围,因此构造函数和变量必须确保没有值超出范围。我观察到一些代码,这些代码使用访问器来验证范围并仅返回有意义的值,从而将实例变量初始化或设置为无效值。如果任何实例变量表示无效值,则必须考虑违反类不变量。

本章中的错误主要集中在建立类不变性上;因此,本章重点介绍构造函数。下一章将单独讨论修改器,修改器是维持类不变性所必需的。这两章截然不同,但必须协调一致,共同努力才能使类不变性保持正确。
