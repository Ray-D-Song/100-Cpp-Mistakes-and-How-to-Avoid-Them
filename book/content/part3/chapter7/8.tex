这个错误严重影响正确性,因为它影响了类的不变性。因为阅读使用默认构造函数的代码,无需检查便可知道初始化值为何,所以可读性也会受到影响。

确保对象的实例化保持类的不变量,是构造函数的主要职责。许多类需要对实例变量进行特定的初始化,以使其处于有意义的状态。当未通过参数指定初始化值时,提供默认构造函数以向变量提供有效(但不一定有意义)的值。如果无法确定正确的默认值,则默认构造函数将实例初始化为不一致的状态。假设类设计者希望开发人员稍后用有意义的值填充这些值。如果没有调用一个或多个变量器来满足这一期望,就会出现问题。

\mySamllsection{问题}

默认构造函数可能看起来更有效,但通常情况并非如此。默认值必须在稍后修改,以添加有意义的数据,但会使有效性和可读性变得复杂。假设在使用实例之前更改初始化值,性能会就会受到影响。

实例的状态中,不应包含无效或无意义的数据。如果使用默认构造函数,则必须将类设计为有意义的默认值,这些默认值无需进一步修改即可使用。

清单 7.14 中的代码有些就这样。由于没有给出开发者提供的默认构造函数,因此编译器提供了其默认版本,其将所有类实例初始化为其默认值。内置实例根本没有初始化,name 为空字符串(一个奇怪的名称),而age 未定义(垃圾)。只有调用容易忘记的 setter 来提供有意义的值,才能实现对此类对象的有意义使用。如果没有验证,无效值可能会分配给实例变量,所以此类 setter 也很危险。

\filename{清单7.14 默认构造函数,导致无意义的结果}

\begin{cpp}
class Person { // 1
private:
  std::string name; // 2
  int age; // 3
public:
  std::string getName() { return name; }
  int getAge() { return age; }
};

int main() {
  Person p1; // 4
  std::cout << p1.getName() << ' ' << p1.getAge() << '\n';
}
\end{cpp}

{\footnotesize
注释1:开发者没有提供默认构造函数;因此,编译器会编写一个

注释2:类实例将调用其默认构造函数

注释3:内置实例将不会初始化(将包含垃圾数据)

注释4:将调用编译器编写的默认构造函数 — name 将初始化,但 age 不会
}

添加显式的默认构造函数也并没有什么好处:

\begin{cpp}
Person() : name(""), age(0) {}
\end{cpp}

但 age 实例变量将初始化,从而避免未定义的行为。

\mySamllsection{分析}

不可能出现没有姓名,且年龄为零的人;但如果默认构造函数是必要的,则必须允许这种危险情况发生。默认值可以更 改为 Lakshmi 和 21,但这样一来,所有默认的人的名字都是Lakshmi,年龄都是 21。这种方法没有任何好处。

唯一有意义的构造函数是

\begin{cpp}
Person(const std::string& name, int age) : name(name), age(age) {}
\end{cpp}

需要两个参数来创建一个 Person 对象;实例变量用这些值初始化,从而避免使用(可能)毫无意义的默认值。

类不变量要求每个数据成员都具有有意义的数据,默认构造函数经常违反这一原则。

创建对象数组时必须使用默认构造函数。在这种情况下创建数组,并通过调用默认构造函数初始化每个元素。除了极少数情况外,每个元素都会修改类的不变量。

\mySamllsection{解决}

创建对象时,初始化所有实例变量对于保持正确性至关重要。少数情况下,默认构造函数可能是一种合理的方法。由于元素的初始化数据尚未确定,必须调用默认构造函数来初始化实例变量。有时,可以使用数组,但会出现问题。代价是效率略有提高,而不正确。数组通常分配在堆栈上(除非使用 new 关键字创建)。并且,教科书隐含地权威低位,使它们在学生心中占据首要地位。当有选择时,先学到的东西通常会首先使用。所以,在几乎所有情况下,都应该首先教授vector,并且优先使用vector而非数组。

vector使用一个间接层(指针)来访问数据,数组在访问过程中会稍微快一些。但开发人员必须能够证明,这种微小的增益对正确性的影响处于合理的范围。数组非常典型也经常使用,但其缺点很大。更好的方法是选择一个不进行默认构造的容器。

清单 7.15 展示了一个使用数组和可比较vector的简单示例。数组使用默认构造函数,导致混乱。vector需要为每个元素使用双参数构造函数,以保证正确(非默认)初始化;vector要求元素可复制。现代 C++ 可以在创建数组时,使用初始化列表来缓解默认构造函数,但前现代 C++ 就没有这么幸运了。

\filename{清单7.15 优先使用vector而非数组}

\begin{cpp}
class Person {
  private:
  std::string name;
  int age;
public:
  Person() : name(""), age(0) {}
  Person(const std::string& name, int age) : name(name), age(age) {}
  std::string& getName() { return name; }
  int getAge() { return age; }
};

int main() {
  Person people[2]; // 1
  for (int i = 0; i < 2; ++i) // 2
    std::cout << people[i].getName() << " is " << people[i].getAge() << '\n';

    std::vector<Person> peeps;
  peeps.push_back(Person("Susan", 21));
  peeps.push_back(Person("Jason", 25)); // 3

  for (int i = 0; i < peeps.size(); ++i) // 4
    std::cout << peeps[i].getName() << " is " << peeps[i].getAge() << '\n';
}
\end{cpp}

{\footnotesize
注释1:调用默认构造函数,会让每个元素错误地初始化

注释2:元素的数量可能会改变,但这个值可能会被忽略

注释3:向vector中添加任意数量的元素,并跟踪其数量

注释4:添加或删除元素后,此循环总能正常工作
}

\mySamllsection{建议}

\begin{itemize}
\item
不要仅因为默认构造函数很熟悉或很常见就添加它们,它们也可能很危险。

\item
请仔细考虑是否要添加默认构造函数,除非每个实例变量都正确初始化。

\item
优先使用vector或其他容器而非数组,因为数组需要使用默认构造函数来初始化每个元素。

\item
切记类不变量和默认数据成员的含义——如果默认构造对象是有意义的,那么就这样做;如果没有,请避免使用默认构造函数。
\end{itemize}







