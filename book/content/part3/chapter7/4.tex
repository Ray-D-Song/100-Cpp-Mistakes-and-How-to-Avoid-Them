内置的 C++ 数据类型(例如 int 和 double)直观、语法简单,且性能卓越。开发人员设计类时,这些特性必不可少。类设计就是类型设计;必须考虑构成正确,且有用类型的所有因素。很少有其他语言能够提供 C++ 为类设计者提供的灵活性,但必须考虑几个问题,包括内存分配和释放、对象实例化、初始化和销毁;重载和友元函数,以及类的不变量。

类型决定了实例所执行的一组操作。经常会需要相应的子类型,因此必须仔细考虑继承的正确使用。如果类或类型是基类,则必须定义一个对所有子类型都有意义的接口。必须确保子类型不会破坏类的不变量。

正确且有意义的类型设计,为读者和开发人员提供了一种易于使用且定义清晰的感觉。正确性对于类设计至关重要,但必须重视实例使用结果的可读性。良好的设计还可以有效地使用类型,这样开发者就不必弥补缺失或尴尬的功能。最后,良好的类型设计会在以下情况下的使用方式:集合或大型集合;这会影响性能。请仔细考虑运行时间和空间成本,以减少与算法选择不当相关的问题。

类的四个主要特征影响其设计、编码和使用方式。必须仔细考虑,以确保所有相关人员都能有效使用该类。有效使用通常归结为类提供的一组操作。设计这组操作时,应尽可能简单让开发者可以直接使用。

\mySamllsection{问题}

清单 7.3 是一个人为的例子,展示了考虑不周的数据类型实现为类时,出现的几个问题。开发者认为,通过开发一个简单的Rational 类,有理数的概念就可以轻松抽象和使用。乍一看,代码看起来很合理,但它包含几个错误。该类旨在允许对Rational 实例进行加法、乘法和输出,但使用此类会产生尴尬的代码。

其理念是使用有理数进行计算,然后输出结果。使用此代码的客户很快就会发现,没有提供常规的预期操作(例如,使用 plus 方法,而非 + 操作符),并且使用起来变得混乱且不清楚。这种难以理解的方法,还会浪费开发人员的时间。

\filename{清单7.3 Rational类的原生实现}

\begin{cpp}
class Rational {
private:
  double num;
  double den;
public:
  Rational() : num(0), den(1) {}
  Rational(double n, double d) : num(n), den(d) { reduce(); }
  void setNumerator(double n) { num = n; }
  void setDenominator(double d) { den = d; }
  double getNumerator() { return num; }
  double getDenominator() { return den; }
  static int gcd(int a, int b) { return a == 0 ? b : gcd(b % a, a); }
  void reduce() {
    int div = gcd(num, den);
    num = (den > 0 ? 1 : -1) * num / div;
    den = abs(den) / div;
  }
  Rational plus(const Rational& o) const { // 1
    int n = num * o.den + den * o.num;
    int d = den * o.den;
    return Rational(n, d);
  }
  Rational times(const Rational& o) const { // 2
    int n = num * o.num;
    int d = den * o.den;
    return Rational(n, d);
  }
  void print() { std::cout << num << '/' << den; } // 3
};

int main() {
  Rational r1(1, 3);
  Rational r2(2, 4);
  Rational r3 = r1.plus(r2);
  r3.print();
  std::cout << '\n';
  Rational(1, 0);
  return 0;
}
\end{cpp}

{\footnotesize
注释1:命名不当的加法操作符

注释2:命名不当的乘法操作符

注释3:命名不当的输出操作符
}

\mySamllsection{分析}

这一类 Rational 最突出的特点是误解了有理数的本质。从本质上讲,有理数由比率组成(因此得名 rational,它并不像我最初认为的那样,指的是“合理”的数字)。数论断言分子和分母是整数,而不是双精度数。遵循合理的数学推理至关重要。接下来,考虑主函数中返回之前的最后一行代码。这种情况未定义,因为分母是一个除以零的问题。

必须提出一个重要问题:有理数会改变吗?我更喜欢不可变数据;这是该原则的一个典型例子。可变版本必须使用不同的考虑因素。数字永远不会改变,所以有理数构造之后,就不应该改变其值(但这取决于要解决的问题)。尽可能在所有地方删除变量。考虑独立于分母获取分子是否有意义——这时,可能没有意义。需要删除访问器。

确定用户是否应该能够调用 gcd 或 reduce。这些函数是作为Rational 类的辅助函数提供,不应成为公共接口的一部分 — 将其标记为 private。最后,确定这些函数是否应该内联;通常,递归函数不会内联,但实际结果取决于编译器。通过在类内部定义,可将其隐式内联。这种方法将按照编译器认为合适的方式实现。可以让 gcd 保持隐式内联,其很简单并且在类中实现;编译器将确定其实际情况,但 reduce 不太可能通过内联来节省重大开销。

考虑一下 plus 和 times 函数的含义。设计者希望其是加法和乘法,但不清楚这些函数是否会改变调用对象(对r1 使用 plus),或者是否会创建一个用计算结果初始化的新对象。源码可以说明这个问题,但因为功能实现的反直觉会严重影响可读性。

print 函数意图良好,但不太直观。人们可能认为 std::cout 是使用 Rational 实例输出的自然流,而好的设计在形式上通用,但这种设计是具体且不灵活的。这三个设计不良的函数严重影响了效率,因为其不直观,也不遵循已知的使用模式。

\mySamllsection{解决}

此代码是重新设计的版本,考虑了有理数的含义,并在代码中进行实现。plus 和 times 方法已重新实现,以使用标准算术 + 和 * 操作符。用户可以使用它们直观地使用预期计算符号的代码。print 方法已更改为为 ostream 类重载 operator<{}<。这种方法能让开发人员使用标准插入操作符,将数据添加到输出流 - 就像所有内置数据类型一样。这种一致性使代码更加自然有效。

\filename{清单7.4 提供更加直观的实现}

\begin{cpp}
class Rational {
  private:
  int num;
  int den;
  static int gcd(int a, int b) { return a == 0 ? b : gcd(b % a, a); }
  void reduce();
  int validate(int v) {
    return v != 0 ? v : throw
      std::invalid_argument("zero denominator");
  }
public:
  Rational(int n, int d=1) : num(n), den(validate(d)) { reduce(); }
  Rational operator+(const Rational& o) const;
  Rational operator*(const Rational& o) const;
  friend std::ostream& operator<<(std::ostream&, const Rational&);
};
void Rational::reduce() {
  int div = gcd(num, den);
  num = (den > 0 ? 1 : -1) * num / div;
  den = abs(den) / div;
}
Rational Rational::operator+(const Rational& o) const { // 1
  return Rational(num * o.den + den * o.num, den * o.den);
}
Rational Rational::operator*(const Rational& o) const { // 2
  return Rational(num * o.num, den * o.den);
}
std::ostream& operator<<(std::ostream& out, const Rational& r) { // 3
  out << r.num << '/' << r.den;
  return out;
}

int main() {
  Rational r1(1, 3);
  Rational r2(2, 4);
  std::cout << r1 + r2 << '\n';
  //Rational(1, 0);
  return 0;
}
\end{cpp}

{\footnotesize
注释1:自然加法操作符

注释2:自然乘法操作符

注释3:自然插入操作符
}

通过将实例变量设为整数,可以体现有理数的含义。辅助函数会巧妙地隐藏起来,仅供类使用。除以零的运算得到适当处理。操作符直观易懂,用户了解加法和乘法不会影响调用(或左侧)对象。插入操作符的工作方式与其他操作符一样,所以用户可以像使用其他操作符一样使用 — 显著提高了可读性和有效性。

考虑内联与外联函数定义时,有几个问题需要考虑。以下是需要考虑的一些问题:

\begin{itemize}
\item
内联函数可以通过将函数代码,直接嵌入到调用点来减少函数调用开销,这可能会提高频繁调用小函数的性能。但是,过度内联会增加代码大小并降低缓存性能。外联函数通常具有函数调用的开销,但不会增加调用点的代码大小,这有助于保持更好的指令缓存局部性。

\item
如果过度使用内联函数或将其用于大型函数,因为代码会插入到每个调用点,会导致代码膨胀。外联函数有助于保持可执行文件的大小较小,因为函数代码是重复使用的,而非重复。

\item
内联函数会使调试更加困难,因为函数代码在多个位置重复,使堆栈跟踪和调试工作变得复杂。外联函数将实现集中起来,使其更易于维护和调试。

\item
内联函数通常在头文件中定义,如果内联函数发生更改,则会增加重新编译多个翻译单元的风险。外联函数在源文件中定义,对这些函数的更改不会影响头文件和依赖于头文件的文件,有助于最大限度地减少重新编译。

\item
内联函数通常在头文件中定义,实现的细节会暴露,很不理想的封装。外联函数可以隐藏源文件中的实现细节,从而改善封装,以及接口和实现的分离。

\item
内联函数最适合简单函数,以内联方式定义可以将相关代码放在一起,提高可读性。外联函数最适合复杂函数,分离实现有助于保持类定义的清晰易读。
\end{itemize}

\mySamllsection{建议}

\begin{itemize}
\item
请记住,实现类设计就是设计一种新的数据类型。对其含义有基本了解的人来说,使用类型时操作不应反直觉。

\item
当基于符号的操作符,能够在数据类型上下文中准确传达操作的含义时,勿必对其进行定义。切勿仅仅因为它们“看起来很酷”而使用。

\item
保持数据流和代码设计的通用性,尽可能不要将用户锁定在一条特定的路径上。最好的方法是设计一种数据类型,使其接近内置类型的自然操作。

\item
明智地使用内联,仔细考虑内联和外联方法之间的权衡。
\end{itemize}
