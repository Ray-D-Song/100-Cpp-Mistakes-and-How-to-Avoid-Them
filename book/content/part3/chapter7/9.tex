可读性(或缺乏可读性)是本次讨论的主要动机,但继承关系不当会对正确性产生不利影响,并影响有效性。

公有继承提供了从现有类创建新类的能力。这些新派生的类必须通过“is-a”关系与基类相关。受保护继承和私有继承本质上是不同的,本错误中不对其进行讨论。当函数具有基类实例的参数(或指向或引用)时,派生类对象将按预期工作。此属性是Liskov替换原则。继承向读者传达信息;因此,为了传达正确且有意义的信息,公有派生类实例必须可以替代基类实例,并且表现得像基类实例。

\mySamllsection{问题}

正如其他地方提到的,一些教科书仍然认为继承是一种共享代码的巧妙方法,可以为开发人员节省一些时间。这种方法听起来像是一种编写更少代码,并获得更多好处的有效方法。然而,这个建议会把粗心的开发人员逼入绝境。

公共继承意味着基类是一个通用的概念或概念,可以通过多种方式进行特化。例如,一个描述二维多边形的 Shape 类。此外,程序需要操作 Circle、Square 和 Triangle 对象。由于所有这些都与 Shapes 相关并且具有典型行为(例如,area 和 perimeter),因此将其建模为 Shape 类的派生类很有意义。这样做之前,必须考虑是否存在 Circle、Square 或 Triangle 对象无法满足 “is-a” 关系的情况。这种情况下,答案是否定的,但有些情况远没有那么清晰或设计得那么干净。

相反,考虑这样一种情况:我们有一个 Bird 类,并希望从中派生出一个 Ostrich 类,因为鸵鸟肯定是鸟类。这种继承似乎满足了 is-a 属性,但这里有一个问题。我们直觉地认为鸟类是会飞的动物,但鸵鸟不能飞。要解决的问题决定了类之间的适当关系;有时,必须最小化或忽略自然关系。

回到 Shape 示例,继承(尤其是公共继承)最令人信服的论据是,程序通常必须以通用方式处理相关对象的集合。基类接口函数指定通用行为——为什么它是基类。派生类可以重写这些函数以提供特有的行为。如果问题不需要处理相关对象的集合,则可能不需要继承层次结构。

\mySamllsection{分析}

开发人员尝试使用继承来关联几何形状,通用行为处理各种类似乎合理。因此,他们认为将使用具有通用功能的基类,而派生类可以根据实例的实际类型特化行为。由于矩形和正方形非常相似,因此从矩形派生正方形,并强制其高度和宽度之间建立更严格的关系是一个好主意。

Square 类是为了共享来自 Rectangle 类的代码而开发的,这保证了有效性,即使可读性略有下降。但共享代码的愿望(可以理解为节省一些击键次数的愿望),很快就在现实的岩石上搁浅。Square 似乎是 Rectangle 的一个特例,直到人们意识到,高度和宽度不能独立变化的矩形并不是矩形——只是具有高度和宽度,而这并没有定义一个真正的矩形。虽然矩形的高度和宽度可以具有相同的值,但这种罕见性并不能证明将 Square 设计为 Rectangle 的一个特例是合理的。矩形必须保持具有独立高度和宽度属性的能力。这种独立性是Rectangle 类对每个实例不变量的一部分。

\filename{清单7.16 不正确的继承,似乎节省了很多编码工作}

\begin{cpp}
class Rectangle {
private:
  double height;
  double width;
public:
  Rectangle(double h, double w) : height(h), width(w) {}
  double getHeight() { return height; }
  void setHeight(double h) { height = h; }
  double getWidth() { return width; }
  void setWidth(double w) { width = w; }
  virtual void validate() { assert(height >= 0 && width >= 0); }
};

class Square : public Rectangle { // 1
public:
  Square(double s) : Rectangle(s, s) {}
  void validate() override {
    Rectangle::validate();
    assert(getHeight() == getWidth());
  }
};

int main() {
  std::vector<Rectangle*> shapes { new Rectangle(3, 4),
      new Square(2) };
  for (auto shape : shapes) {
    // guarantee different lengths
    shape->setHeight(shape->getWidth() + 1); // 2
    shape->validate();
  }
  return 0;
}
\end{cpp}

{\footnotesize
注释1:尝试保存一些击键操作会导致出现此派生

注释2:Square 实例将具有不同的高度和宽度值!
}

虽然此代码正常运行期间不会出现问题,但 validate 函数暴露了问题。Square 实例必须具有相等的高度和宽度(根据定义),但Rectangle 实例必须可以自由地具有不同的大小。当测试类不变量时,问题就暴露出来了 — Square 不是Rectangle(反之亦然)。

\mySamllsection{解决}

清单 7.17 展示了继承的良好用法,每个派生类在需要时都可以像Shape(主函数中的循环)一样工作。每个派生类都通过继承与其他派生类相关,但代表不同的想法。每个类都可以验证其类不变量,但都不会偏离基类的接口。

这种方法不是从 Rectangle 开始,而是从一个抽象基类开始(理想情况下,每个函数都是纯函数;至少有一个必须是纯函数),该基类定义所有派生类的行为。然后,矩形和正方形没有从基类派生的“is-a”关系。Rectangle 和 Square 类是兄弟关系,而非父子关系。每个类都可以强制执行其对边长的约束,而不会影响其他类。想象一下将 Triangle 类添加到这个组合中,将具有更多不同的约束。将类添加到正确排序的继承层次结构,不会导致令人头疼和黑客式的解决方法。

\filename{清单7.17 将要点抽象为基类}

\begin{cpp}
class Shape { // 1
public:
  virtual double area() = 0;
  virtual double perimeter() = 0;
  virtual void validate() = 0;
};

class Rectangle : public Shape { // 2
private:
  double height;
  double width;
public:
  Rectangle(double h, double w) : height(h), width(w) {}
  double area() override { return height * width; }
  double perimeter() override { return (height + width) * 2; }
  void validate() override { assert(height >= 0 && width >= 0); }
};

class Square : public Shape { // 3
private:
  double side;
public:
  Square(double s) : side(s) {}
  double area() override { return side * side; }
  double perimeter() override { return side * 4; }
  void validate() override { assert(side >= 0); }
};

int main() {
  std::vector<Shape*> shapes { new Square(2), new Rectangle(3, 4) };
  for (auto shape : shapes)
    hape->validate();
  return 0;
}
\end{cpp}

{\footnotesize
注释1:抽象基类仅定义行为

注释2:此类以其特定的方式实现行为

注释3:此类以其惯用的方式实现行为,与其他类不同
}

节省一些击键次数来共享代码是一种危险的诱惑,但最终却带来了灾难。只有当派生类是基类,因此在所有情况下可以替代基类时,才能适当地维护公共继承关系。在许多实际层次结构中,一些代码在派生类和基类之间共享,从而导致其重用,从而提高效率。请记住,代码重用(包括节省击键次数)是继承的优点,而非使用继承的原因。

\mySamllsection{建议}

\begin{itemize}
\item
代码共享(重用)是公共继承的一个好处,但绝不是其原因。

\item
仅当派生类在每种情况下都可以正确替代基类实例,并且为每个基类行为都具有有意义的定义函数时,才使用公共继承。

\item
如果现实世界中的行为不需要在基类中建模,请避免实现该行为的诱惑,即使直觉表明应该对其进行建模。删除所有当前不需要的行为 — 这是 you ain’t gonna need it (YAGNI) 原则的实际应用。

\item
类应该通过继承来关联,它们是相关实体,并且用于集合中,其中每个实例都像基类对象一样运行,也许具有其行为的一些特化。
\end{itemize}












