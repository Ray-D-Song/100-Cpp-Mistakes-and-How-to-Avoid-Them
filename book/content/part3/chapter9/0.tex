本章内容

\begin{itemize}
\item
在运算符和方法之间进行选择

\item
处理自赋值和返回值优化

\item
与类成员的编译器协作

\item
在显式和隐式转换之间进行决定

\item
增量和减量运算符的前缀和后缀版本
\end{itemize}

本章继续讨论如何更好地使用实例。类不变量的概念始终存在,但解决这些错误需要与前几章有所不同。这当然并不意味着它不那么重要,只是重点更广泛,涉及不一定直接影响对象状态的领域。

这些错误经常涉及性能类别,其中一些错误通过消除不必要的临时对象来强调这一方面。当表达式求值的中间步骤需要中间对象来保存部分求值的结果(这些结果将在进一步求值中使用)时,就会出现这些临时对象。了解这些临时对象的创建时间以及如何设计一个类来消除其中的许多临时对象,会显著影响构造函数和析构函数调用的次数。

其他错误集中在误用影响常见用法或性能的运算符。C++ 提供了 大量为开发人员提供灵活性,并且必须尊重这种灵活性以避免不当使用功能。
