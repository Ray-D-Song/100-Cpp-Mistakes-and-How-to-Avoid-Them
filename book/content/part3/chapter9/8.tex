这个错误主要影响性能;可读性和有效性可能会受到轻微的负面影响,所以请考虑你的优先事项。

C++ 编译器使使用混合模式计算和函数调用变得非常容易。这种情况通常发生在将函数调用或算术运算应用于两种不同类型时。如果编译器可以找到一种方法将值从一种类型转换为另一种类型,从而使调用成功,那么它将默默地(通常)有帮助地完成此操作。但是,这种转换可能会有性能损失。

\mySamllsection{问题}

隐式转换感觉很好,当它们顺利且正确地工作时很有趣。清单 9.20 中的代码显示了一个适度的 Complex 类,考虑到有人可能希望将双精度值添加到实数部分。不明显的是,没有办法进行这种简单的添加。定义的 operator+ 接受两个 Complex 参数;显然,double 不是Complex——静默的、可能有用的构造函数类型转换来救援!

\filename{清单9.20 隐式、静默和昂贵的构造函数类型转换}

\begin{cpp}
class Complex {
private:
  double real;
  double imag;
public:
  Complex(double real, double imag=0) : real(real), imag(imag) {}
  double getReal() const { return real; }
  double getImag() const { return imag; }
};
const Complex operator+(const Complex& lhs, const Complex& rhs) {
  return Complex(lhs.getReal()+rhs.getReal(), lhs.getImag()+rhs.getImag());
}

int main() {
  Complex c1(2.2); // 1
  Complex c2 = c1 + 3.14159; // 2
  Complex c3 = 2.71828 + c1; // 2
  Complex c4 = 2.71828 + 3.14159; // 1
}
\end{cpp}

{\footnotesize
注释1:一次构造函数调用

注释2:两次构造函数调用
}

正如预期的那样,调用 Complex 构造函数来创建对象 c1。它还为对象 c2 和 c3 调用了两次。第一次调用是转换调用,其中 double 值转换为其等效的 Complex 表示。double 值初始化 real 组件,而 imag 组件默认为零。最后,对象 c4 是单个构造函数调用。四行代码中有六个构造函数调用。如果这是一个更复杂的数据类型,还会调用六个析构函数调用。在这种简单的纯数据类型的情况下,大多数(如果不 是全部)编译器都会优化析构函数。

\mySamllsection{分析}

Complex 对象 c1 和 c4 的构造显而易见,每个对象只需调用一次构造函数。这种行为无法消除。对象 c2 和 c3 必须先将 double 转换为Complex 对象,然后执行加法,因为 operator+ 只接受 Complex 对象。转换需要调用一次构造函数;加法的结果需要调用第二次构造函数。将返回的对象赋值给声明的对象不需要额外的构造函数调用,因为 RVO 会就地复制值。

此操作的所有方面都是正确的,但如果性能是关注点,额外的构造函数(和析构函数)调用可能会成为痛点。如果不关心性能,这仍然是一种低效的方法,可以改进。我们需要一种不需要额外调用构造函数即可进行混合模式转换的方法。

\mySamllsection{解决}

在性能不是问题的情况下,上述方法可以接受,并且代码量最少。这提高了可读性和效率(小菜一碟)。但是,在不允许额外调用构造函数(和析构函数)的情况下,必须有一种方法可以最大限度地减少这些调用。

解决方案是使用不同的参数列表重载 operator+,这些参数列表可以接受 Complex 和 double 的任意组合,但需要多写几行代码(效率会降低)。两个参数均为 double 类型的情况不包括在内,也不可能包括。C++ 确保重载运算符至少具有一个用户定义的参数类型。

如果编译器允许这种情况,则将重新定义两个 double 值的相加,从而导致与内置的相加规则不一致的行为。他们考虑到了一切!性能问题的解决方案很简单:为预期使用的每个组合编写一个重载的operator+。以下代码显示了 Complex 类的改进。再添加两个运算符即可完成可能的重载参数列表集(Complex/Complex、double/ Complex 和 Complex/double)。

\filename{清单9.21 通过提供重载操作符来最小化构造函数转换}

\begin{cpp}
class Complex {
private:
  double real;
  double imag;
public:
  Complex(double real, double imag=0) : real(real), imag(imag) {}
  double getReal() const { return real; }
  double getImag() const { return imag; }
};
const Complex operator+(const Complex& lhs, const Complex& rhs) {
  return Complex(lhs.getReal()+rhs.getReal(), lhs.getImag()+rhs.getImag());
}
const Complex operator+(const Complex& lhs, double rhs) { // 1
  return Complex(lhs.getReal()+rhs, lhs.getImag());
}
const Complex operator+(double lhs, const Complex& rhs) { // 1
  return Complex(lhs+rhs.getReal(), rhs.getImag());
}
int main() {
  Complex c1(2.2);
  Complex c2 = c1 + 3.14159;
  Complex c3 = 2.71828 + c1;
  Complex c4 = 2.71828 + 3.14159;
}
\end{cpp}

{\footnotesize
注释1:重载运算符定义消除了混合模式的构造函数转换
}

确保运算符具有所有必要的重载,这将最大限度地减少临时对象的影响。构造函数(和析构函数)调用的数量已减少到最少四个。此代码表明,在实现混合模式计算、函数调用和类型转换构造函数时,性能会受到显著影响。但是,不要过度使用 - 只重载必要的运算符,而不 是所有运算符。困惑吗?这是我们尽力而为的平衡行为。

\mySamllsection{建议}

\begin{itemize}
\item
如果担心性能,则对混合模式函数或运算符调用重载每个预期模式;如果性能不是问题,这仍然是一种很好的彻底方法。

\item
请记住,运算符必须至少具有一种用户定义的参数类型,以保持语义一致性并防止重新定义现有规则。
\end{itemize}
