此错误主要针对正确性。除非读者理解问题及其解决方案,否则对可读性也会有轻微影响。

许多数据类型表示允许对象重复的概念。std::string可包含产品名称,例如 Automobile 类的品牌和型号。在对停车场中的汽车进行建模时,可以预期会出现多辆具有相同品牌和型号的汽车;同一辆车出现多次很正常。

然而,有些数据类型代表着独特的概念,复制它们的想法违反了类不变量。经典 C++ 没有特定于语言的方法来防止复制唯一对象,因此开发者必须维护此属性。

\mySamllsection{问题}

让我们考虑在一个系统中对艺术品进行建模,也许是一个为潜在买家制作目录的系统。每件艺术品都是独一无二的,必须如此处理。清单 9.4 中的代码没有定义复制对象的方法,表面上保留了不变量唯一性属性。最初,该类有一个公共的复制赋值操作符,必须对其进行更正。开发人员(错误地)认为将其移至私有部分,将排除创建重复项的可能性。

\filename{清单9.4 为避免复制唯一对象而进行的错误编码}

\begin{cpp}
class ArtPiece {
private:
  int id;
  std::string description;
  ArtPiece& operator=(const ArtPiece& o) { // 1
    this->id = o.id;
    this->description = o.description;
    return *this;
  }

public: // 2
  ArtPiece(int id, std::string d) : id(id), description(d) {}
  std::string getDesc() { return description; }
};

int main() {
  ArtPiece ml(333, "Mona Lisa");
  ArtPiece ts(444, "The Scream");
  std::cout << ml.getDesc() << '\n';

  ArtPiece dup(ml); // 3
  std::cout << dup.getDesc() << '\n';
  // dup = ts; // 4
  return 0;
}
\end{cpp}

{\footnotesize
注释1:已存在的复制赋值操作符移到了私有区

注释2:没有编写公共复制构造函数或复制赋值操作符

注释3:这似乎行不通

注释4:私有复制赋值操作符会阻止这段代码运行
}

开发人员认为,随着复制赋值操作符的移动,将消除复制对象的能力;毕竟,无法调用复制赋值操作符——忘记了默认的复制构造函数!开发人员考虑了复制唯一对象的效果,并选择隐藏作为防止这种情况发生的手段。输出为

\begin{shell}
Mona Lisa
Mona Lisa
\end{shell}

一片混乱!

\mySamllsection{分析}

如果开发者未指定内容,编译器会编写默认的复制构造函数和复制赋值操作符。由于复制构造函数未编码,因此 default 版本会创建 dup 对象并生成副本。复制成功,而非失败,并且没有伪造的迹象(没有明显违反类不变量的唯一性属性),这种(草率的)方法存在重大问题。

因为编译器提供了默认版本的复制构造函数,所以创建 dup 的代码可以顺利编译和执行,确保对每个实例变量进行浅拷贝。但默认行为会产生伪造的副本,而无法区分。

开发者没有考虑到,如果一个类没有开发者定义的复制构造函数,编译器会很乐意免费提供一个默认版本。类必须设计时,需要避免过于依赖于编译的操作。可以抑制其他类型的不良功能,仅仅通过不编码函数即可;这两个函数不属于那种类型。开发者必须明确进行阻止。

一种已取得一定成效的方法是,显式定义复制构造函数和复制赋值操作符,设为私有。这在很大程度上解决了问题。但如果类中的代码恰好做了一些不合适的事情(复制或赋值),编译器会很乐意遵守,并破坏不变性(伪造)。

\mySamllsection{解决}

上述方法有部分正确的想法,但需要包含另一个重要方面。缺少的部分是,提供函数体的每个操作符的私有定义都是可执行代码,所以不应指定函数体。解决方案很简单:声明但不定义这两个操作。下面清单中的代码一举解决了这两个问题。

\filename{清单9.5 通过声明默认操作来避免重复}

\begin{cpp}
class ArtPiece {
private:
  int id;
  std::string description;
  ArtPiece(const ArtPiece&); // 1
  ArtPiece& operator=(const ArtPiece&); // 1
public:
  ArtPiece(int id, std::string d) : id(id), description(d) {}
  std::string getDesc() { return description; }
  void badMethod() { ArtPiece a = *this; } // 2
};

int main() {
  ArtPiece ml(333, "Mona Lisa");
  ArtPiece ts(444, "The Scream");
  // ArtPiece dup(ml); // 3
  // ts = ml; // 3
  std::cout << ml.getDesc() << '\n';
  ts.badMethod(); // 4
  return 0;
}
\end{cpp}

{\footnotesize
注释1:私有且未实现

注释2:基于类的赋值编译时没有警告

注释3:因为操作是私有的,所以不会编译

注释4:没有编译错误
}

这两个重复函数被声明为私有函数,且未定义。用户中任何试图使用任一操作的代码都将失败,正如预期的那样,保持类不变,因为面向用户的复制赋值操作符和复制构造函数都不会编译。编译器发现这两个操作都是私有的,并阻止用户代码使用它。但事情并不顺利 。

清单 9.5 中引入了 badMethod 方法。该方法的主体执行了编译器无法阻止的赋值,而在面向用户的代码中可以这样做;语法是合法的,但其语义值得怀疑。编译器对代码没有问题,代码编译得很干净;由于没有错误,所以没有任何消息来说明发生了一个错误。这句话听起来可能不对,但它是正确的。问题出现在链接器中。链接器尝试在外部代码中找到复制赋值操作符的定义,但找不到并产生未定义的引用错误。对于独立于链接进行编译的项目来说,这个错误可能更直接。行为可能与预期不同,避免发生是最安全的途径。

如果复制构造函数和复制赋值操作符,是公共的并且设置为 =delete ,现代 C++ 编译器可以捕获此错误。以下代码清单显示了一个示例 。请注意注释掉的代码;这些错误现在是非法的。

\filename{清单9.6 使用delete=避免重复}

\begin{cpp}
class ArtPiece {
private:
  int id;
  std::string description;
public:
  ArtPiece(int id, std::string d) : id(id), description(d) {}
  ArtPiece(const ArtPiece&) = delete;
  ArtPiece& operator=(const ArtPiece&) = delete;
  const std::string& getDesc() const { return description; }
  //void badMethod() { ArtPiece a = *this; }
};

int main() {
  ArtPiece ml(333, "Mona Lisa");
  ArtPiece ts(444, "The Scream");
  // ArtPiece dup(ml); // 1
  // ts = ml; // 1
  std::cout << ml.getDesc() << '\n';
  // ts.badMethod(); // 1
  return 0;
}
\end{cpp}

{\footnotesize
注释1:现代 C++ 可以避免这些错误。做得好!
}

\mySamllsection{建议}

\begin{itemize}
\item
未实现的复制构造函数和复制赋值操作符由编译器默认,这可能会产生意外或不必要的行为。

\item
确保知道哪些对象是唯一的,并且不能复制;如果源实体留空,则可以传输值。

\item
成员使用现代 C++ =delete 来删除,这会强制出现编译时错误,而非链接时错误。
\end{itemize}
