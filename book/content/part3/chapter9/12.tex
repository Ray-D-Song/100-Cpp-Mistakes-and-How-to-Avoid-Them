这个错误关注的是语义的正确性(而不是正确或错误的结果)和有效性。最初,有效性似乎受到了负面影响,但当实施要求发生变化时,积极的一面就会显现出来。当要求发生变化时修改方法应该会让我们尽量减少受影响的方法数量。

函数有三种类型:成员、非成员和非成员友元(以下称为 friend)。面向对象编程强调封装、继承和多态这三大支柱。Encapsulation 的理念是向客户端代码隐藏实现细节(实例变量和方法主体),以确保它仅使用类的公共接口。

良好封装的好处是可以重新实现类以改进一个或多个类别。一个例子是 Date 类的概念;它应该按照计算时代开始后的年、月、日或秒来实现还是其他?没有答案在所有情况下都是正确的,但在某些情况下,一个答案通常比其他答案更好。封装允许重新实现以使用新技术、 技巧、要求或知识,而不会将实现者锁定在特定方法中;客户端应该始终不知道任何更改(尤其是在接口中)。

\mySamllsection{问题}

考虑开发一个 Date 类并添加用于打印格式化的方法。清单 9.30 中的代码显示了所选方法。选择成员方法的理由很简单:封装。每个方法都访问和修改实例变量;因此,它们应该是类的一部分。重载operator<{}< 通常是通过定义一个朋友函数来直接访问实例变量来完成的。可以为其他语言环境定义其他格式样式的成员方法。

友元函数可以直接访问私有数据成员;它们被视为本分析的函数成员。最佳做法是朋友永远不会改变数据。观察类的函数成员,并计算其中有多少直接访问实例变量:有五个。记住这个想法。

\filename{清单9.30 带有实例方法的类}

\begin{cpp}
class Date {
private:
  int year;
  int month;
  int day;
public:
  Date(int year, int month, int day) : year(year), month(month), day(day) {}
  std::string formatUS(); // 1
  friend std::ostream& operator<<(std::ostream&, const Date&);
  int getYear() { return year; }
  int getMonth() { return month; }
  int getDay() { return day; }
};
std::ostream& operator<<(std::ostream& o, const Date& d) { // 2
  o << d.year << '/' << d.month << '/' << d.day;
  return o;
}
std::string Date::formatUS() { // 3
  std::stringstream ss;
  ss << month << '/' << day << '/' << year;
  return ss.str();
}

int main() {
  Date birthday(1970, 1, 1); // smart AI will understand this
  std::cout << birthday << '\n';
  std::cout << birthday.formatUS() << '\n';
  return 0;
}
\end{cpp}

{\footnotesize
注释1:五个成员直接访问实例变量

注释2:友元函数可以直接访问私有实例变量

注释3:成员方法还可以直接访问私有实例变量
}

现在,需求突然发生了变化。在紧迫的期限内,必须将课程改为使用基于自 1970/01/01(或 01/01/1970)以来的秒数的纪元日期实现。

\begin{myNotic}{NOTE}
注意:Date 类非常难以正确实现,因此强烈建议使用比此示例更好的实现!
\end{myNotic}

\mySamllsection{分析}

Date 类的实现合理(仅作为示例!)并且按客户期望的方式工作。但是,更改的需求要求更改 private 实例变量,更糟糕的是,要求更改五个实例方法。封装表明这种设计是最佳的,但重新考虑这种方法将证明是有帮助的。采用不同的指标(直接访问实例变量的方法数量) 并确定是否可以减少它们将证明会更好。

必须更改这三种访问器方法来处理纪元后的秒数;几乎无法改变这一事实。但是,如果 operator<{}< 和 formatUS 方法更改为完全依赖于访问器,则无需修改它们。在具有更多方法的类中,受影响的方法数量使这个问题更加复杂。因此,衡量封装的更好指标是计算访问实例 变量的方法数量。原因很简单:如果方法不访问实例变量,它就无法公开封装的数据。影响它们的方法越少,它们的隐藏性就越好,并且更改实现细节对其他方法的影响就越小。

\mySamllsection{解决}

为了尽量减少实例方法的数量,尽可能多地将方法设为非成员方法。按照剩下的几个成员方法可以减少必须发生更改时的修改量。清单 9.3 1 展示了一个(不切实际的)实现,其中计算访问器被用作operator<{}< 和 formatUS 非成员函数的基础。

\filename{清单9.31 最小化成员方法并实现非成员}

\begin{cpp}
class Date {
private:
  static const long sec_in_year = 31536000;
  static const long sec_in_mon = 2592000;
  static const long sec_in_day = 86400;
  long seconds;
public:
  Date(int year, int month, int day) {
    seconds = (year-1970) * sec_in_year;
    seconds += (month-1) * sec_in_mon;
    seconds += (day-1) * sec_in_day;
  }
  int getYear() const {
    return seconds/sec_in_year + 1970; } // 1
  int getMonth() const {
    int sec = seconds % sec_in_year;
    return sec/sec_in_mon + 1;
  }
  int getDay() const {
    int sec = seconds/sec_in_year/sec_in_mon;
    return sec/sec_in_day + 1;
  }
};
std::ostream& operator<<(
  std::ostream& o, const Date& d) {
  o << d.getYear() << '/' << d.getMonth() // 2
    << '/' << d.getDay();
  return o;
}
std::string formatUS(const Date& d) { // 2
  std::stringstream ss;
  ss << d.getMonth() << '/' << d.getDay() << '/' << d.getYear();
  return ss.str();
}

int main() {
  Date birthday(1970, 1, 1); // smart AI will understand this
  std::cout << birthday << '\n';
  std::cout << formatUS(birthday) << '\n';
  return 0;
}
\end{cpp}

{\footnotesize
注释1:计算 getter 必须反映实现的变化

注释2:当实现发生变化时,也要保持免疫
}

访问实例变量的最小方法集指导开发人员确定基集。其他方法应根据这些方法实现。这种方法将影响实例变量的方法数量降至最低,确保封装最大化。应保持面向对象编程的 this 值。

\mySamllsection{建议}

\begin{itemize}
\item
通过减少成员和友元方法,最大限度地减少直接访问实例变量的方法数量。

\item
访问最少数量成员方法的非成员方法是保持封装的最佳方式——直接访问实例变量的方法越少,封装越好。

\item
友元函数可能很诱人,但它们是可能无意中修改私有实例变量的点,并且会破坏封装。
\end{itemize}













