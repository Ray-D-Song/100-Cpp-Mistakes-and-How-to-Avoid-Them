这个错误主要集中在性能上,但也有一点,就是效率。使用运算符的类很容易比必要更频繁地创建临时对象。在某些早期的 C++ 编译器 中,由于表达式求值期间临时对象的激增,执行某些算术运算的成本更高。这个问题的核心在于求值此类表达式所需的构造函数调用次数。

已经尝试了各种方案来消除一些临时值。其中两种尝试是返回指针和返回引用。然而,生成的代码很难读懂,有时甚至会出错。其他方法需要笨拙的语法。这些提出的解决方案阻碍了它们帮助减少临时值。

\mySamllsection{问题}

考虑这样一种情况,一些工程师需要编写一个复杂的类。清单 9.7 是一个典型的实现,可以满足他们的需求。一切看起来都很干净、高效,所以工程师们很高兴;然而,随着时间的推移,他们开始抱怨他们的性能迟缓。这段代码可能没有任何问题,所以开发人员会去别处寻找。(一定要衡量你的性能假设!)下面的清单显示了用于创建和添加复数的交付代码。

\filename{清单9.7 添加复数的典型方法}

\begin{cpp}
class Complex {
private:
  double real;
  double imag;
public:
  Complex(double r, double i) : real(r), imag(i) {std::cout<<"x\n";}
  Complex(const Complex& o) : real(o.real), imag(o.imag)
    { std::cout<<"y\n"; }
  friend const Complex operator+(const Complex&, const Complex&);
};
const Complex operator+(const Complex& lhs, const Complex& rhs) {
  Complex sum(lhs.real+rhs.real, lhs.imag+rhs.imag); // 1
  return sum; // 2
}
int main() {
  Complex c1(-2, 3.3); // 3
  Complex c2(3, -1); q // 4
  Complex c3 = c1 + c2; // 5
}
\end{cpp}

{\footnotesize
注释1:第三个构造函数调用

注释2:第四个构造函数调用复制构造函数

注释3:第一个构造函数调用

注释4:第二个构造函数调用

注释5:第五个构造函数——复制构造函数
}

此代码有五个未进行返回值优化 (RVO) 的构造函数调用。前两个是构造 c1 和 c2 对象。第三个是在 operator+ 方法中创建 sum 对象。返回 sum 对象时,将创建一个临时对象并使用复制构造函数用其值初始化该对象,这是第四次调用。最后,从临时对象创建 c3 对象,进行 第五次构造函数调用。

\begin{myNotic}{NOTE}
由于非 RVO 问题普遍存在,当前的编译器会自动实现它。因此,很难找到不执行 RVO 的编译器,所以这个错误只适用于较旧的编译器。但是,仍然建议使用 RVO 的编码建议,因为它们更简洁。GNU g++ 编译器提供了 -fno-elide-constructors 标志来禁用 RVO 以用于实验(仅!)原因。
\end{myNotic}

\mySamllsection{分析}

已经付出了很多努力来消除一些构造函数调用,这将最大限度地减少临时对象的成本。尝试创建一个对象并返回其指针使得调用语法很糟糕。指针方法可以按照以下清单所示实现。

\filename{清单9.8 通过返回一个指针来最小化构造函数调用}

\begin{cpp}
class Complex {
private:
  double real;
  double imag;
public:
  Complex(double r, double i) : real(r), imag(i) { std::cout << "x\n"; }
  Complex(const Complex& o) : real(o.real), imag(o.imag) {
    std::cout<<"y\n"; }
  friend const Complex* operator+(const Complex&, const Complex&);
};
const Complex* operator+(const Complex& lhs, const Complex& rhs) { // 1
  Complex* cpx = new Complex(lhs.real+rhs.real, lhs.imag+rhs.imag);
  return cpx;
}
int main() {
  Complex c1(-2, 3.3);
  Complex c2(3, -1);
  Complex c3 = *(c1 + c2); // 2
}
\end{cpp}

{\footnotesize
注释1:消除一次构造函数调用

注释2:通过使构造函数调用变得笨拙和不自然来消除构造函数调用,但它有效——谁删除对象?
}

此代码消除了一个构造函数调用,但迫使客户端代码以不自然且不直观的方式编写。更糟糕的是,它分配了堆内存,这总是比自动(堆栈)内存更昂贵。这种方法很难阅读,并且浪费了程序员的时间。

尝试在操作符主体内创建对象并返回对该对象的引用是诱人的,而且通常可行(但是,必须忽略非常有用的警告消息)。但要小心以下清单中显示的陷阱——有龙!

\filename{清单9.9通 过返回引用来最小化构造函数调用}

\begin{cpp}
class Complex {
private:
  double real;
  double imag;
public:
  Complex(double r, double i) : real(r), imag(i) { std::cout << "x\n"; }
  Complex(const Complex& o) : real(o.real), imag(o.imag) {
    std::cout<<"y\n"; }
  friend const Complex& operator+(const Complex&, const Complex&);
};
const Complex& operator+(const Complex& lhs, const Complex& rhs) { // 1
  Complex cpx(lhs.real+rhs.real, lhs.imag+rhs.imag);
  return cpx;
}
int main() {
  Complex c1(-2, 3.3);
  Complex c2(3, -1);
  Complex c3 = c1 + c2; // 2
}
\end{cpp}

{\footnotesize
注释1:消除一个构造函数调用

注释2:这是一个简单自然的语法,但这是错误的!
}

运算符中创建的 cpx 对象一直存在,直到函数结束并被销毁。对象是在堆栈框架中创建的,堆栈框架已失效。返回对现已销毁的对象的引用意味着它是对无效对象的引用。使用该对象会导致未定义的行为,具体情况会有所不同,具体取决于各种条件。危险的意外是,对象所在的堆栈框架通常不会被覆盖,代码将正常工作。但有一天,当需要工作时,它会突然神秘地失败。烧毁!

\mySamllsection{解决}

不要尝试创建一个对象并返回一个指针(很尴尬!)或一个引用(错误!),而是编写构造函数,以便编译器可以使用 RVO 消除不必要的临时对象。许多教科书演示的是 create an object and return it 模式,而不是更优雅、更高效的 return the creation of an object 模式。

清单 9.10 演示了更好的方法,其中运算符在 return 语句中创建一个新对象。RVO 消除了在返回点和客户端代码中的赋值处的两个复制构造函数调用。编译器可以使其更加高效,但一如既往,并不是每个人都能从后来的技术中受益。

\filename{清单9.10 为RVO编码}

\begin{cpp}
class Complex {
private:
  double real;
  double imag;
public:
  Complex(double r, double i) : real(r), imag(i) {std::cout<<"x\n";}
  Complex(const Complex& o) : real(o.real), imag(o.imag)
    { std::cout<<"y\n"; }
  friend const Complex operator+(const Complex&, const Complex&);
};
const Complex operator+(const Complex& lhs, const Complex& rhs) {
  return Complex(lhs.real+rhs.real, lhs.imag+rhs.imag); // 1
}
int main() {
  Complex c1(-2, 3.3);
  Complex c2(3, -1);
  Complex c3 = c1 + c2; // 2
}
\end{cpp}

{\footnotesize
注释1:编码充分利用了 RVO——消除了一次复制构造函数调用

注释2: RVO 消除了第二次复制构造函数调用
}

Complex 对象的创建需要调用一次构造函数,但无论如何都必须这样做。目标不是阻止构造,而是消除临时对象。使用 RVO,可以在 c3 占用的空间中构造 c1 和 c2 的总和,从而消除所有复制构造函数调用。必须在运算符主体中构造一个对象;重点是通过编码消除复制构造函数的成本,以便编译器的 RVO 机制发生。

\mySamllsection{建议}

\begin{itemize}
\item
许多当前的编译器都实现了 RVO;请仔细阅读文档以了解您的编译器是否实现了 RVO。

\item
返回构造函数调用的结果,而不是在运算符主体中创建的对象,以确保代码容易成为 RVO 的目标。

\item
运算符可以返回指针,但是调用语法很混乱,并且必须有人负责删除对象。

\item
操作符永远不应该返回对本地对象的引用;当调用代码访问它时,本地对象已经被销毁了。
\end{itemize}










