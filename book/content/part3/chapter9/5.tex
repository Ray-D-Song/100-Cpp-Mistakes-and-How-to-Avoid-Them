此错误主要涉及正确性,而很少涉及性能。这是一个问题,主要是在类中使用动态资源时。无法处理自赋值可能会导致这些资源出现严重错误。

从一个实例到另一个实例的赋值经常发生。复制赋值运算符用于处理 此功能。正确编写的复制赋值运算符必须正确处理动态资源,以防止违反类不变量、唯一对象和资源泄漏。但是,当赋值发生在同一个对象之间时,一个微妙的问题可能会让开发人员犯难。它似乎不太可能发生自赋值,但使用指针时,实际对象比命名变量更 模糊。如果自赋值处理不当,可能会丢失数据、发生崩溃或出现未定义的行为。

\mySamllsection{问题}

考虑清单 9.14 中的代码,它在从不同的 Auto 对象分配时创建一个新的 Engine 对象。处理动态资源的模式是正确的,但代码中隐藏着一个重大问题。假设删除现有的 Engine 是必要的(确实如此),因此代码在复制之前将其删除。由于分配源和目标相同,因此尝试获取 VIN 是对已删除数据的访问。希望代码崩溃时不会出现意外行为或数据损坏;否则,请做好噩梦的准备。使用 valgrind 和内存清理器等工具是一种非常好的做法,正如这个错误所证明的那样。

\filename{清单9.14 带有隐患的对象赋值}

\begin{cpp}
class Engine {
private:
  std::string vin;
public:
  Engine(std::string vin) : vin(vin) {}
  std::string getVin() { return vin; }
};

class Auto {
private:
  Engine* engine;
public:
  Auto(Engine* engine) : engine(engine) {}
  Auto& operator=(const Auto& car) {
    if (engine) // 1
      delete engine;
    engine = new Engine(car.engine->getVin()); // 2
    return *this;
  }
};

int main() {
  Engine* e1 = new Engine("123456789");
  Auto mustang(e1);
  mustang = mustang;
}
\end{cpp}

{\footnotesize
注释1:删除现有引擎

注释2:访问已删除的引擎数据
}

\mySamllsection{分析}

这种资源处理在某些情况下可能是正确的,但使用复制赋值运算符时,就会出现错误。赋值代码首先删除现有的 Engine 对象。然后,它从源提供的 VIN 分配一个新的 Engine 对象。虽然看起来不错,但删除是一个大问题。

删除 Engine 对象可确保删除现有的 Engine 以防止资源泄漏。从参数的 VIN 值分配新的 Engine 旨在仅从该数据创建新的 Engine;但是,数据来自刚刚删除的对象。在许多赋值情况下,这两个对象不会相同,因此此代码可以正常工作。但在这种情况下,源和目标是同一个对象。这种情况是自我赋值,现在代码会导致问题。因此,运行良好的代码现在崩溃(好日子)或执行一些未定义的行为(非常糟糕的一天)。

我的系统很幸运——尝试访问已删除的发动机 VIN 时会收到有关重复释放或损坏的错误消息。此错误表明出了什么严重问题,我必须弄清楚(我感到惭愧)。也许其他系统不会显示错误。祸啊!

\mySamllsection{解决}

问题是,当对象和参数是同一个实体时,删除 Engine 对象会导致问题。在其他情况下(它们不是同一个对象),则不存在问题。需要的是检查接收对象和参数对象是否相同的代码。在这种情况下,不应发生删除。在这种情况下不应该发生任何事情。

以下清单中的代码展示了一种更好的方法,可以处理自赋值并防止对这些动态资源进行不当处理。通过简单的测试确定源和目标是否是同一对象可以防止不当行为。

\filename{清单9.15 在拷贝赋值操作符中处理自赋值}

\begin{cpp}
class Engine {
private:
  std::string vin;
public:
  Engine(std::string vin) : vin(vin) {}
  std::string getVin() { return vin; }
};

class Auto {
private:
  Engine* engine;
public:
  Auto(Engine* engine) : engine(engine) {}
  Auto& operator=(const Auto& car) {
    if (this == &car) // 1
      return *this;
    if (engine) {
      delete engine; // 2
      engine = 0; // 3
    }
    engine = new Engine(car.engine->getVin()); // 4
    return *this;
  }
};

int main() {
  Engine* e1 = new Engine("123456789");
  Auto mustang(e1);
  mustang = mustang;
}
\end{cpp}

{\footnotesize
注释1:自我赋值测试

注释2:仅当源不是目标时才删除

注释3:如果可以,请使用 nullptr

注释4:安全地访问有效数据
}

上述失败案例中的赋值不太可能发生;没有人会那么不负责任,对吧?但是,如果指针存储在容器中,则从视觉上发现将对象赋值给自己的能力会变得模糊不清。类似这样的情况可能会导致自我赋值,但根本不明显:

\begin{cpp}
autos[i] = autos[j];
\end{cpp}

此外,指针可能会引入意外的自我分配,如下例所示:

\begin{cpp}
*pcar1 = *pcar2;
\end{cpp}

必须对自我分配进行测试;如果发现这种情况,则需要快速退出复制分配操作符并且不删除任何资源。

\mySamllsection{建议}

\begin{itemize}
\item
出于性能和正确性原因,始终在每个复制和复合赋值运算符中测试自我赋值。

\item
确保在删除动态资源后将指针清零。

\item
如果目标对象与源对象相同,则立即退出。

\item
请警惕通过指针进行自我赋值,无论是直接进行还是使用容器元素进行。
\end{itemize}






