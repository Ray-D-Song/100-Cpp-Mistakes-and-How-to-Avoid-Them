这个错误主要集中在可读性上,其次是正确性。阅读代码应该能说明全部情况;隐式转换会隐藏一些细节。

当函数调用未收到预期类型的参数时,C++ 提供了一组复杂的规则,用于将一种数据类型转换为另一种数据类型。使用两种形式的隐式转换,具体取决于是否定义了转换函数和转换构造函数。此错误主要针对转换函数。

\mySamllsection{问题}

转换运算符使用 operator 关键字定义,后跟要转换的数据的类型,后跟括号。例如,清单 9.18 中的 Rational 类可能返回表示有理数近似值的 double 值。double() 运算符是 Rational 类成员,它将Rational 值转换为其近似值 double。

\filename{清单9.18 一个带有转换函数的理性类}

\begin{cpp}
class Rational {
private:
  int num;
  int den;
public:
  Rational(int num, int den = 1) : num(num), den(den) {}
  operator double() { return (double)num/den; }
  friend std::ostream& operator<<(std::ostream&, const Rational&);
};
std::ostream& operator<<(std::ostream& out, const Rational& r) {
  out << r.num << '/' << r.den;
  return out;
}
int main() {
  Rational r1(3);
  std::cout << r1 << ' ' << (double)r1 << '\n'; \\ 1
  if (r1 == 3) \\ 2
    std::cout << "equal\n";
  else
    std::cout << "not equal\n";
  return 0;
}
\end{cpp}

{\footnotesize
注释1:显式转换非常易读

注释2:隐式转换容易引起误解;它看起来像是在比较整数
}

清单 9.18 中的代码按预期运行,因为定义了朋友 operator<{}< 。其输出为

\begin{shell}
3/1 3
equal
\end{shell}

没有办法准确预测要比较的实际值(0.333 值)。我的学生经常听到我说:“浮点值是近似值,很少精确。”这种比较浮点值的方法是错误的。应始终使用 delta-epsilon 方法进行比较 - 请参阅“另请参阅” 部分,了解解决此技术的错误。

\mySamllsection{解决}

如果需要转换函数,最好使用函数而不是运算符。当编译器隐式尝试将值从一种类型转换为另一种类型时,不会考虑该函数。在无法进行 转换的情况下,编译器将出现错误,从而提醒开发人员函数不足。开发人员可能会意识到所编码的内容没有意义,并选择不同的方法。

下面的代码通过添加转换函数并消除隐式转换来避免这两个问题。此外,对读者来说,更明显的是发生了显式转换。

\filename{清单9.19 使用显式转换函数的理性类}

\begin{cpp}
class Rational {
private:
  int num;
  int den;
public:
  Rational(int num, int den = 1) : num(num), den(den) {}
  double toDouble() { return (double)num/den; }
  friend std::ostream& operator<<(std::ostream&, const Rational&);
};
std::ostream& operator<<(std::ostream& out, const Rational& r) {
  out << r.num << '/' << r.den;
  return out;
}
int main() {
  Rational r1(3);
  std::cout << r1 << ' ' << r1.toDouble() << '\n';
  if (r1.toDouble() == 3)
    std::cout << "equal\n";
  else
    std::cout << "not equal\n";
  return 0;
}
\end{cpp}

删除 double 运算符并添加 toDouble 函数使隐式变得显式。现在,比较对读者来说更加明显,并且类型不匹配应该显而易见。由于比较双精度值而不是有理数而存在的任何不精确性对于代码的读者来说都是显而易见的。使用 operator== 比较双精度几乎总是一个错误的选择。

\mySamllsection{建议}

\begin{itemize}
\item
尽量少用隐式类型转换运算符,除非读者能明显看出它们的用途,但这种情况很少见。

\item
编写显式类型转换函数可增强可读性并防止意外转换和错误假设。
\end{itemize}
