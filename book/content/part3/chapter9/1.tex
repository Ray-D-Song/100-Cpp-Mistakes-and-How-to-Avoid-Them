
这种错误主要影响可读性,如果对变量的范围和含义的理解出现错误,则可能会对正确性和有效性产生不利影响。

命名变量是一项艰巨的任务。开发人员必须很好地命名实体以避免混淆。随着开发人员对代码的理解不断发展,找到良好的命名实践可能需要多次迭代。存在许多可以声明变量的范围。这些范围包括语句、本地、类、命名空间、全局和文件。

变量在写入其类型和名称时,其作用域从声明点开始,并限于声明它的文件、块或语句的末尾。通常,作用域以文件末尾、语句作用域中的右括号或分号结束。当后面的内部作用域定义另一个具有相同名称,但不一定相同类型的变量时,变量将被遮蔽。内部作用域内的代码,无法在外部作用域中看到同名变量。某些情况下,可以使用作用域解析 operator:: 来访问被遮蔽的外部作用域变量。

程序规模大于几个函数,代码中变量的名称通常相同。由于这些函数不共享作用域,不会发生遮蔽,但当内部作用域声明与外部作用域中名称相同的变量时,开发者必须在赋值或计算中使用正确的变量。

\mySamllsection{问题}

当作用域定义同名变量时,就会发生遮蔽。作用域越广,越有可能发生遮蔽。全局变量对于编译单元中的所有函数都可见,从声明点到源文件末尾。一般不鼓励使用全局变量,但人们却常常这么做。过度使用全局变量首先会损害正确性和可读性。使用全局变量时,必须谨慎命名。只有当这些短名称是众所周知的概念(如 PI 或 E)时,为全局变量选择简单、简短的名称才合理。对于鲜为人知的值,应使用命名方案(如用下划线分隔的全大写字母(或类似方案))来清楚地区分全局变量和非全局变量。最好使用命名空间来避免遮蔽错误。

通常,需要更好地实践清晰的命名。假设有一个名为 I 的全局变量;只有阅读过它的定义(以及有用的注释)后才能理解其含义。声明或定义离使用代码越远,变量的名称就应该越长、越有表现力。大多数公司或项目都有一些指导方针;请严格遵循这些指导方针——这关乎可读性和合作性。

许多情况下,短变量名会在多个作用域中使用。这种情况使得理解正在使用哪个变量,变得更加困难。然而,短变量名很常见且方便。

当使用变量名时出现错误时,就会发生内容类型隐藏错误;所用变量的作用域为内部,而预期变量位于外部作用域。清单 9.1 中的代码演示了全局变量的名称与实例变量相同。仅出于说明目的,sum 函数将错误地使用变量作用域。

sum 函数错误地使用了实例变量两次,其使用了内部和外部作用域的变量。应该使用作用域解析操作符,来消除变量名称的歧义。这是对内容的错误掩盖,是一种语法错误——可能只错别字,可以通过正确命名变量轻易解决。

\filename{清单9.1 不同作用域中相同的变量名}

\begin{cpp}
double rad = 1.0; // 1
class Circle {
private:
  double rad; // 2
public:
  Circle(double rad) : rad(rad) {} // 3
  void setRadius(double rad) { // 3
    if (rad < 0)
      throw std::invalid_argument("negative radius");
    rad = rad;
  }
  double sum() const {
    return rad + rad; // 4
  }
  double getRadius() { return rad; } // 5
};

int main() {
  Circle c(3);
  std::cout << "radius is " << c.getRadius() << '\n';
  std::cout << "enlarged radius is " << c.sum() << '\n';
  return 0;
}
\end{cpp}

{\footnotesize
注释1:名称很短的全局变量,含义不明确

注释2:名称很短的实例变量

注释3:与实例变量同名的参数,需要消除歧义

注释4:两个变量同名,但使用局部变量,遮蔽了全局版本

注释5:使用实例变量 ,它是作用域内唯一的变量
}

清单9.2的目的是将一个随机值数组传递给 SearchAnalyzer,这将确定数组中出现多少个值。由于此代码必须修改数组才能进行高效搜索,其他 Analyzer 派生类不需要复制数据,共享数据的思维模式建立。调用 analyze 方法时,预期结果是一行文本,解释在数据中找到多少个随机搜索值,但学生惊讶地发现没有找到任何值。那么,下一步是给教授发电子邮件求助!

\filename{清单9.2 不同变量名含义的混淆}

\begin{cpp}
class Analyzer{
protected:
  int* cloneValues(int* a, int size){
    int* arr = new int[size];
    for (int i = 0; i < size;i++)
    arr[i] = a[i];
    return arr;
  }

  int* array; // 1
  int size;
public:
  Analyzer(int*values, int size) : array(values), size(size) {}
  virtual std::string analyze() = 0;
  virtual ~Analyzer() { delete[] array; }
};

class SearchAnalyzer : public Analyzer{
public:
  SearchAnalyzer(int* values, int size) : Analyzer(cloneValues(values,
      size), size) { // 2
    selection_sort(values, size); // 3
  }
  std::string analyze() {
    int count = 0;
    for (int i = 0; i < 100; ++i)
      if (binary_search(array, rand() % SIZE, size)) // 4
        ++count;
    std::stringstream ss;
    ss << "There were "<< count << " random values found.";
    return ss.str();
  }
};

int main() {
  int* numbers = createArray(SIZE);
  SearchAnalyzer searcher(numbers, SIZE);
  std::cout << searcher.analyze() << '\n';
  return 0;
}
\end{cpp}

{\footnotesize
注释1:指向数组的受保护指针

注释2:对 cloneValues 方法的调用

注释3:错误地将 values 变量用于排序方法

注释4:搜索数组变量;并未排序!
}

这个问题很简单,但做起来却很容易。参数值可理解为要搜索的随机值列表,但这个列表需要复制和排序,从而创建一个代表预期数据的新变量。人们可以专注于输入变量,而忘记那些不显眼的复制值;当其他代码没有单独的数据时,尤其如此。

修复很简单:在对 selection\_sort 的调用中将values 更改为 array,一切即可按预期工作。

不完全是遮蔽,当对相同内容使用相似的变量名时,就会发生一种常见错误。比实际遮蔽更有害的是,清单 9.2 中的代码故意遮蔽,其中同名变量不会导致问题;因此,修复内容遮蔽的简单方法并不适用。相反,两个具有相似含义的变量是错误的——语义错误。该代码取自提交了无法完成项目的学生。

\mySamllsection{分析}

全局变量经常使用,但很少以一种容易与类或局部变量区分开来的方式进行限定。清单 9.1 中的全局变量与代码中的类和参数变量同名。本例中,很容易看出 setRadius 参数 rad 应该初始化同名的实例变量。然而,更广泛或更晦涩的代码,会极大地掩盖明显的含义和用法。开发人员倾向于在类或函数中,为全局变量提供简短的名称,使其看起来很“明显”。只要充分说明变量的含义和目的,使用就很简单了。

再来看一下 setRadius 方法:代码需要引用全局 rad 变量,但这是什么意思,它在哪里?如果类和方法没有这样的名称,则会进行搜索以查找外部范围变量;这种情况下,将使用全局变量。这种方法在时间和理解方面都不直观。

当可变参数与实例变量同名时,this 关键字必须区分实例变量和参数变量。如果没有这种区别,参数将赋值给自身,如清单 9.1 所示。有些编译器不会对这种情况发出警告,而是愉快地编译,并生成不符合预期的代码。我教学生使用这种模式作为经验法则。与正在初始化的实例变量同名的参数,是重要的文档;使用 this 指针对于消除歧义至关重要。如果使用这种方法,请养成习惯;否则,偶尔可能会无法使用 this 指针,并会需要调试。

\mySamllsection{解决}

为了便于阅读,读者必须理解初始化列表每个部分的含义。括号外的变量是实例变量(清单 9.3 中的 this->radius 部分),括号内的变量是参数(本例中的 double radius)。效率略有提高,开发人员可以复制名称,而无需在短期记忆中保留单个字母或缩写。

\filename{清单9.3 由命名空间区分的相同变量名}

\begin{cpp}
namespace global { // 1
  double radius = 1.0;
};
class Circle {
private:
  double radius;
public:
  Circle(double radius) : radius(radius) {}
  void setRadius(double radius) {
    if (radius < 0)
      throw std::invalid_argument("negative radius");
    this->radius = radius; // 2
  }
  double sum() const {
    return radius + global::radius; // 3
  }
  double getRadius() const { return radius; }
};

int main() {
  Circle c(3);
  std::cout << "radius is " << c.getRadius() << '\n';
  std::cout << "enlarged radius is " << c.sum() << '\n';
  return 0;
}
\end{cpp}

{\footnotesize
注释1:使用命名空间来确保不会发生遮蔽

注释2:记住要养成使用此模式的习惯,否则就不要使用它

注释3:变量名的明确使用
}

隐藏名称是一种记录用于初始化或分配实例变量的参数的有效方法;构造函数或赋值器的参数名称和实例变量名称相同。一些开发人员建议将参数编码为单个字符(实例变量的第一个字符)或名称的缩写版本,以合理地传达意图。更好的建议是在实例变量前面或后面加上一个符号(例如:下划线),以表明该变量是私有的(只有实例变量可以是私有的)。

全局变量的命名方式应使其明显是全局变量。这种方法可避免在需要使用其他变量时,出现无意使用或混淆。有两种方法有助于确保这一点清晰。首先,考虑一种标准化的全局变量命名方式。例如,在每个变量前加上 GLOB\_ 或类似的标记。另一种方法是将 \_g 或 \_global 作为后缀添加到变量名中。记住变量距离其使用位置越远,其名称就应该越长、越具有描述性,以传达其意图和目的;这两个选项很好地解决了这个建议。其次,考虑使用命名空间来包含所有全局变量。这使用种方法使理解全局变量会容易得多。命名空间包含可确保全局变量不会分散在源码中,并且以明显的模式命名。此外,在 enum 或类中包含全局变量,提供了具有相同好处的替代方案。代码演示了在计算连续复利时,如何使用命名空间:

\begin{cpp}
namespace Constants {
  const double e = 2.7182828283;
}
...
amount = principal * std::pow(Constants::e, rate*time);
\end{cpp}

\mySamllsection{建议}

\begin{itemize}
\item
名称与用途的距离越远,名称就应该越长、越具有描述性,使用方式也应该越具体(例如:命名空间)。

\item
尽可能避免使用全局变量,以避免出现设计和遮蔽问题。

\item
请记住,重复变量名称可能会导致难以理解代码的意图及其正确用法。

\item
注意内容的遮蔽(简单版本)和意图的遮蔽(更复杂的版本) ;第一个是语法的,第二个是语义的,当类似的代码使用其他变量名时,这使得它更难注意到。
\end{itemize}
