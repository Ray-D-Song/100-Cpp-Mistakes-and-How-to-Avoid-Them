
这种错误主要影响可读性。但是,如果对变量的范围和含义的理解出现错误,则可能会对正确性和有效性产生不利影响。

变量命名是一项艰巨的任务。开发人员必须很好地命名实体以避免混淆和错误。随着开发人员对代码的理解不断发展,找到良好的命名实践可能需要多次迭代。存在许多可以声明变量的范围。这些范围包括语句、本地、类、命名空间、全局和文件。了解哪个范围对于正确使用适当的变量至关重要。

变量在写入其类型和名称时即被声明。其作用域从声明点开始,并限于声明它的文件、块或语句的末尾。通常,作用域以文件末尾、语句 作用域中的右括号或分号结束。当后面的内部作用域定义另一个具有相同名称但不一定相同类型的变量时,变量将被遮蔽。内部作用域内的代码无法在外部作用域中看到同名变量。在某些情况下,可以使用作用域解析 operator:: 来访问被遮蔽的外部作用域变量。

程序规模大于几个函数,代码中变量的名称通常相同。由于这些函数不共享作用域,因此不会发生遮蔽,因为没有作用域重叠。但是,当内部作用域声明与外部作用域中名称相同的变量时,程序员必须在赋值或计算中使用正确的变量。

\mySamllsection{问题}

当作用域定义同名变量时,就会发生遮蔽。作用域越广,越有可能发生遮蔽。全局变量对于编译单元中的所有函数都是可见的,从声明点到源文件末尾。一般不鼓励使用全局变量,但人们却常常这么做。过度使用全局变量首先会损害正确性和可读性。使用全局变量时,必须谨慎命名。只有当这些短名称是众所周知的概念(如 PI 或 E)时,为全局变量选择简单、简短的名称才是合理的。对于鲜为人知的值,应使用命名方案(如用下划线分隔的全大写字母(或类似方案))来清楚地区分全局变量和非全局变量。最好使用命名空间来防止容易出现的遮蔽错误。

通常,需要更好地实践清晰的命名。例如,假设有一个名为 I 的全局变量;只有读过它的定义(以及有用的注释)后才能理解它的含义。声明或定义离使用它的代码越远,变量的名称就应该越长、越有表现力。大多数公司或项目都有一些指导方针;请严格遵循这些指导方针——这关乎可读性和合作性。

在许多情况下,短变量名会在多个作用域中使用。这种情况使得理 解正在使用哪个变量变得更加困难。然而,短变量名很常见且方便 。知道何时使用它们需要了解内容和意图。

当使用变量名时出现错误时,就会发生内容类型隐藏错误;所用变量 的作用域为内部,而预期变量位于外部作用域。清单 9.1 中的代码演示了全局变量的名称与实例变量相同。仅出于说明目的,sum 函数将错误地使用变量作用域。

sum 函数错误地使用了实例变量两次,认为它使用了内部和外部作用域的变量。应该使用作用域解析运算符来消除变量名称的歧义。这是对内容的错误遮蔽,是一种语法错误——可能只是打字错误。这些问题可能会引起惊愕,但通过正确命名变量相对容易解决。

\filename{清单9.1 不同作用域中相同的变量名}

\begin{cpp}
double rad = 1.0; // 1
class Circle {
private:
  double rad; // 2
public:
  Circle(double rad) : rad(rad) {} // 3
  void setRadius(double rad) { // 3
    if (rad < 0)
      throw std::invalid_argument("negative radius");
    rad = rad;
  }
  double sum() const {
    return rad + rad; // 4
  }
  double getRadius() { return rad; } // 5
};

int main() {
  Circle c(3);
  std::cout << "radius is " << c.getRadius() << '\n';
  std::cout << "enlarged radius is " << c.sum() << '\n';
  return 0;
}
\end{cpp}

{\footnotesize
注释1:名称很短的全局变量,含义不明确

注释2:名称很短的实例变量

注释3:与实例变量同名的参数,需要消除歧义

注释4:两个变量同名,但使用最局部的变量,掩盖了全局版本

注释5:使用实例变量 ,它是作用域内唯一的变量
}

目的是将一个随机值数组传递给 SearchAnalyzer,这将确定数组中出现多少个值。由于此代码必须修改数组才能进行高效搜索,因此它通过排序克隆和修改值。其他 Analyzer 派生类不需要克隆数据;因此,共享数据的思维模式已经建立。调用 analyze 方法时,预期结果是一行文本,解释在数据中找到多少个随机搜索值。学生惊讶地发现没有找到任何值。下一步是给教授发电子邮件寻求帮助!

\filename{清单9.2 不同的变量名在其含义上混淆}

\begin{cpp}
class Analyzer{
protected:
  int* cloneValues(int* a, int size){
    int* arr = new int[size];
    for (int i = 0; i < size;i++)
    arr[i] = a[i];
    return arr;
  }

  int* array; // 1
  int size;
public:
  Analyzer(int*values, int size) : array(values), size(size) {}
  virtual std::string analyze() = 0;
  virtual ~Analyzer() { delete[] array; }
};

class SearchAnalyzer : public Analyzer{
public:
  SearchAnalyzer(int* values, int size) : Analyzer(cloneValues(values,
      size), size) { // 2
    selection_sort(values, size); // 3
  }
  std::string analyze() {
    int count = 0;
    for (int i = 0; i < 100; ++i)
      if (binary_search(array, rand() % SIZE, size)) // 4
        ++count;
    std::stringstream ss;
    ss << "There were "<< count << " random values found.";
    return ss.str();
  }
};

int main() {
  int* numbers = createArray(SIZE);
  SearchAnalyzer searcher(numbers, SIZE);
  std::cout << searcher.analyze() << '\n';
  return 0;
}
\end{cpp}

{\footnotesize
注释1:指向数组的受保护指针

注释2:对 cloneValues 方法的调用

注释3:错误地将 values 变量用于排序方法

注释4:搜索数组变量;它未排序!
}

这个问题很简单,但做起来却很容易。参数值被理解为要搜索的随机值列表;但是,这个列表必须被克隆和排序,从而创建一个代表预期数据的新变量。人们可以专注于输入变量,而忘记那些不显眼的克隆值;当其他代码没有单独的数据时,尤其如此。

修复很简单:在对 selection\_sort 的调用中将values 更改为 array,一切即可按预期工作。

不完全是遮蔽,当对相同内容使用相似的变量名时,就会发生一种常见错误。比实际遮蔽更有害的是,清单 9.2 中的代码是故意遮蔽,其中同名变量不会导致问题;因此,修复内容遮蔽的简单方法并不适用。相反,两个具有相似含义的变量是错误的——语义错误。该代码取自提交了无法完成的项目的学生。

\mySamllsection{分析}

全局变量经常使用,但很少以一种容易与类或局部变量区分开来的方式进行限定。清单 9.1 中的全局变量与代码中的类和参数变量同名。在本例中,很容易看出 setRadius 参数 rad 应该初始化同名的实例变量。然而,更广泛或更晦涩的代码会极大地掩盖明显的含义和用法。开发人员倾向于在类或函数中为全局变量提供简短的名称,因为它们当时看起来很“明显”。只要充分说明变量的含义和目的,它的使用就很简单了。

考虑 setRadius 方法:代码需要引用全局 rad 变量,但这是什么意思,它在哪里?如果类和方法没有这样的名称,则会进行搜索以查找外部范围变量;在这种情况下,将使用全局变量。这种方法在时间和理 解方面都很昂贵。

当 mutator 参数与实例变量同名时,this 关键字必须区分实例变量和参数变量。如果没有这种区别,参数将被赋值给自身,如清单 9.1 所示。有些编译器不会对这种情况发出警告,而是愉快地编译并生成不 符合预期的代码。我教学生使用这种模式作为经验法则。与它正在初始化的实例变量同名的参数是重要的文档;使用 this 指针对于消除歧义至关重要。如果使用这种方法,请养成习惯;否则,偶尔可能会无法使用 this 指针,并会导致令人讨厌的调试会话。

\mySamllsection{解决}

为了便于阅读,读者必须理解初始化列表每个部分的含义。括号外的变量是实例变量(清单 9.3 中的 this->radius 部分),括号内的变量是参数(本例中的 double radius)。效率略有提高,因为开发人员可以复制名称,而无需在短期记忆中保留单个字母或缩写。

\filename{清单9.3 由名称空间区分的相同变量名}

\begin{cpp}
namespace global { // 1
  double radius = 1.0;
};
class Circle {
private:
  double radius;
public:
  Circle(double radius) : radius(radius) {}
  void setRadius(double radius) {
    if (radius < 0)
      throw std::invalid_argument("negative radius");
    this->radius = radius; // 2
  }
  double sum() const {
    return radius + global::radius; // 3
  }
  double getRadius() const { return radius; }
};

int main() {
  Circle c(3);
  std::cout << "radius is " << c.getRadius() << '\n';
  std::cout << "enlarged radius is " << c.sum() << '\n';
  return 0;
}
\end{cpp}

{\footnotesize
注释1:使用命名空间来确保不会发生阴影

注释2:记住要养成使用此模式的习惯,否则就不要使用它

注释3:变量名的明确使用
}

隐藏名称是一种记录用于初始化或分配实例变量的参数的有效方法;构造函数或赋值器的参数名称和实例变量名称相同。一些开发人员建议将参数编码为单个字符(实例变量的第一个字符)或名称的缩写版本,以合理地传达意图。更好的建议是在实例变量前面或后面加上一个符号(例如下划线),以表明该变量是私有的(只有实例变量可以是私有的)。

全局变量的命名方式应使其明显是全局变量。这种方法可防止在需要使用其他变量时出现无意使用或混淆。有两种方法有助于确保这一点清晰。首先,考虑一种标准化的全局变量命名方式。例如,在每个变量前加上 GLOB\_ 或类似的标记。另一种方法是将 \_g 或 \_global 作为后缀添加到变量名中。记住变量距离其使用位置越远,其名称就应该越长、越具有描述性,以传达其意图和目的;这两个选项很好地解决了这个建议。其次,考虑使用命名空间来包含所有全局变量。这种方法使理解全局变量的使用情况变得容易得多。命名空间包含可确保全局变量不会分散在源代码中,并且以明显的模式命名。此外,在 enum 或类中包含全局变量提供了具有相同好处的替代方案。此代码片段演示了在计算连续复利时使用命名空间:

\begin{cpp}
namespace Constants {
  const double e = 2.7182828283;
}
...
amount = principal * std::pow(Constants::e, rate*time);
\end{cpp}

\mySamllsection{建议}

\begin{itemize}
\item
名称与用途的距离越远,名称就应该越长、越具有描述性,使用方式也应该越具体(例如命名空间)。

\item
尽可能避免使用全局变量 ,以防止出现设计和阴影问题。

\item
请记住,重复变量名称可能会导致难以理解代码的意图及其正确用法。

\item
注意内容的阴影(简单版本)和意图的阴影(更复杂的版本) ;第一个是句法的,第二个是语义的,这使得它更难被注意到,主要是当类似的代码使用其他变量名时。
\end{itemize}
