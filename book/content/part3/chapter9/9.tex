这个错误主要影响性能,对可读性或有效性没有影响。算术运算通常以代数形式实现。这种默认方法在许多情况下是合理的,我们无需担心它对性能的影响。但在其他情况下,可以从阻止临时对象的创建和销毁中受益。算术表达式是一个非直观形式,可以最小化临时对象并提高执行速度的领域。

\mySamllsection{问题}

考虑一下此代码,其中创建了一些 Complex 对象,然后将其相加以初始化另一个对象。这种方法是使用类对象创建算术表达式的常见且直观的方法。代码中有5个构造函数调用:3个用于前三个对象,即c1、c2 和 c3,这些对象无法最小化,两个是在求和表达式中创建的临时对象。消除这些临时对象将提高性能。许多现代编译器都会尝试消除它们,但可以用来理解这个问题。

\filename{清单9.22 在表达式求值中具有过多临时项的类}

\begin{cpp}
class Complex {
private:
  double real;
  double imag;
public:
  Complex(double real=0, double imag=0) : real(real), imag(imag) {}
  double getReal() const { return real; }
  double getImag() const { return imag; }
};

const Complex operator+(const Complex& lhs, const Complex& rhs) {
  return Complex(lhs.getReal()+rhs.getReal(), lhs.getImag()+rhs.getImag());
}

int main() {
  Complex c1(2, 2); // 1
  Complex c2(0, -1);
  Complex c3(-2.2, 4.2);
  Complex c4 = c1 + c2 + c3; // 2
}
\end{cpp}

{\footnotesize
注释1:创建每个对象都需要调用一次构造函数

注释2:计算需要两个临时对象
}

\mySamllsection{分析}

前三个构造函数调用必不可少,必须存在才能创建 Complex 对象。我们无法在没有构造函数调用的情况下,神奇地获得对象,但求和会创建两个临时对象。从性能角度考虑,只要创建了临时对象,就有可能消除它的创建。请记住,每个非平凡数据类型的构造函数在某个时刻,都有一个关联的析构函数调用,这会使使用临时对象的成本加倍。大多数编译器可以优化简单对象的析构函数调用。

第一个临时变量保存 c1 和 c2 的总和,第二个临时变量保存第一个临时变量和 c3 的总和的结果。然后,调用默认的 operator= 来使用第二个临时变量中保存的值初始化 c4 对象。问题是,是否可以减少构造函数调用的次数,同时仍提供相同的功能。

\mySamllsection{解决}

可以通过直接使用复合赋值操作符或与独立 operator+ 结合使用,来减少临时变量的数量。如果同时提供独立版本和复合版本,则应根据复合赋值形式实现独立版本。此外,消除了代码重复的可能性,将基本函数单独封装在复合赋值版本中。独立版本调用此函数,但不提供基本逻辑。以下代码通过根据复合赋值版本定义独立操作,并在计算中继续使用独立操作符形式展示了这些改进。

\filename{清单9.23 表达式求值中具有最小化的临时类}

\begin{cpp}
class Complex {
private:
  double real;
  double imag;
public:
  Complex(double real=0, double imag=0) : real(real), imag(imag) {}
  Complex& operator+=(const Complex&);
  double getReal() const { return real; }
  double getImag() const { return imag; }
};

Complex& Complex::operator+=(const Complex& o) { // 1
  real += o.real;
  imag += o.imag;
  return *this;
}

const Complex operator+(const Complex& lhs, const Complex& rhs) { // 2
  return Complex(lhs) += rhs;
}

int main() {
  Complex c1(2, 2);
  Complex c2(0, -1);
  Complex c3(-2.2, 4.2);
  Complex c4 = c1 + c2 + c3;
}
\end{cpp}

{\footnotesize
注释1:操作符逻辑包含在复合赋值版本中

注释2:独立操作符按照复合赋值版本实现
}

当独立操作符以复合赋值版本实现时,就不再需要临时对象了。清单9.23 中的实现进行了三次构造函数调用,这些调用是构造 c1、c2 和c3 对象所必需的。求和操作符的求值不会导致创建任何临时对象,从而提高了性能,同时又不损害表达式求值的有效性。

清单 9.24 中的代码展示了实现操作符调用序列的另一种方法。独立版本单独会为每次对象调用创建一个临时对象,而此形式会就地更新对象,而不会创建临时对象。Complex 类的定义与清单 9.23 中的相同,只是 main 函数已更改。

\filename{清单9.24 多操作符调用的另一种形式}

\begin{cpp}
int main() {
  Complex c1(2, 2);
  Complex c2(0, -1);
  Complex c3(-2.2, 4.2);
  Complex c4(c1); // 1
  c4 += c2; // 2
  c4 += c3;
}
\end{cpp}

{\footnotesize
注释1:调用复制构造函数

注释2:不需要临时变量的顺序求和
}

这种方法的优势在于复合赋值形式效率更高,它们会就地更新值。之前的独立版本表明它必须返回一个新对象,并且这个新对象需要通过构造函数调用创建临时对象。

\mySamllsection{建议}

\begin{itemize}
\item
为了防止创建临时对象,请考虑实现算术操作符的复合赋值版本,并从独立版本调用这些版本。

\item
可以有效地按顺序多次使用该操作符,而不会创建临时对象。
\end{itemize}













