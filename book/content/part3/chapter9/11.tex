这个错误主要针对可读性和有效性。类由状态(实例变量)和行为( 方法)组成。声明方法有两种选择:实例方法或类方法。实例方法可以访问所有实例和类变量(使用 static 关键字声明)。类方法只能访问类变量,这意味着这些方法不能访问任何对象值。可能会出现一个问题:为什么任何方法都不想访问实例变量,以及这样做有什么好处。

当我的学生发现所教的内容不一致或无法立即理解时,我会鼓励他们提出这样的问题,而不是盲目地接受我的教学。这种错误是提出问题并寻求澄清的机会之一。

类方法的一个众所周知的优点是,执行代码不需要类实例。这种情况允许定义与类相关的辅助方法,但可以像独立函数一样执行。例如,考虑 Java 提供的 Math 类。创建该类的实例,用特定值初始化它,然后调用 sqrt 方法会非常尴尬。直接调用 sqrt 方法,传递特定值作为参数要好得多。C++ 提供免费函数,易于使用且清晰易懂;此外,它通过在类中声明类方法来实现与 Java 类似的功能。

\mySamllsection{问题}

假设我们正在编写一个 Rational 类,其中某个数学团队想要使用精确值,而不是浮点值提供的近似值。例如,有理数 10/3 是精确的,但无法用十进制或二进制精确表示。清单 9.27 中的代码显示了实现Rational 子集的第一次尝试。每次对象的状态发生变化时都会使用reduce 和 gcd 方法。该类的每个实例都需要频繁调用该方法。

\filename{清单9.27 对理性类的第一次尝试}

\begin{cpp}
class Rational {
private:
  int num;
  int den;
public:
  Rational(int num, int den = 1) : num(num), den(den)
    { reduce(); } // 1
  friend std::ostream& operator<<(std::ostream&, const Rational&);
  Rational& operator*=(const Rational&);
  int gcd(int a, int b) {
    if (b == 0)
      return a;
    return gcd(b, a % b);
  }
  void reduce() {
    int div = gcd(num, den);
    num /= div;
    den /= div;
  }
};
std::ostream& operator<<(std::ostream& out, const Rational& r) {
  out << r.num << '/' << r.den;
  return out;
}

Rational& Rational::operator*=(const Rational& o) {
  num *= o.num;
  den *= o.den;
  reduce(); // 2
  return *this;
}

int main() {
  Rational r1(3, 9);
  std::cout << r1 << '\n';
  return 0;
}
\end{cpp}

{\footnotesize
注释1:reduce 和 gcd 在构造函数中被调用

注释2:reduce 和 gcd 在 operator*= 中被调用
}

\mySamllsection{分析}

reduce 和 gcd 方法是公共实例方法。从正确性角度看,这种实现没有任何问题,但这一事实会影响可读性。代码暗示这些方法是necessarily 实例方法。也就是说,只有当存在实例时才应调用这些方法,并且每个方法都需要访问至少一个实例变量。reduce 方法按编写方式访问 num 和 den,但 gcd 方法仅对其参数进行操作。

函数式编程定义了一个术语,用于描述 gcd 的实现方式;它是一个纯函数。在这种情况下,pure 表示该方法不会访问或影响其范围之外的任何状态;换句话说,它没有副作用。它完全根据其参数传递给它的数据来计算其结果 - 相同的输入始终产生相同的输出。由于 gcd 是纯函数,因此没有理由坚持它需要一个对象来操作。因此,通过使用static 关键字标记此方法,该方法可以作为类方法的候选。

reduce 方法使用实例变量,因此它必须是实例方法。或者它确实是实例方法?可以轻松重写该方法以消除该要求并使其成为纯方法。如果可以将其变为纯方法,那么应该这样做吗?在这种情况下,我们将假设答案是肯定的。

\mySamllsection{解决}

清单 9.28 中的代码重新编写了对 Rational 类的第一次尝试。它立即将 gcd 标记为静态方法。此标记现在表明该方法在实例之间共享,并且不影响任何实例变量。通过添加 static 关键字,可以更清楚地描述其含义和范围。gcd 方法是纯方法(它不会改变自身之外的状态) 。不幸的是,reduce 方法不能是纯方法,因为它会修改实例变量。由于这两种方法现在都是类方法,因此它们在语义上与以前不同,并传达了不同的含义。通常最好将它们设为私有,因为它们仅供在类中使用。

\filename{清单9.28 Rational重新设计了两个类方法}

\begin{cpp}
class Rational {
  private:
  int num;
  int den;
  static int gcd(int a, int b) { // 1
    if (b == 0)
      return a;
    return gcd(b, a % b);
  }
  static void reduce(int& num, int& den) { // 1
    int div = gcd(num, den);
    num /= div;
    den /= div;
  }
public:
  Rational(int num, int den = 1) : num(num), den(den) {
    reduce(this->num, this->den); }
  friend std::ostream& operator<<(std::ostream&, const Rational&);
  Rational& operator*=(const Rational&);
};
std::ostream& operator<<(std::ostream& out, const Rational& r) {
  out << r.num << '/' << r.den;
  return out;
}
Rational& Rational::operator*=(const Rational& o) {
  num *= o.num;
  den *= o.den;
  reduce(num, den);
  return *this;
}

int main() {
  Rational r1(3, 9);
  std::cout << r1 << '\n';
  return 0;
}
\end{cpp}

{\footnotesize
注释1:现在,一个私有的纯方法
}

为了看起来更简洁,两个方法都设为私有;但是,这真的是考虑此语义特征的最佳方法吗?这个疑问让我们有时间思考这些变化。在盯着reduce 方法时,可能会质疑它是否正确传达了正确的行为。毕竟,reduce 旨在与实例变量交互。传递引用当然允许它成为纯方法,但现在,它几乎与其实际意图相冲突,并且引用对于影响两个变量是必要的。最好将 reduce 返回到实例方法,而不是使用有点难以阅读的引用。引用非常方便,但在这种情况下,它们往往会掩盖方法的含义。将实现返回到实例方法更有意义。清单 9.29 表明,按照这种推理进行 的修改已经取得了成果。

\filename{清单9.29 Rational再次进行了修改,以更好地指示一个类方法}

\begin{cpp}
class Rational {
private:
  int num;
  int den;
  void reduce(); // 1
public:
  Rational(int num, int den = 1) : num(num), den(den) { reduce(); }
  friend std::ostream& operator<<(std::ostream&, const Rational&);
  Rational& operator*=(const Rational&);
};
std::ostream& operator<<(std::ostream& out, const Rational& r) {
  out << r.num << '/' << r.den;
  return out;
}
Rational& Rational::operator*=(const Rational& o) {
  num *= o.num;
  den *= o.den;
  reduce();
  return *this;
}
static int gcd(int a, int b) { // 2
  if (b == 0)
    return a;
  return gcd(b, a % b);
}
void Rational::reduce() {
  int div = gcd(num, den);
  num /= div;
  den /= div;
}

int main() {
  Rational r1(3, 9);
  std::cout << r1 << '\n';
  return 0;
}
\end{cpp}

{\footnotesize
注释1:私有,因为只有实例可以使用此方法

注释2:公开,因为客户端可能希望使用该功能
}

清单 9.29 中考虑了另一个方面。数学团队表示,他们希望 gcd可供外部使用。static 关键字的主要特征之一是 sharing 的概念。使用类方法,类可以与外界共享其某些行为或知识。

\mySamllsection{建议}

\begin{itemize}
\item
仔细考虑是否可以将方法设为纯方法;如果可以,请将其标记为static,这样可以更好地传达其意图,因为它不会影响任何实例 变量。

\item
考虑静态方法是否应设为 public 或 private;public 版本可以与客户端共享以供一般使用,而 private 版本应仅供类使用。

\item
如果派生类中需要 private 方法,请考虑将它们设为protected,这样继承类就可以直接访问它们,但可以防止它们被客户端代码使用。
\end{itemize}








