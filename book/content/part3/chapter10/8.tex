此错误主要影响正确性和可读性,从而影响性能。正确性的问题在于代码可能无法按预期工作;可读性的问题在于代码看似直观但实际上并非如此。

异常提供了一种通过备用控制路径报告错误的简洁方法,从而大大降低了代码复杂性。记录预期的异常是通过读取所有相关的捕获块来完成的;但是,这种方法还有待改进。一个好主意是通过在异常规范列 表中指定它们来记录函数抛出的异常。看一眼规范列表就应该知道全部情况。

\mySamllsection{问题}

当开发人员编写所有受影响的代码时,异常处理会变得简单。使用库或其他模块时,查看和理解各种困难的能力会大大降低。清单 10.15 中的代码计算向量中一组值的标准偏差。开发人员小心地确保太少的值会引发异常,以防止在 std\_dev 函数中错误计算结果。作为一名细心的开发人员,他们添加了异常规范来记录如果给出的值太少,函数可能会引发 std::invalid\_argument。开发人员确保调用代码可以从异常中恢复,因为程序终止是有问题的。

在正常情况下,此代码按预期工作,但如果没有给出任何值,情况就会变得很糟糕。如果传递给 std\_dev 函数的值太少,预计会抛出异常。然而,开发人员忽略了 arith\_mean 函数是在向量大小测试之前执行的。被调用的函数检测到值太少的问题,并抛出一个异常。然而,尽管开发人员很小心,程序还是突然终止了。

\filename{清单10.15 列出可以抛出的异常}

\begin{cpp}
double arith_mean(const std::vector<double>& values) {
  if (values.size() == 0)
    throw std::domain_error("missing values"); // 1
  double sum = 0.0;
  for (int i = 0; i < values.size(); ++i)
    sum += values[i];
  return sum/values.size();
}
double std_dev(const std::vector<double>& values)
                throw (std::invalid_argument) { // 2
  double mean = arith_mean(values);
  if (values.size() < 2)
    throw std::invalid_argument("too few values"); // 3
  double sum = 0.0;
  for (int i = 0; i < values.size(); ++i)
    sum += std::pow(values[i] - mean, 2);
  return std::sqrt(sum / (values.size() - 1));
}

int main() {
  std::vector<double> values;
  values.push_back(3.14159);
  values.push_back(2.71828);
  try {
    std::cout << "standard deviation is " << std_dev(values) << '\n';
  } catch (...) {
    std::cout << "exception caught, but we did not crash\n";
  }
  return 0;
}
\end{cpp}

{\footnotesize
注释1:如果存在的元素太少,则抛出std::domain\_error

注释2:预测可能出现的std::invalid\_argument异常

注释3:如果存在的元素太少,抛出std::invalid\_argument异常
}

开发人员对异常条件下的行为感到惊讶,并且对为什么主函数的ca tch块无法拯救程序感到困惑。

\mySamllsection{分析}

开发人员试图通过指定异常来控制程序行为,这种做法虽然合理,但并不完善。除非您彻底理解代码,否则在规范列表中指定异常就是一种猜测。如果您猜对了,程序就会按预期运行;如果您猜错了 ,程序就会毫不留情地崩溃。

编译器必须付出一些努力来处理异常规范。在不使用异常规范的程序中,任何抛出的异常都会随着堆栈展开而渗透到代码层次结构中。一旦添加规范,行为就会以非直观的方式发生变化。编译器会添加代码来检查抛出的异常并将其与规范进行匹配。如果匹配,则异常会按预期传播;但是,如果检测到的异常不在规范中,则检测代码会抛出std::unexpected 异常,其默认行为是调用 terminate(后者会调用 abort)。此附加代码会影响性能。

如果开发人员希望将意外异常转换为更易于管理的异常,请编写一个抛出已知异常(标准异常或您自己制造的异常)的函数,并将该转换函数作为 set\_unexpected 函数的参数传递。简而言之,代码如下所示:

\begin{cpp}
class MyException {};
void transformException() { throw MyException(); }
set_unexpected(transformException);
\end{cpp}

如果使用这种方法,则应在某个 catch 块中捕获生成的 MyException 。这种方法的一个重大缺点是整个程序会将 unexpected 异常转换为MyException,这使得捕获其上下文非常困难。此解决方案是可行的,但其过于笼统是有害的。尝试记录异常会掩盖需要更好处理的问题。

\mySamllsection{解决}

由于不确定是否可以知道所有可能的异常并将其列在异常规范中,因此最彻底和最有帮助的方法是消除规范。消除它们可以消除与对异常的直观理解背道而驰的意外(和不想要的)行为的可能性。库中许多调用的代码无法充分分析潜在异常,因此使用规范列表可能会产生意外行为。

异常规范是一个实验,它已从现代 C++ 中删除。这一事实表明专家们不信任它们的使用。我们可以将此视为一个明确的暗示,采取相应行动并删除所有异常规范。

以下代码显示了此调整。作为替代方案,注释掉的函数头指定了两种可能的异常;但是,在实际系统中可能无法知道可能异常的完整列表。除非管理层或技术领导坚持,否则请避免采用这种方式。

\filename{清单10.16消除异常规范}

\begin{cpp}
double arith_mean(const std::vector<double>& values) {
  if (values.size() == 0)
    throw std::domain_error("missing values");
  double sum = 0.0;
  for (int i = 0; i < values.size(); ++i)
    sum += values[i];
  return sum/values.size();
}
// double std_dev(const std::vector<double>& values) throw (std::invalid_argument, std::domain_error) { // 1
double std_dev(const std::vector<double>& values) { // 2
  if (values.size() < 2) // 3
    throw std::invalid_argument("too few values");
  double mean = arith_mean(values);
  double sum = 0.0;
  for (int i = 0; i < values.size(); ++i)
    sum += std::pow(values[i] - mean, 2);
  return std::sqrt(sum / (values.size() - 1));
}

int main() {
  std::vector<double> values;
  values.push_back(3.14159);
  values.push_back(2.71828);
  try {
    std::cout << "standard deviation is " << std_dev(values) << '\n';
  } catch (...) {
    std::cout << "exception caught, but we did not crash\n";
  }
  return 0;
}
\end{cpp}

{\footnotesize
注释1:可选:添加所有发现的异常(这个列表完整吗?)

注释2:完全消除异常规范

注释3:确保在调用任何其他代码之前进行本地验证
}

虽然异常规范列表是一个好主意(理论上),但意外行为、额外的实施成本以及无法兑现其承诺意味着最好的方法是消除它们。

\mySamllsection{建议}

\begin{itemize}
\item
如果可能的话,消除异常规范列表。

\item
了解未包含在异常列表中的抛出异常的意外行为。

\item
记住使用异常规范的性能影响。
\end{itemize}






