这个错误关注的是正确性,会影响有效性,但这是确保正确操作必须付出的代价。异常是发现和处理错误的一种宝贵手段。有人认为,当参数值无效时,构造函数应该快速抛出。部分构造的对象对正确性来说是一个真实的危险;因此,在析构函数中遵循这个建议是可行的。毕竟,如果析构函数检测到错误,有什么比抛出异常更好的方法来发现它呢?这种直觉值得称赞,但实现却并非如此。

\mySamllsection{问题}

假设一个热切的开发者已经学会了一些关于抛出异常的要点,这些知识被其扩展到了析构函数。清单 10.3 中的代码展示了开发者尝试应用这些知识,来为他们的代码生成一个健壮,且一致的错误处理策略的结果。假设在构造之后的某个时间会添加一个Paragraph 对象。析构时,该类会删除一个动态的Paragraph 对象。如果在没有有效 Paragraph 的情况下销毁了 Page ,则将其视为错误。这种情况似乎是抛出异常的理想时机。

\filename{清单10.3 异常的过度使用}

\begin{cpp}
struct Paragraph {};

class Page {
private:
  std::string title;
  Paragraph* pgph;
public:
  Page(const std::string& title) : title(title), pgph(0) {}
  ~Page() {
    if (pgph == 0)
      throw std::string("destructor"); // 1
    delete pgph;
  }
};

int main() {
  try {
    Page p("Catching Up"); // 2
  } catch (const std::string& ex) {
    std::cout << "Exception caught: " << ex << '\n';
  }
  try {
    Page p("Trouble Ahead"); // 3
    throw std::string("try block");
  } catch (const std::string& ex) {
    std::cout << "Exception caught: " << ex << '\n';
  }
  return 0;
}
\end{cpp}

{\footnotesize
注释1:从析构函数中抛出异常

注释2:对象正常销毁

注释3:对象错误销毁
}

代码产生以下输出(不同的编译器和系统可能看起来有些不同):

\begin{shell}
Exception caught: destructor terminate called after throwing an instance of 'std::__cxx11::basic_string<char, std::char_traits<char>, std::allocator<char> >'
Aborted
\end{shell}

程序在 "Trouble Ahead" try 块中毫不客气地终止。第一个 try 块成功了——或者更准确地说,没有失败——但它的执行是一个幸运的意外,允许一些险恶的东西潜伏着。其他编译器和系统可能会有不同的行为。

\mySamllsection{分析}

第一个 try 块之所以能正常工作,是因为没有正在进行的异常。第一个 catch 块可以捕获异常并按预期进行处理。这种看似正确的操作具有误导性;当策略失败时,会让开发人员吃惊。

第二个 try 块是那个暂停的惊喜。Page 对象是在 try 块中创建的,但在正常退出块时销毁它之前会引发异常。以下 catch 块应该处理该异常;但必须先销毁 Page 对象。调用 Page 析构函数,并且它遇到错误。因此,它引发异常以表明,它缺少 Paragraph 对象。

当异常正在进行时,第二个异常抛出会导致调用 terminate 函数。std::terminate 函数调用的默认行为是,通过调用 std::abort 在该点中止执行而不展开堆栈。abort 函数确保调试器(例如,dbg)可以在终止点看到程序的确切状态。如果堆栈展开并且要执行其他清理函数,调试将缺少其大部分上下文。

出乎意料的是,这个程序应该以这种方式运行。开发者没有意识到抛出一个当另一个正在执行时,将引发异常。析构函数特别容易受到这种危险的影响,它们在两种情况下调用:正常终止和错误终止。当在错误条件下调用抛出的析构函数时,就会出现这种意外行为。

\mySamllsection{解决}

最好的方法是避免从析构函数中抛出异常;这可以完全避免问题,因为异常发生的概率大到足以成为问题。但某些情况下,析构函数中抛出异常。如果析构函数执行抛出的代码,则析构函数无法控制该行为,并可能成为终止调用的受害者。

对于可能(或实际)抛出的析构函数代码,必要的方法是将代码包装在 try 块中。清单 10.4 中的代码展示了这种方法;它捕获异常,并大概记录问题,但不会重新抛出当前异常或抛出另一个异常。通用的 catch 规范将捕获任何异常;如果特定处理需要更多特异性,请在通用 catch 之前添加。关键是任何异常都不应逃过析构函数。

\filename{清单10.4 析构函数中抛出异常,而不是从析构函数抛出异常}

\begin{cpp}
struct Paragraph {};

class Page {
private:
  std::string title;
  Paragraph* pgph;
public:
  Page(const std::string& title) : title(title), pgph(0) {}
  ~Page() {
    try {
      if (pgph == 0)
      throw std::string("destructor"); // 1
    delete pgph;
    } catch (...) { // 2
      std::cout << "ERROR: exception captured in destructor\n";
    }
  }
};

int main() {
  try {
    Page p("Catching Up"); // 3
  } catch (const std::string& ex) {
    std::cout << "Exception caught: " << ex << '\n';
  }
  try {
    Page p("Trouble Ahead");
    throw std::string("try block"); // 4
  } catch (const std::string& ex) {
    std::cout << "Exception caught: " << ex << '\n';
  }
  return 0;
}
\end{cpp}

{\footnotesize
注释1:仍在析构函数中抛出

注释2:捕获析构函数中的抛出,避免其离开函数体

注释3:对象的正常销毁

注释4:对象的错误销毁
}

\mySamllsection{建议}

\begin{itemize}
\item
不要从析构函数中抛出异常;异常很可能已经发生,这将导致程序立即终止。

\item
如果析构函数中的代码可以抛出异常,则将其包装在 try 块中以捕获它,并将其转换为日志(或类似的)消息;不要重新抛出异常。
\end{itemize}
























