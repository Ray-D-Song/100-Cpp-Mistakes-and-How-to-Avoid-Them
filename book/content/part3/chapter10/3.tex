这个错误关注的是正确性,它会影响有效性;然而,这是确保正确操作必须付出的代价。异常是发现和处理错误的一种宝贵手段。有人认为,当参数值无效时,构造函数应该快速抛出。部分构造的对象对正确性来说是一个微妙但真实的危险;因此,在析构函数中遵循这个建议是可行的。毕竟,如果析构函数检测到错误,有什么比抛出异常更 好的方法来发现它呢?这种直觉值得称赞;但它的实现却并非如此。

\mySamllsection{问题}

我们假设一个热切的程序员已经学会了一些关于抛出异常的要点。在他们的热切中,这些知识被扩展到了析构函数。清单 10.3 中的代码展示了程序员尝试应用这些知识来为他们的代码生成一个健壮且一致的错误处理策略的结果。假设在构造之后的某个时间会添加一个Paragraph 对象。在析构时,该类被设计为删除一个动态的Paragraph 对象。如果在没有有效 Paragraph 的情况下销毁了 Page ,则将其视为错误。这种情况似乎是抛出异常的理想时机。

\filename{清单10.3 异常的过度使用}

\begin{cpp}
struct Paragraph {};

class Page {
private:
  std::string title;
  Paragraph* pgph;
public:
  Page(const std::string& title) : title(title), pgph(0) {}
  ~Page() {
    if (pgph == 0)
      throw std::string("destructor"); // 1
    delete pgph;
  }
};
int main() {
  try {
    Page p("Catching Up"); // 2
  } catch (const std::string& ex) {
    std::cout << "Exception caught: " << ex << '\n';
  }
  try {
    Page p("Trouble Ahead"); // 3
    throw std::string("try block");
  } catch (const std::string& ex) {
    std::cout << "Exception caught: " << ex << '\n';
  }
  return 0;
}
\end{cpp}

{\footnotesize
注释1:从析构函数中抛出异常

注释2:对象正常销毁

注释3:对象错误销毁
}

代码产生以下输出(不同的编译器和系统可能看起来有些不同):

\begin{shell}
Exception caught: destructor terminate called after throwing an instance of 'std::__cxx11::basic_string<char, std::char_traits<char>, std::allocator<char> >'
Aborted
\end{shell}

程序在 "Trouble Ahead" try 块中被毫不客气地终止。第一个 try 块成功了——或者更准确地说,没有失败——但它的执行是一个幸运的意外,允许一些险恶的东西潜伏着。其他编译器和系统可能会有不同的行为。

\mySamllsection{分析}

第一个 try 块之所以能正常工作,是因为没有正在进行的异常。第一个 catch 块可以捕获异常并按预期进行处理。这种看似正确的操作具有误导性;当策略失败时,它会让开发人员大吃一惊。

第二个 try 块是那个暂停的惊喜。Page 对象是在 try 块中创建的,但在正常退出块时销毁它之前会引发异常。以下 catch 块应该处理该异常;但是,必须先销毁 Page 对象。调用 Page 析构函数,并且它遇到错误。因此,它引发异常以表明它缺少其 Paragraph 对象。

当异常正在进行时,第二个异常被抛出会导致调用 terminate 函数。std::terminate 函数调用的默认行为是通过调用 std::abort 在该点中止执行而不展开堆栈。中止函数的行为是正确的,也是预期的。abort 函数确保调试器(例如,dbg)可以在终止点看到程序的确切状态。如果堆栈展开并且要执行其他清理函数,调试将缺少其大部分上下文。

出乎意料的是,这个程序应该以这种方式运行。程序员没有意识到抛出一个当另一个正在执行时,将引发异常。析构函数特别容易受到这种危险的影响,因为它们在两种情况下被调用:正常终止和错误终止。当在错误条件下调用抛出的析构函数时,就会出现这种意外行为。

\mySamllsection{解决}

最好的方法是避免从析构函数中抛出异常;这可以完全避免问题,因为异常发生的概率大到足以成为问题。但是,在某些情况下,在析构函数中抛出异常是必要的。关键是抛出 in 析构函数和抛出 from 析构函数之间的区别。如果析构函数执行抛出的代码,则析构函数无法控制该行为,并可能成为终止调用的受害者。

对于可能(或实际)抛出的析构函数代码,必要的方法是将代码包装在 try 块中。清单 10.4 中的代码展示了这种方法;它捕获任何异常,并大概记录问题,但不会重新抛出当前异常或抛出另一个异常。通用的 catch-all 规范将捕获任何异常;如果特定处理需要更多特异性,请在通用 catch 之前添加它。关键是任何异常都不应逃过析构函数。

\filename{清单10.4 在析构函数中抛出异常,而不是从析构函数抛出异常态}

\begin{cpp}
struct Paragraph {};

class Page {
private:
  std::string title;
  Paragraph* pgph;
public:
  Page(const std::string& title) : title(title), pgph(0) {}
  ~Page() {
    try {
      if (pgph == 0)
      throw std::string("destructor"); // 1
    delete pgph;
    } catch (...) { // 2
      std::cout << "ERROR: exception captured in destructor\n";
    }
  }
};
int main() {
  try {
    Page p("Catching Up"); // 3
  } catch (const std::string& ex) {
    std::cout << "Exception caught: " << ex << '\n';
  }
  try {
    Page p("Trouble Ahead");
    throw std::string("try block"); // 4
  } catch (const std::string& ex) {
    std::cout << "Exception caught: " << ex << '\n';
  }
  return 0;
}
\end{cpp}

{\footnotesize
注释1:仍在析构函数中抛出

注释2:捕获析构函数中的任何抛出,防止其离开函数体

注释3:对象的正常销毁

注释4:对象的错误销毁
}

\mySamllsection{建议}

\begin{itemize}
\item
不要从析构函数中抛出异常;异常很可能已经发生,这将导致程序立即终止。

\item
如果析构函数中的代码可以抛出异常,则将其包装在 try 块中以捕获它并将其转换为日志(或类似的)消息;不要重新抛出异常。
\end{itemize}
























