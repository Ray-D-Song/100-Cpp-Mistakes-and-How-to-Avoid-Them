这个错误影响正确性,同时对可读性和有效性产生轻微的负面影响。比较数字似乎简单明了,但浮点值应该是近似值,而不是精确值。

\mySamllsection{问题}

数学直觉可能会告诉我们,除数除以一个数,结果乘以除数应该得到原始数。代数强调了这一真理,大多数人毫不犹豫地接受了它。当开始用浮点数编程时,这种直觉可能会让我们走上歧途。

考虑清单 12.15 中的代码,其中一个简单的除法问题产生了意想不到的结果。下面的比较使用了这种直觉,并期望结果相等。结果非常非常接近但不精确,开发人员熟悉使用相等操作符来确定值的等价性,这种熟悉会导致他们误用该操作符。

开发人员原本预计比较会返回 true 结果,但却惊讶地发现结果竟然是 false。直觉告诉我,比较应该是 true。

\filename{清单12.15 使用相等操作符进行比较}

\begin{cpp}
int main() {
  double amount = 100.0 / 3;
  std::cout << (amount == 33.3333333333 ? "true" : "false") << '\n'; // 1
  return 0;
}
\end{cpp}

{\footnotesize
注释1:预期结果是正确的,但事实并非如此
}

\mySamllsection{分析}

计算机是功能有限的机器,只有少量的位分配来表示浮点值。浮点值是一组最小实数的近似值。然而,从代数到编程,却遇到了意想不到的惊喜。必须记住的是,浮点数无法表达大多数实数。

如果开发者将除法结果乘以 3,则与值 100 的比较会成功,但并不正确。除法结果的实际值不是以数字 3 结尾,但最后一位数字可能是 4。IEEE 754 标准定义了此行为;所有符合标准的编译器都会表示尊重。在我的计算机上,将输出精度设置为 18 位,结尾数字是 ... 57 ;更多数字将以出乎意料的形式表示。这些非 3 数字不是我们所期望的,也不是我们的直觉告诉我们的。如果有人提出反对意见,说增加更多数字可以解决这个问题,那么问题就变成了到底需要多少位数字?没有数字可以精确地表示该值。关键的想法是浮点数是近似值,很少是正确的,但对于大多数应用来说足够接近。这个问题不仅限于计算机;三分之一的精确值无法用十进制(基数为 10)数来表示——有人喜欢十二进制(基数为 12)数制吗?

\mySamllsection{解决}

由于浮点数是近似值,不能使用 operator== 进行比较。需要进行比较以确定两个近似值之间的接近程度。如果它们之间的差异足够小,则应考虑等效。清单 12.16 中的代码演示了 delta-epsilon 比较方法。我将其作为“足够接近”函数来讲述。

Delta是数学家用来表示值之间的差异的希腊字母——可以将其视为从一个值中减去另一个值。Epsilon 是一个希腊字母,表示非常小的值(字符串中,代表空字符串)。其目的是找出两个值之间的差异,看看它是否小于一个微小的数字。如果是这样,这两个值就“足够接近”,即使它们的最后数字可能不同,也可视为等价。

必须从较大的值中减去较小的值才能产生有意义的差值。由于不知道哪个值更大,可计算差值的绝对值,确保结果为正数或零。Epsilon 通常定义为 $10^{-14}$,大致是可靠地确定 64 位双精度值的最小有意义值。但请确保系统可以有效地使用,较小的位大小将需要较大的 epsilon 值。此外,如果可以降低所需的精度,请使用较大的epsilon 值。以下代码使用精度 $10^{-10}$,因为对于这个问题来说,这已经足够接近了。

\filename{清单12.16 使用delta-epsilon方法进行比较}

\begin{cpp}
bool close_enough(double value, double target, double epsilon=1e-14) { // 1
  return fabs(value-target) < epsilon; // 2
}

int main() {
  double amount = 100.0 / 3;
  std::cout << (close_enough(amount, 33.3333333333, 1e-10) ? "true" : "false") << '\n';
  return 0;
}
\end{cpp}

{\footnotesize
注释1:默认为一个合理的小值

注释2:确定差异是否足够接近,可以视为相等
}

\mySamllsection{建议}

\begin{itemize}
\item
代数和编程是相关的,但并不等同。

\item
通过比较浮点值与任意小值的差来比较浮点值。
\end{itemize}
