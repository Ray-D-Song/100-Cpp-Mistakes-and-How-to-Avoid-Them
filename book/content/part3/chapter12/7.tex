这种错误会影响有效性和可读性,在某些情况下,可能会对正确性产生负面影响。客户端代码永远不必知道算法或函数如何工作的具体细节;相反,它应该直观地调用函数,只提供解决问题所需的数据。

\mySamllsection{问题}

考虑以递归方式实现二分搜索的情况。为了尽量减少性能问题,数据集合通过引用传递。起始和结束索引值会针对每个后续递归调用进行调整。

客户端代码调用递归函数并提供初始的起始和结束索引值。在这种情况下,客户端代码倾向于对索引值进行硬编码,因为它们必须确定函数调用的机制以及数据容器与该调用的关系。如果这样做,值很脆弱,如果在未适当改变索引值的情况下改变容器大小,将导致不正确的行为。

希望客户始终使用正确的起始和结束索引值。他们可能会猜测他们的理解是否错误,或者他们的经验是否有限。这种方法很危险。

\filename{清单12.13 公开实现细节的递归函数}

\begin{cpp}
bool bin_search(const std::vector<int>& values, int key, int start, int end)
{
  int mid = start + (end-start) / 2;
  if (values[mid] == key) // mention previous 'else' issue
    return true;
  else if (values[mid] < key)
    return bin_search(values, key, start+1, end);
  else
    return bin_search(values, key, start, end-1);
}
int main() {
  std::vector<int> values;
  for (int i = 0; i < 100; ++i)
    values.push_back(i);
  std::cout << bin_search(values, 55, 0, 99) << '\n'; // 1
}
\end{cpp}

{\footnotesize
注释1:客户端代码必须了解起始和结束索引值。信息量太大了!
}

\mySamllsection{分析}

编写 main 函数的开发人员得到的索引值是正确的。不过,尚不清楚他们是否理解结束索引必须是容器大小 (100) 或最大索引 (99)。在略 有不同的情况下,可能会使用结束索引 100。要更正索引值,开发人员必须了解递归搜索如何使用该值。此信息负载会对有效性产生负面影响。

理想情况下,搜索函数的开发人员应该对其进行编码,以便客户端无需了解索引值或函数的实现方式。在类开发中,封装会抽象这些细节,并提供一个最小的、希望是直观的界面。函数也应如此。

\mySamllsection{解决}

可以通过创建辅助函数来封装递归函数的功能。此函数采用搜索函数的名称,但消除了有关索引值或其他实现细节的任何知识。函数重载允许名称保持不变。递归函数名称无需更改即可表明它是递归的(除非这对开发人员来说是有用的文档细节)。辅助函数(可能是由递归函数的开发人员编写的)确定有关起始和结束索引值的详细信息。

辅助函数的另一个好处是,结束索引是根据容器计算的,因此消除了 对该值进行硬编码的诱惑。如何调用递归函数的实现细节被抽象出来,客户端的接口被最小化为仅包含基本数据 - 值的容器和键。

清单 12.14 在客户端代码和递归函数之间引入了辅助函数,与类的公共方法非常相似。这样做的目的是尽量减少客户端开发人员的认知负担,并允许在以后需要时重新实现该函数。

\filename{清单12.14 从抽象的辅助函数调用的递归函数}

\begin{cpp}
bool bin_search(const std::vector<int>& values, int key, int start, int end) {
  int mid = start + (end-start) / 2;
  if (values[mid] == key)
    return true;
  if (values[mid] < key)
    return bin_search(values, key, start+1, end);
  return bin_search(values, key, start, end-1);
}

bool bin_search(const std::vector<int>& values, int key) { // 1
  return bin_search(values, key, 0, values.size()-1);
}

int main() {
  std::vector<int> values;
  for (int i = 0; i < 100; ++i)
    values.push_back(i);
  std::cout << bin_search(values, 55) << '\n'; // 2
}
\end{cpp}

{\footnotesize
注释1:在客户端和递归函数调用之间进行转换

注释2:使调用尽可能简单
}

\mySamllsection{建议}

\begin{itemize}
\item
尽量减少函数调用者需要提供的数据量;坚持只提供必要的数据。

\item
寻找辅助函数可以最小化界面并简化调用函数的情况。
\end{itemize}









