这个错误主要针对的是有效性和可读性。许多教科书倾向于展示完整的 if/else 结构,而不描述它们何时会掩盖代码的意图。

\mySamllsection{问题}

当开发人员学习新的语言技术时,学习的风格经常会融入到其使用中。以下代码是确定一组值的算术平均值的简单示例。C++ 语言提供了 使用 if/else 构造从一组选项(在本例中为三个)中进行选择的方法。第一个选项是没有值,第二个选项是一个值,第三个选项是多个值。每个选项必须以不同的方式处理;因此,使用演示的结构是很自然的。但是,读者必须在这些选项之间做出决定,并确定哪些是短路,哪些是主要计算。

短路选项是直接计算回报的选项(这些是递归函数中的基本情况) 。将这些选项与计算混合在一起通常不如此代码所示那么明显。

\filename{清单12.11 使用if/else链}

\begin{cpp}
double mean(const std::vector<double>& values) {
  unsigned int size = values.size();
  if (size == 0) // 1
    throw std::invalid_argument("no values to average");
  else if (size() == 1)
    return values[0];
  else {
    double sum = 0.0;
    for (int i = 0; i < size; ++i)
      sum += values[i];
    return sum/size;
  }
}
int main() {
  std::vector<double> values;
  values.push_back(3.14159);
  values.push_back(2.71828);
  std::cout << mean(values) << '\n';
  return 0;
}
\end{cpp}

{\footnotesize
注释1:一个 if/else 链,它使计算的返回条件变得模糊
}

\mySamllsection{分析}

返回直接计算结果的两个选项(好吧,抛出异常可能不是计算,但还是接受它)与计算选项暗中混合在一起。逻辑从无到一,再到多个值。不需要按照这种逻辑顺序编写,并且在代码更广泛的情况下,计算可能会完全被混合所掩盖。最好采用一种更好的方法来理解单独的选项并使其清晰。

\mySamllsection{解决}

清单 12.12 对清单 12.11 中的代码进行了轻微修改,但这样做使结果更容易理解。处理了两种短路情况 — 要么它们是 true,并且处理了 throw 或 return,要么不处理。如果两种情况都是 false,则执行 主计算情况。

这两个例子在视觉上有所区别。前者使用了更多的缩进,这总是会给短期记忆带来负担。

后者省去了缩进,更有力地阐明了如果前两种情况不成立,它们很快就会被遗忘。计算与 if 关键字的缩进级别相同,清楚地表明如果控制流到达那么远,此代码将被执行。减少 else 关键字的数量可让读者更 快地省去短路情况并专注于主要计算任务。

\filename{清单12.12 使用短路逻辑}

\begin{cpp}
double mean(const std::vector<double>& values) {
  unsigned int size = values.size();
  if (size == 0) // 1
    throw std::invalid_argument("no values to average");
  if (size() == 1) // 1
    return values[0];
  double sum = 0.0;
  for (int i = 0; i < size; ++i)
    sum += values[i];
  return sum/size;
}

int main() {
  std::vector<double> values;
  values.push_back(3.14159);
  values.push_back(2.71828);
  std::cout << mean(values) << '\n';
  return 0;
}
\end{cpp}

{\footnotesize
注释1:返回条件的短路评估
}

\mySamllsection{建议}

\begin{itemize}
\item
组织一个函数,首先测试无效或失败的情况,如果通过,则将主要计算放在后面。

\item
尽可能减少缩进级别。

\item
只要有意义就使用短路情况;这些短路情况会直接计算或抛出,并且在尝试理解主代码时很快就会被遗忘。
\end{itemize}







