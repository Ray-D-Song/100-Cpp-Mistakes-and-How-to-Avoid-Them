这个错误主要针对的是效率和可读性。很多开发者更喜欢使用if 语句,而不是强大的三元运算符。

\mySamllsection{问题}

教科书会教授 if/else 条件结构,并在代码示例中使用它们。通常,三元运算符很少使用,但它的包含通常是偶然的。需要包含有关 if/ else 语句和三元表达式的区别和使用的详细讨论。语句不返回值; 表达式总是返回值。

考虑以下代码,其中谓词 isEven 决定从 if 测试中遵循哪条路径。真路径执行 if 关键字后面的语句,假路径执行 else 关键字后面的语句(如果存在)。

\filename{清单12.9 过于冗长的条件代码}

\begin{cpp}
bool isEven(int n) {
  return n % 2 == 0;
}
int main() {
  int n = 42;
  if (isEven(n)) // 1
    std::cout << n << " is even\n";
  else
    std::cout << n << " is odd\n";
  ++n;
  if (isEven(n))
    std::cout << n << " is even\n";
  else
    std::cout << n << " is odd\n";
  return 0;
}
\end{cpp}

{\footnotesize
注释1:if/else 语句很冗长
}

\mySamllsection{分析}

代码确定值的均匀性并输出结果。由于使用 if 语句进行此确定,因此输出发生在 true 或 false 路径上,或者该路径必须使用测试结果设置局部变量。如果选择第二个选项,则输出使用局部变量的内容。if/ else 语句无法返回值,但只能计算值。计算代码通常在真路径和假路 径上重复,使代码冗长且阅读起来稍微复杂一些。

\mySamllsection{解决}

if/else 结构无法返回值,因为它是一个语句。三元运算符本质上是一个 if/else 结构,是一个可以产生值的表达式。使用运算符作为表达式意味着可以直接计算值,而无需分配局部变量或重复代码。

清单 12.10 显示了运算符返回值的用法。表达式应该括在括号中,以防止编译器混淆。同样重要的是,括号将表达式与其周围的代码区分开来,使其更易于阅读。

本讨论并不是说所有 if/else 结构都应强制放入表达式中,而是说,在许多情况下,这种方法很合适。输出语句尤其受益于使用三元运算符。

\filename{清单12.10 最小化条件代码}

\begin{cpp}
bool isEven(int n) {
  return n % 2 == 0;
}
int main() {
  int n = 42;
  std::cout << n << " is " << (isEven(n) ? "even" : "odd") << '\n'; // 1
  ++n;
  std::cout << n << " is " << (isEven(n) ? "even" : "odd") << '\n';
  return 0;
}
\end{cpp}

{\footnotesize
注释1:使用三元运算符最小化 if/else 逻辑
}

\mySamllsection{建议}

\begin{itemize}
\item
使用三元运算符直接计算值,消除局部变量和一些重复代码;此示例代码并不理想,因为最好将其放入函数中。

\item
查找可以从使用表达式形式中受益的输出语句。
\end{itemize}
